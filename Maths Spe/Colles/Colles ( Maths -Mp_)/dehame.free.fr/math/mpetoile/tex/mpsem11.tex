\documentclass{article}
\begin{document}

\parindent=-8mm\leftskip=8mm
\def\new{\par\hskip 8.3mm}
\def\sect{\par\quad}
\hsize=147mm  \vsize=230mm
\hoffset=-10mm\voffset=0mm

\everymath{\displaystyle}       % �vite le textstyle en mode
                                % math�matique

\font\itbf=cmbxti10

\let\dis=\displaystyle          %raccourci
\let\eps=\varepsilon            %raccourci
\let\vs=\vskip                  %raccourci


\frenchspacing

\let\ie=\leq
\let\se=\geq



\font\pc=cmcsc10 % petites capitales (aussi cmtcsc10)

\def\tp{\raise .2em\hbox{${}^{\hbox{\seveni t}}\!$}}%



\font\info=cmtt10




%%%%%%%%%%%%%%%%% polices grasses math�matiques %%%%%%%%%%%%
\font\tenbi=cmmib10 % bold math italic
\font\sevenbi=cmmi7% scaled 700
\font\fivebi=cmmi5 %scaled 500
\font\tenbsy=cmbsy10 % bold math symbols
\font\sevenbsy=cmsy7% scaled 700
\font\fivebsy=cmsy5% scaled 500
%%%%%%%%%%%%%%% polices de presentation %%%%%%%%%%%%%%%%%
\font\titlefont=cmbx10 at 20.73pt
\font\chapfont=cmbx12
\font\secfont=cmbx12
\font\headfont=cmr7
\font\itheadfont=cmti7% at 6.66pt



% personnel Monasse
\def\euler{\cal}
\def\goth{\cal}
\def\phi{\varphi}
\def\epsilon{\varepsilon}

%%%%%%%%%%%%%%%%%%%%  tableaux de variations %%%%%%%%%%%%%%%%%%%%%%%
% petite macro d'�criture de tableaux de variations
% syntaxe:
%         \variations{t    && ... & ... & .......\cr
%                     f(t) && ... & ... & ...... \cr
%
%etc...........}
% � l'int�rieur de cette macro on peut utiliser les macros
% \croit (la fonction est croissante),
% \decroit (la fonction est d�croissante),
% \nondef (la fonction est non d�finie)
% si l'on termine la derni�re ligne par \cr, un trait est tir� en dessous
% sinon elle est laiss�e sans trait
%%%%%%%%%%%%%%%%%%%%%%%%%%%%%%%%%%%%%%%%%%%%%%%%%%%%%%%%%%%%%%%%%%%

\def\variations#1{\par\medskip\centerline{\vbox{{\offinterlineskip
            \def\decroit{\searrow}
    \def\croit{\nearrow}
    \def\nondef{\parallel}
    \def\tableskip{\omit& height 4pt & \omit \endline}
    % \everycr={\noalign{\hrule}}
            \def\cr{\endline\tableskip\noalign{\hrule}\tableskip}
    \halign{
             \tabskip=.7em plus 1em
             \hfil\strut $##$\hfil &\vrule ##
              && \hfil $##$ \hfil \endline
              #1\crcr
           }
 }}}\medskip}   % MONASSE

%%%%%%%%%%%%%%%%%%%%%%%% NRZCQ %%%%%%%%%%%%%%%%%%%%%%%%%%%%
\def\nmat{{\rm I\kern-0.5mm N}}  % MONASSE
\def\rmat{{\rm I\kern-0.6mm R}}  % MONASSE
\def\cmat{{\rm C\kern-1.7mm\vrule height 6.2pt depth 0pt\enskip}}  % MONASSE
\def\zmat{\mathop{\raise 0.1mm\hbox{\bf Z}}\nolimits}
\def\qmat{{\rm Q\kern-1.8mm\vrule height 6.5pt depth 0pt\enskip}}  % MONASSE
\def\dmat{{\rm I\kern-0.6mm D}}
\def\lmat{{\rm I\kern-0.6mm L}}
\def\kmat{{\rm I\kern-0.7mm K}}

%___________intervalles d'entiers______________
\def\[ent{[\hskip -1.5pt [}
\def\]ent{]\hskip -1.5pt ]}
\def\rent{{\bf ]}\hskip -2pt {\bf ]}}
\def\lent{{\bf [}\hskip -2pt {\bf [}}

%_____def de combinaison
\def\comb{\mathop{\hbox{\large C}}\nolimits}

%%%%%%%%%%%%%%%%%%%%%%% Alg�bre lin�aire %%%%%%%%%%%%%%%%%%%%%
%________image_______
\def\im{\mathop{\rm Im}\nolimits}
%________determinant_______
\def\det{\mathop{\rm det}\nolimits}  % MONASSE
\def\Det{\mathop{\rm Det}\nolimits}
\def\diag{\mathop{\rm diag}\nolimits}
%________rang_______
\def\rg{\mathop{\rm rg}\nolimits}
%________id_______
\def\id{\mathop{\rm id}\nolimits}
\def\tr{\mathop{\rm tr}\nolimits}
\def\Id{\mathop{\rm Id}\nolimits}
\def\Ker{\mathop{\rm Ker}\nolimits}
\def\bary{\mathop{\rm bar}\nolimits}
\def\card{\mathop{\rm card}\nolimits}
\def\Card{\mathop{\rm Card}\nolimits}
\def\grad{\mathop{\rm grad}\nolimits}
\def\Vect{\mathop{\rm Vect}\nolimits}
\def\Log{\mathop{\rm Log}\nolimits}

%________GL_______
\def\GLR#1{{\rm GL}_{#1}(\rmat)}  % MONASSE
\def\GLC#1{{\rm GL}_{#1}(\cmat)}  % MONASSE
\def\GLK#1#2{{\rm GL}_{#1}(#2)}  % MONASSE
\def\SO{\mathop{\rm SO}\nolimits}
\def\SDP#1{{\cal S}_{#1}^{++}}
%________spectre_______
\def\Sp{\mathop{\rm Sp}\nolimits}
%_________ transpos�e ________
%\def\t{\raise .2em\hbox{${}^{\hbox{\seveni t}}\!$}}
\def\t{\,{}^t\!\!}

%_______M gothL_______
\def\MR#1{{\cal M}_{#1}(\rmat)}  % MONASSE
\def\MC#1{{\cal M}_{#1}(\cmat)}  % MONASSE
\def\MK#1{{\cal M}_{#1}(\kmat)}  % MONASSE

%________Complexes_________ % MONASSE
\def\Re{\mathop{\rm Re}\nolimits}
\def\Im{\mathop{\rm Im}\nolimits}

%_______cal L_______
\def\L{{\euler L}}

%%%%%%%%%%%%%%%%%%%%%%%%% fonctions classiques %%%%%%%%%%%%%%%%%%%%%%
%________cotg_______
\def\cotan{\mathop{\rm cotan}\nolimits}
\def\cotg{\mathop{\rm cotg}\nolimits}
\def\tg{\mathop{\rm tg}\nolimits}
%________th_______
\def\tanh{\mathop{\rm th}\nolimits}
\def\th{\mathop{\rm th}\nolimits}
%________sh_______
\def\sinh{\mathop{\rm sh}\nolimits}
\def\sh{\mathop{\rm sh}\nolimits}
%________ch_______
\def\cosh{\mathop{\rm ch}\nolimits}
\def\ch{\mathop{\rm ch}\nolimits}
%________log_______
\def\log{\mathop{\rm log}\nolimits}
\def\sgn{\mathop{\rm sgn}\nolimits}

\def\Arcsin{\mathop{\rm Arcsin}\nolimits}   % CLENET
\def\Arccos{\mathop{\rm Arccos}\nolimits}   % CLENET
\def\Arctan{\mathop{\rm Arctan}\nolimits}   % CLENET
\def\Argsh{\mathop{\rm Argsh}\nolimits}     % CLENET
\def\Argch{\mathop{\rm Argch}\nolimits}     % CLENET
\def\Argth{\mathop{\rm Argth}\nolimits}     % CLENET
\def\Arccotan{\mathop{\rm Arccotan}\nolimits}
\def\coth{\mathop{\rm coth}\nolimits}
\def\Argcoth{\mathop{\rm Argcoth}\nolimits}
\def\E{\mathop{\rm E}\nolimits}
\def\C{\mathop{\rm C}\nolimits}

\def\build#1_#2^#3{\mathrel{\mathop{\kern 0pt#1}\limits_{#2}^{#3}}} %CLENET

%________classe C_________
\def\C{{\cal C}}
%____________suites et s�ries_____________________
\def\suiteN #1#2{(#1 _#2)_{#2\in \nmat }}  % MONASSE
\def\suite #1#2#3{(#1 _#2)_{#2\ge#3 }}  % MONASSE
\def\serieN #1#2{\sum_{#2\in \nmat } #1_#2}  % MONASSE
\def\serie #1#2#3{\sum_{#2\ge #3} #1_#2}  % MONASSE

%___________norme_________________________
\def\norme#1{\|{#1}\|}  % MONASSE
\def\bignorme#1{\left|\hskip-0.9pt\left|{#1}\right|\hskip-0.9pt\right|}

%____________vide (perso)_________________
\def\vide{\hbox{\O }}
%____________partie
\def\P{{\cal P}}

%%%%%%%%%%%%commandes abr�g�es%%%%%%%%%%%%%%%%%%%%%%%
\let\lam=\lambda
\let\ddd=\partial
\def\bsk{\vspace{12pt}\par}
\def\msk{\vspace{6pt}\par}
\def\ssk{\vspace{3pt}\par}
\let\noi=\noindent
\let\eps=\varepsilon
\let\ffi=\varphi
\let\vers=\rightarrow
\let\srev=\leftarrow
\let\impl=\Longrightarrow
\let\tst=\textstyle
\let\dst=\displaystyle
\let\sst=\scriptstyle
\let\ssst=\scriptscriptstyle
\let\divise=\mid
\let\a=\forall
\let\e=\exists
\let\s=\over
\def\vect#1{\overrightarrow{\vphantom{b}#1}}
\let\ov=\overline
\def\eu{\e !}
\def\pn{\par\noi}
\def\pss{\par\ssk}
\def\pms{\par\msk}
\def\pbs{\par\bsk}
\def\pbn{\bsk\noi}
\def\pmn{\msk\noi}
\def\psn{\ssk\noi}
\def\nmsk{\noalign{\msk}}
\def\nssk{\noalign{\ssk}}
\def\equi_#1{\build\sim_#1^{}}
\def\lp{\left(}
\def\rp{\right)}
\def\lc{\left[}
\def\rc{\right]}
\def\lci{\left]}
\def\rci{\right[}
\def\Lim#1#2{\lim_{#1\vers#2}}
\def\Equi#1#2{\equi_{#1\vers#2}}
\def\Vers#1#2{\quad\build\longrightarrow_{#1\vers#2}^{}\quad}
\def\Limg#1#2{\lim_{#1\vers#2\atop#1<#2}}
\def\Limd#1#2{\lim_{#1\vers#2\atop#1>#2}}
\def\lims#1{\Lim{n}{+\infty}#1_n}
\def\cl#1{\par\centerline{#1}}
\def\cls#1{\pss\centerline{#1}}
\def\clm#1{\pms\centerline{#1}}
\def\clb#1{\pbs\centerline{#1}}
\def\cad{\rm c'est-�-dire}
\def\ssi{\it si et seulement si}
\def\lac{\left\{}
\def\rac{\right\}}
\def\ii{+\infty}
\def\eg{\rm par exemple}
\def\vv{\vskip -2mm}
\def\vvv{\vskip -3mm}
\def\vvvv{\vskip -4mm}
\def\union{\;\cup\;}
\def\inter{\;\cap\;}
\def\sur{\above .2pt}
\def\tvi{\vrule height 12pt depth 5pt width 0pt}
\def\tv{\vrule height 8pt depth 5pt width 1pt}
\def\rplus{\rmat_+}
\def\rpe{\rmat_+^*}
\def\rdeux{\rmat^2}
\def\rtrois{\rmat^3}
\def\net{\nmat^*}
\def\ret{\rmat^*}
\def\cet{\cmat^*}
\def\rbar{\ov{\rmat}}
\def\deter#1{\left|\matrix{#1}\right|}
\def\intd{\int\!\!\!\int}
\def\intt{\int\!\!\!\int\!\!\!\int}
\def\ce{{\cal C}}
\def\ceun{{\cal C}^1}
\def\cedeux{{\cal C}^2}
\def\ceinf{{\cal C}^{\infty}}
\def\zz#1{\;{\raise 1mm\hbox{$\zmat$}}\!\!\Bigm/{\raise -2mm\hbox{$\!\!\!\!#1\zmat$}}}
\def\interieur#1{{\buildrel\circ\over #1}}
%%%%%%%%%%%% c'est la fin %%%%%%%%%%%%%%%%%%%%%%%%%%%
\catcode`@=12 % at signs are no longer letters
\catcode`\�=\active
\def�{\'e}
\catcode`\�=\active
\def�{\`e}
\catcode`\�=\active
\def�{\^e}
\catcode`\�=\active
\def�{\`a}
\catcode`\�=\active
\def�{\`u}
\catcode`\�=\active
\def�{\^u}
\catcode`\�=\active
\def�{\^a}
\catcode`\"=\active
\def"{\^o}
\catcode`\�=\active
\def�{\"e}
\catcode`\�=\active
\def�{\"\i}
\catcode`\�=\active
\def�{\"u}
\catcode`\�=\active
\def�{\c c}
\catcode`\�=\active
\def�{\^\i}


\def\boxit#1#2{\setbox1=\hbox{\kern#1{#2}\kern#1}%
\dimen1=\ht1 \advance\dimen1 by #1 \dimen2=\dp1 \advance\dimen2 by #1
\setbox1=\hbox{\vrule height\dimen1 depth\dimen2\box1\vrule}%
\setbox1=\vbox{\hrule\box1\hrule}%
\advance\dimen1 by .4pt \ht1=\dimen1
\advance\dimen2 by .4pt \dp1=\dimen2 \box1\relax}


\catcode`\@=11
\def\system#1{\left\{\null\,\vcenter{\openup1\jot\m@th
\ialign{\strut\hfil$##$&$##$\hfil&&\enspace$##$\enspace&
\hfil$##$&$##$\hfil\crcr#1\crcr}}\right.}
\catcode`\@=12
\pagestyle{empty}





\overfullrule=0mm


\cl{{\bf SEMAINE 10}}\msk
\cl{{\bf INT\'EGRALE SUR UN SEGMENT. FONCTIONS INT\'EGRABLES}}
\bsk

{\bf EXERCICE 1 :}\msk
Soit $x=(x_n)_{n\in\net}$ une suite \`a valeurs dans $[0,1]$. Pour tout
 entier naturel non nul $n$ et toute partie $A$ de $[0,1]$, on note $N(A,n)$
 le nombre d'indices $k\in\[ent1,n\]ent$ tels que $x_k\in A$.\ssk
 On dit que la suite $x$ est {\bf \'equir\'epartie} si, pour tous r\'eels $a$
 et $b$  v\'erifiant $0\le a<b\le 1$, on a\vv
 $$\Lim{n}{\ii}{N\big(]a,b[,n\big)\s n}=b-a\;.$$\par
 Montrer que la suite $x$ est \'equir\'epartie si et seulement si, pour toute
 fonction $f:[0,1]\vers\rmat$ continue par morceaux (c.p.m.), on a\vv
 $$\Lim{n}{\ii}{1\s n}\sum_{k=1}^nf(x_k)=\int_0^1f\;.\eqno\hbox{\bf (*)}$$
 
\msk
\cl{- - - - - - - - - - - - - - - - - - - - - - - - - - - - - -}
\msk

 
Remarquons d'abord que, pour tout $n\in\net$ et pour toute partie $A$ de
$[0,1]$, on a\break $N(A,n)=\sum_{k=1}^n\chi_A(x_k)$, o\`u $\chi_A$ est la
fonction caract\'eristique de $A$.\par
Pour tout entier naturel non nul $n$ et toute fonction $f$ continue par morceaux sur
$[0,1]$, posons $\;S_n(f)={1\s n}\>\sum_{k=1}^nf(x_k)$.
\ssk
$\bullet$ Supposons la relation {\bf (*)} vraie pour toute fonction $f$
continue par morceaux sur $[0,1]$~; avec $f=\chi_{]a,b[}$ (fonction en
escalier), on obtient\vv
$$\Lim{n}{\ii}S_n(f)=\Lim{n}{\ii}{N\big(]a,b[,n\big)\s n}=
  \int_0^1\chi_{]a,b[}=b-a\;,$$
d'o\`u la propri\'et\'e d'\'equir\'epartition.
\msk
$\bullet$ R\'eciproquement, supposons la suite \'equir\'epartie.\pn
Soit d'abord $f:[0,1]\vers\rmat$ en escalier, soit $(0=a_0,a_1,\cdots,a_m=1)$
une subdivision de $[0,1]$ subordonn\'ee \`a $f$, soit $\lam_j$ la valeur
 (constante) de $f$ sur $]a_j,a_{j+1}[\;$ ($0\le j\le m-1$). On a alors\vv
 $$\int_0^1f=\sum_{j=0}^{m-1}(a_{j+1}-a_j)\>\lam_j$$
 et, comme $f=\sum_{j=0}^{m-1}\lam_j\>\chi_{]a_j,a_{j+1}[}+\sum_{j=0}^m
 f(a_j)\>\chi_{\{a_j\}}$, on a
 $$S_n(f)=\sum_{j=0}^{m-1}\lam_j\>{N\big(]a_j,a_{j+1}[,
   n\big)\s n}+\sum_{j=0}^mf(a_j)\>{N\big(\{a_j\},n\big)\s n}\;.$$
 Or, par hypoth\`ese, $\Lim{n}{\ii}{N\big(]a_j,a_{j+1}[,n\big)\s n}=a_{j+1}
 -a_j$ pour tout $j\in\[ent0,m-1\]ent$.\pn
 Il reste \`a prouver que, pour tout
 $a\in[0,1]$, $\Lim{n}{\ii}{N\big(\{a\},n\big)\s n}=0$.
 Pour cela, il suffit de constater que, si $a\in\;]0,1[$ ,\vv
 $$N\big(\{a\},n\big)=N\big(]0,1[,n\big)-N\big(]0,a[,n\big)-N\big(]a,1[,n
   \big)$$
 et, pour $a=0$ et $a=1$,\vv
 $$0\le N\big(\{0\},n\big)+N\big(\{1\},n\big)=n-N\big(]0,1[,n\big)\;.$$
 Soit maintenant $f:[0,1]\vers\rmat$, continue par morceaux. On sait que $f$
 est limite uniforme d'une suite de fonctions en escalier sur $[0,1]$.
 Donc, si on se donne $\eps>0$, on peut trouver $g:[0,1]\vers\rmat$ en
 escalier telle que $\|f-g\|_{\infty}\le{\eps\s3}$. D'apr\`es ce qui pr\'ec\`ede, on peut
 trouver un entier $N$ tel que, pour tout $n\ge N$, on ait
 $\;\left|S_n(g)-\int_0^1g\right|\le{\eps\s3}$. Or, pour tout $n$,
 $$\left|S_n(f)-\int_0^1f\right|\le|S_n(f)-S_n(g)|+
   \left|S_n(g)-\int_0^1g\right|+\left|\int_0^1g-\int_0^1f\right|\;.$$
 De $\|f-g\|_{\infty}\le{\eps\s3}$, on d�duit que $\;\left|\int_0^1g-
 \int_0^1f\right|\le\int_0^1|g-f|\le{\eps\s3}\;$ et
 $$|S_n(f)-S_n(g)|={1\s n}\left|\sum_{k=1}^n\big(f(x_k)-g(x_k)\big)\right|
   \le{1\s n}\cdot n\cdot{\eps\s3}={\eps\s3}\;,$$
 donc, pour $n\ge N$, on a $\;\left|S_n(f)-\int_0^1f\right|\le\eps$, ce
 qui prouve que $\Lim{n}{\ii}S_n(f)=\int_0^1f$.

\bsk
\hrule
\bsk

{\bf EXERCICE 2 :}\msk
Soit $f$ une bijection continue et strictement croissante de $\rplus$ vers
 lui-m\^eme.\msk
{\bf  a.} Montrer que, pour tout $a\in\rplus$, $\;\int_0^af+\int_0^{f(a)}f^{-1}=
   a\>f(a)$.\ssk
   {\it On commencera par traiter le cas o\`u $f$ est de classe $\ceun$. Dans
   le cas g\'en\'eral, on montrera la d\'erivabilit\'e de la fonction $\ffi:a
   \mapsto\int_0^{f(a)}f^{-1}-a\>f(a)$.}\msk
{\bf  b.} Montrer que, pour tous $(a,b)\in\rplus^2$, $\;\int_0^af+\int_0^bf^{-1}\ge
   ab$.

\msk
\cl{- - - - - - - - - - - - - - - - - - - - - - - - - - - - - -}
\msk


{\bf a.} Si $f$ est suppos\'ee de classe $\ceun$, une simple d\'erivation par
rapport \`a la variable $a$ permet de conclure.\ssk\sect
 Dans le cas g\'en\'eral, essayons aussi de d\'eriver par rapport \`a la variable $a$.
 Le terme $\int_0^af$ est d\'erivable, de d\'eriv\'ee $f(a)$. Les termes
 $\int_0^{f(a)}f^{-1}$ et $af(a)$, pris s\'epar\'ement,
 ne sont pas, en g\'en\'eral, d\'erivables, mais \'etudions leur diff\'erence
 $\ffi(a)$ ({\it cf.} \'enonc\'e) et formons un taux
 d'accroissement.\vv
 $$\ffi(a+h)-\ffi(a)=\int_{f(a)}^{f(a+h)}f^{-1}-(a+h)\>f(a+h)+a\>f(a)\;.$$\sect
 Pour $h>0$, il est facile d'\'ecrire un encadrement de l'int\'egrale~:\vv
 $$a\>\big(f(a+h)-f(a)\big)\le\int_{f(a)}^{f(a+h)}f^{-1}\le
   (a+h)\>\big(f(a+h)-f(a)\big)\;,$$
 d'o\`u l'on tire sans difficult\'e\vv
 $$-f(a+h)\le{\ffi(a+h)-\ffi(a)\s h}\le-f(a)\;,\quad\hbox{donc}\quad
   \Lim{h}{0^+}{\ffi(a+h)-\ffi(a)\s h}=-f(a)$$
 car $f$ est continue au point $a$. On obtient de m\^eme, pour $h<0$,\vv
 $$-f(a)\le{\ffi(a+h)-\ffi(a)\s h}\le-f(a+h)\;,\quad\hbox{donc}\quad
   \Lim{h}{0^-}{\ffi(a+h)-\ffi(a)\s h}=-f(a).$$\sect
 Finalement, $\ffi$ est d\'erivable sur $\rplus$, de d\'eriv\'ee $-f$, donc
 la fonction \break $g:a\mapsto\int_0^af+\int_0^{f(a)}f^{-1}-a\>f(a)\;$
 est d\'erivable, de d\'eriv\'ee nulle. Elle est donc cons\-tante sur $\rplus$, et
 $g(0)=0$, ce qui r\'epond \`a la question.

\msk
 {\bf b.} Fixons $a\ge0$. Pour tout $b\in\rplus$, posons
 $\;\psi(b)=\int_0^af+\int_0^bf^{-1}-ab$. La fonction $\psi$ est
 d\'erivable, de d\'eriv\'ee $\psi'(b)=f^{-1}(b)-a$. La croissance
 stricte des fonctions $f$ et $f^{-1}$ permet d'affirmer que\vv
 $$\psi'(b)>0\iff b>f(a)\;.$$
 On en d\'eduit sans peine que $\psi(b)$ est minimal lorsque
 $b=f(a)$ et sa valeur est alors $\psi\big(f(a)\big)=g(a)=0$ d'apr\`es le
 {\bf a.}, ce qui prouve l'in\'egalit\'e \`a d\'emontrer.

\bsk
\hrule
\bsk

{\bf EXERCICE 3 :}\msk
Pour tout $f\in {\cal E}=\ce\big([a,b],\rmat\big)$ et tout r\'eel non nul $p$, on pose $\;\mu_p(f)=\lp{1\s b-a}\>\int_a^b|f|^p\rp^{\sst1\sur\sst p}\;$~:
{\bf moyenne d'ordre} $p$ de la fonction $|f|$ sur $[a,b]$.\msk
{\bf 1.} Calculer $\Lim{p}{\ii}\mu_p(f)$.\msk
{\bf 2.} Montrer que, si $f\in {\cal E}=\ce\big([a,b],\rmat\big)$ ne s'annule
 pas sur $[a,b]$, alors\vv
 $$\Lim{p}{0}\mu_p(f)=\exp\lp{1\s b-a}\>\int_a^b\ln |f(x)|\;dx\rp\;.$$

\msk
\cl{- - - - - - - - - - - - - - - - - - - - - - - - - - - - - - }
\msk 

{\bf 1.} Si $f=0$, alors $\mu_p(f)=0$ pour tout $p$. Excluons d\'esormais ce cas.
\msk\sect
Soit $\;M=\max_{[a,b]}|f|>0$.
Soit $\eps>0$. Alors, par continuit\'e, il existe un segment $[\alpha,\beta]$ de longueur non nulle sur lequel $\;|f|\ge(1-\eps)M$.\ssk\new
Alors, pour tout $p>0$,\vv
$$ \int_a^b|f|^p\ge(\beta-\alpha)\cdot
   (1-\eps)^p\>M^p\;,\quad\hbox{d'o\`u}\quad \mu_p(f)\ge
   \Big({\beta-\alpha\s b-a}\Big)^{\sst1\sur\sst p}\>(1-\eps)\>M\;.$$
 Or, $\Lim{p}{\ii}\Big({\beta-\alpha\s b-a}\Big)^{\sst1\sur\sst p}=1$, donc
 $$\e P\in\rpe\quad\a p\in\rmat\qquad p\ge P\impl
   \mu_p(f)\ge(1-\eps)^2\>M\;.$$
Pour $p\ge P$, on a alors $\;(1-\eps)^2\>M\le \mu_p(f)\le M$. On en d\'eduit que\vv
 $$\Lim{p}{\ii}\mu_p(f)=M=\max_{[a,b]}|f|=N_{\infty}(f)\;.$$

\msk
{\bf b.} Il s'agit de montrer que $\Lim{p}{0}\mu_p(f)=e^J$, o\`u
 $J={1\s b-a}\int_a^b\ffi$, avec $\ffi=\ln|f|$. La fonction $|f|$ est continue et strictement
positive sur le segment $[a	,b]$, donc $\ffi$ est born\'ee  sur $[a,b]$~: $|\ffi(x)|\ie k$ sur $[a,b]$. La fonction $\;u\mapsto{e^u-1-u\s u^2}$ \'etant prolongeable par continuit\'e en z\'ero, il existe un r\'eel positif $M$ tel que $\;\a y\in
\ffi\big([a,b]\big)\quad\left|{e^y-1-y\s y^2}\right|\le M$.
\msk\sect
Alors, pour  $|p|\le1$, on a\vv
$$\left|\big(\mu_p(f)\big)^p-1-p\,J\right|={1\s b-a}\>\left|\int_a^b\lp
   e^{p\ffi}-1-p\,\ffi\rp\right|\le {M\>p^2\s b-a}\lp\int_a^b\ffi^2\rp\ie Mk^2p^2\;.$$
On peut donc \'ecrire $\;\mu_p(f)^p=e^{p\>\ln\mu_p(f)}=1+J p+O(p^2)$, d'o\`u $\;p\>\ln\mu_p(f)=Jp+O(p^2)$, donc $\Lim{p}{0}\ln\mu_p(f)=J\;$ et enfin $\Lim{p}{0}\mu_p(f)=e^J$, ce qu'il fallait d\'emontrer.
\msk
{\it Le nombre $\mu_0(f)=\Lim{p}{0}\mu_p(f)$ est la} {\bf moyenne g\'eom\'etrique} {\it de $|f|$ sur $[a,b]$}.

\bsk
\hrule
\bsk

{\bf EXERCICE 4 :}\msk
Soit $f:\rmat\vers\rmat$ une fonction de classe ${\cal C}^2$ telle que $\;\a t\in\rmat\quad f''(t)\se m$, o\`u $m$ est un r\'eel strictement positif.\msk
Montrer qu'il existe une constante $C$ ``universelle'' (c'est-\`a-dire ind\'ependante de $m$ et de la fonction $f$) telle que\vv
$$\a (a,b)\in\rmat^2\qquad\left|\int_a^be^{if(t)}\>dt\right|\ie{C\s\sqrt{m}}\;.$$
\ssk
{\it On pourra commencer par prouver que $f$ admet un minimum en un point $t_0$ et on majorera $\left|\int_{t_0}^{t_1}e^{if(t)}\>dt\right|$ ind\'ependamment du r\'eel $t_1$.}

\msk
{\it Source : Antoine CHAMBERT-LOIR, St\'efane FERMIGIER, Vincent MAILLOT, Exercices de math\'ematiques pour l'Agr\'egation, Analyse 1, \'Editions Masson, ISBN 2-225-84692-8}
\msk
\cl{- - - - - - - - - - - - - - - - - - - - - - - - - - - - - - -}
\msk

Soit $a$ un r\'eel. D'apr\`es le th\'eor\`eme des accroissements finis, on a, pour tout r\'eel $t$,\vv
$$f'(t)=f'(a)+(t-a)\>f''(c)\;,\qquad{\rm avec}\quad c\in[t,a]\;{\rm ou}\;[a,t]\;.$$\par
Donc,\ssk\sect
$\bullet$ pour $t\se a$, on a $f'(t)\se f'(a)+m(t-a)$~;\ssk\sect
$\bullet$ pour $t\ie a$, on a $f'(t)\ie f'(a)+m(t-a)$.\ssk
Il en r\'esulte que $\lim_{-\infty}f'=-\infty$ et $\lim_{\ii}f'=\ii$. Comme $f'$ est continue et strictement croissante, il existe un unique point $t_0$ tel que $f'(t_0)=0$ et il est clair que c'est le minimum global de la fonction $f$.\ssk
Quitte \`a translater $f$, on peut supposer que $t_0=0$ et chercher \`a majorer $\left|\int_0^ae^{if(t)}\>dt\right|$ ind\'ependamment du r\'eel $a$.
\msk
$\bullet$ Supposons $a>0$. Posons $\;I=\int_0^a e^{if(t)}\>dt$. Pour tout $\eps>0$, on peut \'ecrire, en int\'egrant par parties~:\vvvv
\begin{eqnarray*}
I & = & \int_0^{\eps}e^{if(t)}\>dt+\int_{\eps}^ae^{if(t)}\>dt\\
   & = & \int_0^{\eps}e^{if(t)}\>dt+ \int_{\eps}^a{1\s i\>f'(t)}\>\Big(i\>f'(t)\>e^{if(t)}\Big)\>dt\\
   & = & \int_0^{\eps}e^{if(t)}\>dt+{e^{if(a)}\s i\>f'(a)}-{e^{if(\eps)}\s i\>f'(\eps)}-i\>\int_{\eps}^ae^{if(t)}\>{f''(t)\s f'(t)^2}\>dt\;.
\end{eqnarray*}
Pour $\eps\se a$, on a bien s\^ur $\;|I|\ie\int_0^a|e^{if(t)}|\>dt= a\ie\eps\;$ et, si $\eps>a$, on a\vv
\begin{eqnarray*}
|I| & \ie & \eps+{1\s f'(a)}+{1\s f'(\eps)}+\int_{\eps}^a{f''(t)\s f'(t)^2}\>dt\\
     & =  & \eps+{1\s f'(a)}+{1\s f'(\eps)}-\Big[{1\s f'(t)}\Big]_{\eps}^a=\eps+{2\s f'(\eps)}\;.
\end{eqnarray*}
On a donc $\;|I|\ie\eps+{2\s f'(\eps)}\;$ pour tout $\eps>0$, mais $f'(\eps)\se m\eps$, donc $\;\a\eps>0\quad |I|\ie\eps+{2\s m\eps}$, puis $|I|\ie\min_{\eps>0}\Big(\eps+{2\s m\eps}\Big)$. Recherchons donc ce minimum~: une petite \'etude de fonction, laiss\'ee au lecteur, montre qu'il est atteint pour $\eps=\sqrt{2\s m}$ et vaut ${2\sqrt{2}\s\sqrt{m}}$. On a ainsi obtenu la majoration\vv
$$\a a\in\rplus\qquad\left|\int_0^a e^{if(t)}\>dt\right|\ie{2\sqrt{2}\s\sqrt{m}}\;.$$\par
$\bullet$ Si $a<0$, on applique ce qui pr\'ec\`ede \`a la fonction $\;g:t\mapsto g(t)=f(-t)\;$ qui v\'erifie aussi $g''\se m$ et $g'(0)=0$, donc\vv
$$\left|\int_0^ae^{if(t)}\>dt\right|=\left|\int_0^{-a}e^{ig(u)}\>du\right|\ie{2\sqrt{2}\s\sqrt{m}}\;.$$\par
$\bullet$ Pour tous $a$ et $b$ r\'eels, on a alors\vv
$$\left|\int_a^be^{if(t)}\>dt\right|\ie\left|\int_0^ae^{if(t)}\>dt\right|+\left|\int_0^be^{if(t)}\>dt\right|\ie{4\sqrt{2}\s\sqrt{m}}\;,$$
d'o\`u le r\'esultat demand\'e avec $C={4\s\sqrt{2}}$.

\bsk
\hrule
\bsk

{\bf EXERCICE 5 :}\msk
Soit ${\cal E}$ l'ensemble des fonctions $f:\rmat\vers\rmat$, de classe
 $\ceun$, telles que les fonctions $f'$ et $\;g:t\mapsto t f(t)\;$ soient de
 carr\'e int\'egrable sur $\rmat$.\msk
{\bf  a.} V\'erifier que ${\cal E}$ est un espace vectoriel.\msk
{\bf  b.} Soit $f\in {\cal E}$. Montrer que $f$ est de carr\'e int\'egrable et que\vv
 $$\int_{\rmat}f^2\le 2\;\lp\int_{\rmat}f'^2\rp^{1\sur2}\>
   \lp\int_{\rmat}g^2\rp^{1\sur2}\;.$$\par
{\bf  c.} \'Etudier les cas d'\'egalit\'e.
 
\msk
\cl{- - - - - - - - - - - - - - - - - - - - - - - - - - - - - - }
\msk


{\bf a.} Une fonction $f$ appartient \`a ${\cal E}$ si et seulement si les
 fonctions $f'$ et $g:t\mapsto t f(t)$ (qui d\'ependent lin\'eairement de $f$)
 appartiennent \`a l'espace vectoriel ${\rm L}^2(\rmat)$, donc ${\cal E}$
 est un s.e.v. de l'espace $\ceun(\rmat,\rmat)$.

\msk
{\bf b.} L'in\'egalit\'e $\;f(t)^2\le g(t)^2$, vraie pour $|t|\ge1$, montre
 que $f\in{\rm L}^2(\rmat)$.\psn
 Posons $\;I=\int_{\rmat}f'^2\;$ et
 $\;J=\int_{\rmat}g^2$. La fonction
 $\;t\mapsto t\>f(t)\>f'(t)\;$ est int\'egrable sur $\rmat$ comme produit de
 deux fonctions de carr\'e int\'egrable et l'in\'egalit\'e de Cauchy-Schwarz montre
 que\vv
 $$(f'|g)=\left|\int_{-\infty}^{\ii}t\>f(t)\>f'(t)\;dt\right|\le\sqrt{IJ}
   =N_2(f')\cdot N_2(g)\;.\eqno\hbox{\bf (*)}$$
 Or, une int\'egration par parties montre que, pour tous r\'eels $A$ et $B$,\vv
 $$\int_A^B f(t)^2\;dt=\big[t\>f(t)^2\big]_A^B\>-2\>\int_A^B t\>f(t)\>f'(t)\;dt\;.\eqno\hbox{\bf (**)}
   $$
 Les fonctions $f^2$ et $\;t\mapsto t\>f(t)\>f'(t)\;$ \'etant int\'egrables sur
 $\rmat$, l'expression $t\>f(t)^2\;$ admet une limite finie $l$ lorsque $t$
 tend vers $\ii$. Si on avait $l\not=0$, alors $\;g(t)^2=t^2\>f(t)^2\;$
 aurait une limite infinie en $\ii$, ce qui contredit l'int\'egrabilit\'e de
 $g^2$ sur $\rmat$.\pn
 On a donc $\;\Lim{t}{\ii}t\>f(t)^2=0\;$ et, de m\^eme, $\;\Lim{t}{-\infty}t\>
 f(t)^2=0$.\pn
 De {\bf (**)}, on d\'eduit alors que $\;\int_{\rmat} f^2=-2\int_{-\infty}^{\ii}
 t\>f(t)\>f'(t)\;dt$, d'o\`u, en vertu de {\bf (*)}, l'in\'egalit\'e demand\'ee.

 \msk
 {\bf c.} L'\'egalit\'e a lieu si et seulement s'il y a \'egalit\'e ci-dessus dans
 {\bf (*)} (Cauchy-Schwarz), c'est-\`a-dire si et seulement si les fonctions
 $f'$ et $g$ sont li\'ees, donc si $f'=0$ ($f$ constante) ou si $f$ est solution
 d'une \'equation diff\'erentielle de la forme\vv
 $$\lam x'+tx=0\eqno\hbox{\bf (E)}$$
 ($t$ : variable, $x$ : fonction inconnue).\pn
 Les fonctions constantes autres que la fonction nulle sont \`a exclure.
 Les solutions de {\bf (E)} sont les fonctions de la forme $t\mapsto C
 \cdot e^{{}^{-{\sst t^2\sur\sst2\lam}}}\;$ ($C\in\rmat$). On ne conservera
 que les valeurs strictement positives de $\lam$ (sinon, la fonction $f$
 n'appartient pas \`a ${\cal E}$). On obtient donc les fonctions\vv
 $$f:t\mapsto C\cdot e^{-\alpha t^2}\qquad\qquad(\alpha\in\rpe\;,\;C\in\rmat)
   \;.$$

\bsk
\hrule
\bsk

{\bf EXERCICE 6 :}\msk
Dans cet exercice, on admettra le lemme suivant ({\bf th\'eor\`eme de Hardy})~: {\it si $\sum_nu_n$ est une s\'erie convergente \`a termes positifs, alors la s\'erie de terme g\'en\'eral $v_n=\root n\of{u_0u_1\cdots u_{n-1}}$ est convergente}.\msk
Soit $f:\rplus\vers\rpe$ une fonction continue. On suppose que la fonction ${1\s f}$ est int\'egrable sur $\rplus$.\ssk Pour tout $x\in\rplus$, on pose $F(x)=\int_0^xf(t)\>dt$. On d\'efinit enfin $g$ par
$$g(0)={1\s f(0)}\qquad{\rm et}\qquad\a x\in\rpe\quad g(x)={x\s F(x)}\;.$$\par
{\bf 1.} Pour tout $n\in\nmat$, on pose $\;a_n=F(n+1)-F(n)$. D\'emontrer l'in\'egalit\'e $\;{1\s a_n}\ie\int_n^{n+1}{dt\s f(t)}$. Cons\'equence pour la s\'erie $\sum_n{1\s a_n}$~?\msk
{\bf 2.} Montrer que la s\'erie $\sum_ng(n)$ est convergente.\msk
{\bf 3.} En d\'eduire que la fonction $g$ est int\'egrable sur $\rplus$.\msk
{\bf 4.} S'il reste du temps... d\'emontrer le th\'eor\`eme de Hardy~: {\it pour cela, on pourra \'ecrire\break $\prod_{k=0}^{n-1}u_k={1\s (n+1)^n}\>\prod_{k=0}^{n-1}\lp{(k+2)^{k+1}\s(k+1)^k}\;u_k\rp$}.

\msk
{\it Source : Jean-Marie ARNAUDI\`ES, L'int\'egrale de Lebesgue sur la droite, \'Editions Vuibert, ISBN 2-7117-8904-7}

\msk
\cl{- - - - - - - - - - - - - - - - - - - - - - - - - - - - - -}
\msk

{\bf 1.} On a $a_n=\int_n^{n+1}f(t)\>dt$\quad ($a_n$ est strictement positif) et l'in\'egalit\'e de Cauchy-Schwarz donne
$$a_n\lp\int_n^{n+1}{dt\s f(t)}\rp=\lp\int_n^{n+1}f(t)\>dt\rp\lp\int_n^{n+1}{dt\s f(t)}\rp\se\lp\int_n^{n+1}dt\rp^2=1\;,$$
d'o\`u l'in\'egalit\'e voulue. La fonction ${1\s f}$ \'etant int\'egrable sur $[0,\ii[$, la s\'erie de terme g\'en\'eral $\;\int_n^{n+1}{dt\s f(t)}\;$ converge, il en est donc de m\^eme de la s\'erie $\sum_n{1\s a_n}$.

\msk
{\bf 2.} On a $F(x)>0$ pour $x>0$, donc $g$ est bien d\'efinie sur $\rplus$, elle est par ailleurs d\'erivable sur $\rpe$ et continue en z\'ero. Par l'in\'egalit\'e arithm\'etico-g\'eom\'etrique, on a, pour tout entier naturel $n$ non nul,\vv
$${1\s g(n)}={F(n)\s n}={1\s n}\sum_{k=0}^{n-1}a_k\se\Big(\prod_{k=0}^{n-1}a_k\Big)^{{}^{\sst 1\sur\sst n}}\;,$$
donc $\;g(n)\ie\lp\prod_{k=0}^{n-1}{1\s a_k}\rp^{\!\!{\sst 1\sur\sst n}}\;$ et, d'apr\`es le th\'eor\`eme de Hardy, la s\'erie de terme g\'en\'eral $g(n)$ est convergente.

\msk
{\bf 3.} La fonction $g$ \'etant \`a valeurs positives, l'int\'egrabilit\'e de $g$ sur $\rplus$ \'equivaut \`a la convergence de la s\'erie de terme g\'en\'eral $\;c_n=\int_n^{n+1}g(t)\>dt$.
Or, la fonction $g$ est de classe ${\cal C}^1$ sur $[1,\ii[$ et, si on prouve que sa d\'eriv\'ee $g'$ est int\'egrable sur $[1,\ii[$, la s\'erie $\sum_nc_n$ sera de m\^eme nature que la s\'erie $\sum_ng(n)$, donc convergente ({\it eh oui, c'est un th\'eor\`eme au programme!}). On a $\;g'(x)={1\s F(x)}-{x\>f(x)\s F(x)^2}\;$ sur $\rpe$.\msk\sect
Posons $I=\int_0^{\ii}{dx\s f(x)}>0$. Pour tout r\'eel $x$ strictement positif, l'in\'egalit\'e de Cauchy-Schwarz donne\vv
$$\lp\int_0^x{dt\s f(t)}\rp\lp\int_0^x f(t)\>dt\rp\se\lp\int_0^x dt\rp^2=x^2\;,$$
d'o\`u $F(x)\se{x^2\s I}$. On a donc ${1\s F(x)}\ie{I\s x^2}$ et la fonction ${1\s F}$ est int\'egrable sur $[1,\ii[$. Par ailleurs, une int\'egration par parties (\'ecrite ici sur des int\'egrales ind\'efinies) donne\break $\int{x\>f(x)\s F(x)^2}\>dx=-{x\s F(x)}+\int{dx\s F(x)}$~; on a $\Lim{x}{\ii}{x\s F(x)}=0$ et la fonction ${1\s F}$ est int\'egrable sur $[1,\ii[$, il en r\'esulte que la fonction $x\mapsto{x\>f(x)\s F(x)^2}\;$ (\`a valeurs positives) est int\'egrable sur $[1,\ii[$. On a ainsi prouv\'e l'int\'egrabilit\'e sur $[1,\ii[$ de la fonction $g'$, c.q.f.d.
\msk
{\bf 4.} On v\'erifie l'\'egalit\'e propos\'ee par l'\'enonc\'e. On a donc\vv
$$\Big(\prod_{k=0}^{n-1}u_k\Big)^{\sst1\sur\sst n}={1\s n+1}\cdot\prod_{k=0}^{n-1}\lp{(k+2)^{k+1}\s (k+1)^k}\;u_k\rp^{\sst1\sur\sst n}\ie{1\s n(n+1)}\cdot\sum_{k=0}^{n-1}{(k+2)^{k+1}\s (k+1)^k}\;u_k$$
par l'in\'egalit\'e arithm\'etico-g\'eom\'etrique. Posons alors $\;w_n=\sum_{k=0}^{n-1}{(k+2)^{k+1}\s(k+1)^k}\;u_k\;$ et\break $v_n=\Big(\prod_{k=0}^{n-1}u_k\Big)^{1\s n}$. Alors, pour tout $N\in\net$,\vv
\begin{eqnarray*}
\sum_{n=1}^Nv_n & \ie & \sum_{n=1}^N{1\s n(n+1)}\;w_n=\sum_{n=1}^N\lp{1\s n}-{1\s n+1}\rp w_n=\sum_{n=1}^N{w_n\s n}-\sum_{n=2}^{N+1}{w_{n-1}\s n}\\
 & = & \sum_{n=2}^N{w_n-w_{n-1}\s n}+w_1-{w_N\s N+1}\\
 & \ie & \sum_{n=2}^N\Big(1+{1\s n}\Big)^n u_{n-1}+2u_0\ie e\>\sum_{n=0}^{N-1}u_n\;.
\end{eqnarray*}
car $\;\Big(1+{1\s n}\Big)^n\ie e\;$ pour tout $n\in\net$.\msk\sect
Ainsi, la convergence de $\sum_nu_n$ entra\^\i ne la convergence de $\sum_n v_n$, et on a, plus pr\'ecis\'ement, la majoration $\;\sum_{n=1}^{\infty}v_n\ie e\>\sum_{n=0}^{\infty}u_n\;$
({\bf in\'egalit\'e de Carleman}).






















\end{document}