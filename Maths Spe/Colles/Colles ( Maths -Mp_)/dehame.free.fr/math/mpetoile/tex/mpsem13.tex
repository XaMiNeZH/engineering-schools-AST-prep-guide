\documentclass{article}
\begin{document}

\parindent=-8mm\leftskip=8mm
\def\new{\par\hskip 8.3mm}
\def\sect{\par\quad}
\hsize=147mm  \vsize=230mm
\hoffset=-10mm\voffset=0mm

\everymath{\displaystyle}       % \'evite le textstyle en mode
                                % math\'ematique

\font\itbf=cmbxti10

\let\dis=\displaystyle          %raccourci
\let\eps=\varepsilon            %raccourci
\let\vs=\vskip                  %raccourci


\frenchspacing

\let\ie=\leq
\let\se=\geq



\font\pc=cmcsc10 % petites capitales (aussi cmtcsc10)

\def\tp{\raise .2em\hbox{${}^{\hbox{\seveni t}}\!$}}%



\font\info=cmtt10




%%%%%%%%%%%%%%%%% polices grasses math\'ematiques %%%%%%%%%%%%
\font\tenbi=cmmib10 % bold math italic
\font\sevenbi=cmmi7% scaled 700
\font\fivebi=cmmi5 %scaled 500
\font\tenbsy=cmbsy10 % bold math symbols
\font\sevenbsy=cmsy7% scaled 700
\font\fivebsy=cmsy5% scaled 500
%%%%%%%%%%%%%%% polices de pr\'esentation %%%%%%%%%%%%%%%%%
\font\titlefont=cmbx10 at 20.73pt
\font\chapfont=cmbx12
\font\secfont=cmbx12
\font\headfont=cmr7
\font\itheadfont=cmti7% at 6.66pt



% divers
\def\euler{\cal}
\def\goth{\cal}
\def\phi{\varphi}
\def\epsilon{\varepsilon}

%%%%%%%%%%%%%%%%%%%%  tableaux de variations %%%%%%%%%%%%%%%%%%%%%%%
% petite macro d'\'ecriture de tableaux de variations
% syntaxe:
%         \variations{t    && ... & ... & .......\cr
%                     f(t) && ... & ... & ...... \cr
%
%etc...........}
% \`a l'int\'erieur de cette macro on peut utiliser les macros
% \croit (la fonction est croissante),
% \decroit (la fonction est d\'ecroissante),
% \nondef (la fonction est non d\'efinie)
% si l'on termine la derni\`ere ligne par \cr, un trait est tir\'e en dessous
% sinon elle est laiss\'ee sans trait
%%%%%%%%%%%%%%%%%%%%%%%%%%%%%%%%%%%%%%%%%%%%%%%%%%%%%%%%%%%%%%%%%%%

\def\variations#1{\par\medskip\centerline{\vbox{{\offinterlineskip
            \def\decroit{\searrow}
    \def\croit{\nearrow}
    \def\nondef{\parallel}
    \def\tableskip{\omit& height 4pt & \omit \endline}
    % \everycr={\noalign{\hrule}}
            \def\cr{\endline\tableskip\noalign{\hrule}\tableskip}
    \halign{
             \tabskip=.7em plus 1em
             \hfil\strut $##$\hfil &\vrule ##
              && \hfil $##$ \hfil \endline
              #1\crcr
           }
 }}}\medskip}   

%%%%%%%%%%%%%%%%%%%%%%%% NRZCQ %%%%%%%%%%%%%%%%%%%%%%%%%%%%
\def\nmat{{\rm I\kern-0.5mm N}}  
\def\rmat{{\rm I\kern-0.6mm R}}  
\def\cmat{{\rm C\kern-1.7mm\vrule height 6.2pt depth 0pt\enskip}}  
\def\zmat{\mathop{\raise 0.1mm\hbox{\bf Z}}\nolimits}
\def\qmat{{\rm Q\kern-1.8mm\vrule height 6.5pt depth 0pt\enskip}}  
\def\dmat{{\rm I\kern-0.6mm D}}
\def\lmat{{\rm I\kern-0.6mm L}}
\def\kmat{{\rm I\kern-0.7mm K}}

%___________intervalles d'entiers______________
\def\[ent{[\hskip -1.5pt [}
\def\]ent{]\hskip -1.5pt ]}
\def\rent{{\bf ]}\hskip -2pt {\bf ]}}
\def\lent{{\bf [}\hskip -2pt {\bf [}}

%_____d\'ef de combinaison
\def\comb{\mathop{\hbox{\large C}}\nolimits}

%%%%%%%%%%%%%%%%%%%%%%% Alg\`ebre lin\'eaire %%%%%%%%%%%%%%%%%%%%%
%________image_______
\def\im{\mathop{\rm Im}\nolimits}
%________d\'eterminant_______
\def\det{\mathop{\rm det}\nolimits} 
\def\Det{\mathop{\rm Det}\nolimits}
\def\diag{\mathop{\rm diag}\nolimits}
%________rang_______
\def\rg{\mathop{\rm rg}\nolimits}
%________id_______
\def\id{\mathop{\rm id}\nolimits}
\def\tr{\mathop{\rm tr}\nolimits}
\def\Id{\mathop{\rm Id}\nolimits}
\def\Ker{\mathop{\rm Ker}\nolimits}
\def\bary{\mathop{\rm bar}\nolimits}
\def\card{\mathop{\rm card}\nolimits}
\def\Card{\mathop{\rm Card}\nolimits}
\def\grad{\mathop{\rm grad}\nolimits}
\def\Vect{\mathop{\rm Vect}\nolimits}
\def\Log{\mathop{\rm Log}\nolimits}

%________GL_______
\def\GLR#1{{\rm GL}_{#1}(\rmat)}  
\def\GLC#1{{\rm GL}_{#1}(\cmat)}  
\def\GLK#1#2{{\rm GL}_{#1}(#2)}
\def\SO{\mathop{\rm SO}\nolimits}
\def\SDP#1{{\cal S}_{#1}^{++}}
%________spectre_______
\def\Sp{\mathop{\rm Sp}\nolimits}
%_________ transpos\'ee ________
%\def\t{\raise .2em\hbox{${}^{\hbox{\seveni t}}\!$}}
\def\t{\,{}^t\!\!}

%_______M gothL_______
\def\MR#1{{\cal M}_{#1}(\rmat)}  
\def\MC#1{{\cal M}_{#1}(\cmat)}  
\def\MK#1{{\cal M}_{#1}(\kmat)}  

%________Complexes_________ 
\def\Re{\mathop{\rm Re}\nolimits}
\def\Im{\mathop{\rm Im}\nolimits}

%_______cal L_______
\def\L{{\euler L}}

%%%%%%%%%%%%%%%%%%%%%%%%% fonctions classiques %%%%%%%%%%%%%%%%%%%%%%
%________cotg_______
\def\cotan{\mathop{\rm cotan}\nolimits}
\def\cotg{\mathop{\rm cotg}\nolimits}
\def\tg{\mathop{\rm tg}\nolimits}
%________th_______
\def\tanh{\mathop{\rm th}\nolimits}
\def\th{\mathop{\rm th}\nolimits}
%________sh_______
\def\sinh{\mathop{\rm sh}\nolimits}
\def\sh{\mathop{\rm sh}\nolimits}
%________ch_______
\def\cosh{\mathop{\rm ch}\nolimits}
\def\ch{\mathop{\rm ch}\nolimits}
%________log_______
\def\log{\mathop{\rm log}\nolimits}
\def\sgn{\mathop{\rm sgn}\nolimits}

\def\Arcsin{\mathop{\rm Arcsin}\nolimits}   
\def\Arccos{\mathop{\rm Arccos}\nolimits}  
\def\Arctan{\mathop{\rm Arctan}\nolimits}   
\def\Argsh{\mathop{\rm Argsh}\nolimits}     
\def\Argch{\mathop{\rm Argch}\nolimits}     
\def\Argth{\mathop{\rm Argth}\nolimits}     
\def\Arccotan{\mathop{\rm Arccotan}\nolimits}
\def\coth{\mathop{\rm coth}\nolimits}
\def\Argcoth{\mathop{\rm Argcoth}\nolimits}
\def\E{\mathop{\rm E}\nolimits}
\def\C{\mathop{\rm C}\nolimits}

\def\build#1_#2^#3{\mathrel{\mathop{\kern 0pt#1}\limits_{#2}^{#3}}} 

%________classe C_________
\def\C{{\cal C}}
%____________suites et s\'eries_____________________
\def\suiteN #1#2{(#1 _#2)_{#2\in \nmat }}  
\def\suite #1#2#3{(#1 _#2)_{#2\ge#3 }}  
\def\serieN #1#2{\sum_{#2\in \nmat } #1_#2}  
\def\serie #1#2#3{\sum_{#2\ge #3} #1_#2}  

%___________norme_________________________
\def\norme#1{\|{#1}\|}  
\def\bignorme#1{\left|\hskip-0.9pt\left|{#1}\right|\hskip-0.9pt\right|}

%____________vide (perso)_________________
\def\vide{\hbox{\O }}
%____________partie
\def\P{{\cal P}}

%%%%%%%%%%%%commandes abr\'eg\'ees%%%%%%%%%%%%%%%%%%%%%%%
\let\lam=\lambda
\let\ddd=\partial
\def\bsk{\vspace{12pt}\par}
\def\msk{\vspace{6pt}\par}
\def\ssk{\vspace{3pt}\par}
\let\noi=\noindent
\let\eps=\varepsilon
\let\ffi=\varphi
\let\vers=\rightarrow
\let\srev=\leftarrow
\let\impl=\Longrightarrow
\let\tst=\textstyle
\let\dst=\displaystyle
\let\sst=\scriptstyle
\let\ssst=\scriptscriptstyle
\let\divise=\mid
\let\a=\forall
\let\e=\exists
\let\s=\over
\def\vect#1{\overrightarrow{\vphantom{b}#1}}
\let\ov=\overline
\def\eu{\e !}
\def\pn{\par\noi}
\def\pss{\par\ssk}
\def\pms{\par\msk}
\def\pbs{\par\bsk}
\def\pbn{\bsk\noi}
\def\pmn{\msk\noi}
\def\psn{\ssk\noi}
\def\nmsk{\noalign{\msk}}
\def\nssk{\noalign{\ssk}}
\def\equi_#1{\build\sim_#1^{}}
\def\lp{\left(}
\def\rp{\right)}
\def\lc{\left[}
\def\rc{\right]}
\def\lci{\left]}
\def\rci{\right[}
\def\Lim#1#2{\lim_{#1\vers#2}}
\def\Equi#1#2{\equi_{#1\vers#2}}
\def\Vers#1#2{\quad\build\longrightarrow_{#1\vers#2}^{}\quad}
\def\Limg#1#2{\lim_{#1\vers#2\atop#1<#2}}
\def\Limd#1#2{\lim_{#1\vers#2\atop#1>#2}}
\def\lims#1{\Lim{n}{+\infty}#1_n}
\def\cl#1{\par\centerline{#1}}
\def\cls#1{\pss\centerline{#1}}
\def\clm#1{\pms\centerline{#1}}
\def\clb#1{\pbs\centerline{#1}}
\def\cad{\rm c'est-\`a-dire}
\def\ssi{\it si et seulement si}
\def\lac{\left\{}
\def\rac{\right\}}
\def\ii{+\infty}
\def\eg{\rm par exemple}
\def\vv{\vskip -2mm}
\def\vvv{\vskip -3mm}
\def\vvvv{\vskip -4mm}
\def\union{\;\cup\;}
\def\inter{\;\cap\;}
\def\sur{\above .2pt}
\def\tvi{\vrule height 12pt depth 5pt width 0pt}
\def\tv{\vrule height 8pt depth 5pt width 1pt}
\def\rplus{\rmat_+}
\def\rpe{\rmat_+^*}
\def\rdeux{\rmat^2}
\def\rtrois{\rmat^3}
\def\net{\nmat^*}
\def\ret{\rmat^*}
\def\cet{\cmat^*}
\def\rbar{\ov{\rmat}}
\def\deter#1{\left|\matrix{#1}\right|}
\def\intd{\int\!\!\!\int}
\def\intt{\int\!\!\!\int\!\!\!\int}
\def\ce{{\cal C}}
\def\ceun{{\cal C}^1}
\def\cedeux{{\cal C}^2}
\def\ceinf{{\cal C}^{\infty}}
\def\zz#1{\;{\raise 1mm\hbox{$\zmat$}}\!\!\Bigm/{\raise -2mm\hbox{$\!\!\!\!#1\zmat$}}}
\def\interieur#1{{\buildrel\circ\over #1}}
%%%%%%%%%%%% c'est la fin %%%%%%%%%%%%%%%%%%%%%%%%%%%

\def\boxit#1#2{\setbox1=\hbox{\kern#1{#2}\kern#1}%
\dimen1=\ht1 \advance\dimen1 by #1 \dimen2=\dp1 \advance\dimen2 by #1
\setbox1=\hbox{\vrule height\dimen1 depth\dimen2\box1\vrule}%
\setbox1=\vbox{\hrule\box1\hrule}%
\advance\dimen1 by .4pt \ht1=\dimen1
\advance\dimen2 by .4pt \dp1=\dimen2 \box1\relax}


\catcode`\@=11
\def\system#1{\left\{\null\,\vcenter{\openup1\jot\m@th
\ialign{\strut\hfil$##$&$##$\hfil&&\enspace$##$\enspace&
\hfil$##$&$##$\hfil\crcr#1\crcr}}\right.}
\catcode`\@=12
\pagestyle{empty}
\def\lap#1{{\cal L}[#1]}
\def\DP#1#2{{\partial#1\s\partial#2}}
\def\cala{{\cal A}}
\def\fhat{\widehat{f}}
\let\wh=\widehat
\def\ftilde{\tilde{f}}

% ********************************************************************************************************************** %
%                                                                                                                                                                                   %
%                                                                    FIN   DES   MACROS                                                                              %
%                                                                                                                                                                                   %
% ********************************************************************************************************************** %










\def\lap#1{{\cal L}[#1]}
\def\DP#1#2{{\partial#1\s\partial#2}}



\overfullrule=0mm


\cl{{\bf SEMAINE 13}}\msk
\cl{{\bf INT\'EGRALES D\'EPENDANT d'UN PARAM\`ETRE}}
\bsk

{\bf EXERCICE 1 :}\msk
Si $f:\rplus\vers\cmat$ est une fonction continue par morceaux, la {\bf transform\'ee de Laplace} de $f$ est la fonction ${\cal L}[f]$ d\'efinie par\vv
$$\lap{f}(p)=\int_0^{\ii}e^{-pt}\>f(t)\>dt$$
pour tout r\'eel $p$ tel que cette int\'egrale est convergente.
\msk
{\bf 1.} {\bf Th\'eor\`eme de la valeur finale}\pn
Soit $f:\rplus\vers\cmat$, continue par morceaux, admettant une limite finie
en $\ii$~: $\Lim{t}{\ii}f(t)=l$.\pn
Montrer que la transform\'ee $\lap{f}$ est d\'efinie (au moins) sur $\rpe$ et
que\vv
$$\Lim{p}{0^+}p\cdot\lap{f}(p)=l=\Lim{t}{\ii}f(t)\;.$$
\par
{\bf 2.} Soit $f:\rplus\vers\cmat$, continue, telle que l'int\'egrale $\int_0^{\ii}f(t)
\;dt$ soit convergente (\'eventuellement ``semi-convergente''). Montrer alors
que, pour tout $p\ge0$, l'int\'egrale $\int_0^{\ii}f(t)\>e^{-pt}\;dt\;$ converge
et que la fonction $\;p\mapsto\int_0^{\ii}f(t)\>e^{-pt}\;dt\;$ est continue
sur $\rplus$.
\msk
{\bf 3.} Utiliser la question pr\'ec\'edente pour calculer l'int\'egrale $\;I=\int_0^{\ii}{\sin t\s t}\>dt$.

\msk
\cl{- - - - - - - - - - - - - - - - - - - - - - - - - - - - - -}
\msk

{\bf 1.} La fonction $f$ est born\'ee sur $\rplus$ donc, pour tout $p>0$, la fonction $\>t\mapsto e^{-pt}\>f(t)\>$ est int\'egrable sur $\rplus$. \'Ecrivons\vv
$$p\cdot\lap{f}(p)-l=\int_0^{\ii}p\>e^{-pt}\>\big(f(t)-l\big)\;dt\;.$$
Soit $M$ un majorant de $|f(t)-l|$ sur $\rplus$. Pour tout $A>0$, on peut alors \'ecrire\vv
\begin{eqnarray*}
|p\cdot\lap{f}(p)-l| & \ie & \left|\int_0^Ap\>e^{-pt}\>\big(f(t)-l\big)\;dt\right|+\left|\int_A^{\ii}p\>e^{-pt}\>\big(f(t)-l\big)\;dt\right|\\
& \ie & M\>\int_0^A p\>e^{-pt}\>dt+\int_A^{\ii}p\>e^{-pt}\>|f(t)-l|\>dt\;.\qquad\hbox{\bf (*)}
\end{eqnarray*}
\sect
Donnons-nous alors $\eps>0$. Fixons $A$ tel que $|f(t)-l|<{\eps\s2}$ pour $t\se A$, ce qui rend la deuxi\`eme int\'egrale de {\bf (*)} inf\'erieure \`a ${\eps\s2}$. Comme $\int_0^A p\>e^{-pt}\>dt=1-e^{-pA}\Vers{p}{0}0$, on peut rendre la premi\`ere int\'egrale inf\'erieure \`a ${\eps\s2}$ en prenant $p$ suffisamment proche de $0$. On a ainsi prouv\'e que $\Lim{p}{0^+}\big(p\cdot\lap{f}(p)-l\big)=0$.\msk
{\it On en d\'eduit que, si $l\not=0$, alors l'ensemble de d\'efinition de $\lap{f}$ est exactement $\rpe$ et que\break$\lap{f}(p)
\equi_{p\vers0}{l\s p}$}.

\msk
{\bf 2.} Soit $F$ la primitive de $f$ qui s'annule en z\'ero. La fonction
$F$ est de classe ${\cal C}^1$ sur $\rplus$ et admet une limite finie en $\ii$, donc est born\'ee sur $\rplus$.
Pour tout $p>0$, la fonction $t\mapsto F(t)\>e^{-pt}$ est int\'egrable sur $\rplus$ et $\Lim{t}{\ii}F(t)\>e^{-pt}=0$, ce qui permet une int\'egration par parties~:
$$\a p\in\rpe\qquad\int_0^{\ii}f(t)\>e^{-pt}\;dt=p\cdot\int_0^{\ii}F(t)\>e^{-pt}\;dt\;.$$
La transform\'ee de Laplace $\lap{f}$ est donc d\'efinie (au moins) sur $\rplus$ et on a\vv
$$\a p\in\rpe\qquad \lap{f}(p)=p\cdot\lap{F}(p)\;.\qquad\hbox{\bf (*)}$$
La transform\'ee $\lap{F}$, d\'efinie au moins sur $\rpe$, est continue sur cet intervalle~: en effet, si on fixe $p_0>0$, la fonction $t\mapsto F(t)\>e^{-p_0t}$ est int\'egrable sur $\rplus$ et une domination \'evidente montre la continuit\'e de $\lap{F}$ sur l'intervalle $[p_0,\ii[\>$. Gr\^ace \`a {\bf (*)}, on d\'eduit la continuit\'e de $\lap{f}$ sur $\rpe$. Enfin,\vv
$$\lap{f}(0)=\int_0^{\ii}f(t)\>dt=\lim_{\ii}F=\Lim{p}{0^+}p\cdot\lap{F}(p)=\Lim{p}{0^+}\lap{f}(p)$$
d'apr\`es le th\'eor\`eme de la valeur finale, d'o\`u la continuit\'e de la fonction $\lap{f}$ en 0.

\msk
{\bf 3.} Il est bien connu que cette int\'egrale $I$ est ``semi-convergente''.
Appliquons alors la question {\bf 2.} \`a la fonction ``sinus cardinal'', \`a savoir $f:t\mapsto{\sin t\s t}$, prolong\'ee par continuit\'e en z\'ero~: sa transform\'ee de Laplace est donc d\'efinie et continue sur $\rplus$. Or, il est assez ais\'e de calculer l'expression de $\;\lap{f}(p)=\int_0^{\ii}e^{-pt}\>{\sin t\s t}\>dt\;$ pour $p>0$.\msk\sect
Pour cela, consid\'erons $g:(p,t)\mapsto e^{-pt}\;{\sin t\s t}$. La fonction $g$ est
continue sur
$(\rpe)^2$ et, si $a>0$, on~a $|g(p,t)|\le{e^{-at}\>|\sin t|\s t}$ pour
$(p,t)\in[a,\ii[\times\rpe$. La fonction $\;t\mapsto{e^{-at}\>|\sin t|\s t}$
\'etant int\'egrable sur $\rpe$, cela prouve la continuit\'e de la fonction $\lap{f}$ sur $[a,\ii[$
pour tout $a>0$, donc sur $\rpe$.\ssk\sect
De plus, $\DP{g}{p}(p,t)=-e^{-pt}\;\sin t\;$ et, si $a>0$, la majoration\vv
$$\left|\DP{g}{p}(p,t)\right|\le e^{-at}\;,\;\hbox{valable pour}\quad
  (p,t)\in[a,\ii[\times\rpe\;,$$
prouve que la fonction $\Phi=\lap{f}$ est de classe ${\cal C}^1$ sur $[a,\ii[$ pour tout $a>0$,
donc sur $\rpe$, avec\vv
$$\Phi'(p)=-\int_0^{\ii}e^{-pt}\;\sin t\;dt=-{1\s1+p^2}\;.$$
Donc $\;\Phi(p)=C-\arctan p\;$ sur $\rpe$ et le th\'eor\`eme de convergence domin\'ee
(``version familiale'', c'est-\`a-dire appliqu\'e \`a une famille de fonctions) permet de montrer que $\lim_{\ii}\Phi=0$, donc
$C={\pi\s2}$ et $$\a p\in\rpe\qquad \Phi(p)=\lap{f}(p)={\pi\s2}-\Arctan p\;.$$
La question {\bf 2.} permet d'affirmer que la fonction $\lap{f}$ est continue en z\'ero ({\it ce que les th\'eor\`emes du cours ne suffisent pas \`a garantir puisque la fonction sinus cardinal n'est pas int\'egrable sur $\rplus$}), d'o\`u\vv
$$I=\int_0^{\ii}{\sin t\s t}\>dt=\lap{f}(0)=\Lim{p}{0^+}\lap{f}(p)={\pi\s2}\;.$$


\bsk
\hrule
\bsk

{\bf EXERCICE 2 :}\msk
Pour tout $x>0$, on pose $\;\Gamma(x)=\int_0^{\ii}e^{-t}\>t^{x-1}\>dt$.\msk
{\bf 1.} D\'emontrer la relation~: \qquad $\a x\in\rpe\qquad\Gamma(x)=\Lim{n}{\infty}\int_0^n\lp1-{t\s n}\rp^n\>t^{x-1}\>dt$.\msk
{\bf 2.} En d\'eduire~:\qquad$\a x\in\rpe\qquad \Gamma(x)=\Lim{n}{\ii}{n^x\;n!\s x(x+1)\cdots(x+n)}$.\msk\sect
En d\'eduire, pour tout $x>0$ fix\'e, l'\'equivalence\vv
$$x(x+1)\cdots(x+n)\sim{n^x\>n!\s\Gamma(x)}$$
lorsque $n$ tend vers $\ii$.\msk
Dans la suite de l'exercice, on note  $f$ une fonction logarithmiquement convexe (c'est-\`a-dire
la fonction $x\mapsto\ln\big(f(x)\big)$ est convexe) de $\rpe$ vers $\rpe$, v\'erifiant $f(1)=1$ et
la relation fonctionnelle $\;\a x\in\rpe\quad f(x+1)=x\>f(x)$.\msk
{\bf 3.} Soient $x>0$, $y>0$, $\lam\in[0,1]$. Posons $t=\lam x+(1-\lam)y$.
Montrer, pour tout $n\in\nmat$, l'in\'egalit\'e\vv
$$t(t+1)\cdots(t+n)\>f(t)\le\big(x(x+1)\cdots(x+n)\;f(x)\big)
  ^{\lam}\cdot\big(y(y+1)\cdots(y+n)\;f(y)\big)^{1-\lam}\;.$$\ssk\sect
En d\'eduire que\vvvv
$${f(t)\s\Gamma(t)}\le\lp{f(x)\s\Gamma(x)}\rp^{\lam}\>
  \lp{f(y)\s\Gamma(y)}\rp^{1-\lam}\;.$$\ssk
{\bf 4.} Montrer que $f=\Gamma$.



\msk
\cl{- - - - - - - - - - - - - - - - - - - - - - - - - - - - - -}
\msk

{\bf 1.} Plus g\'en\'eralement, soit $f:\;]0,\ii[\vers\cmat$ une fonction continue telle que la
fonction\break $g:t\mapsto e^{-t}\>f(t)$ soit int\'egrable sur $\rmat_+^*$. Alors\vv
$$\int_0^{\ii}e^{-t}\;f(t)\;dt=\Lim{n}{\ii}\int_0^n\lp1-{t\s n}\rp
^n\;f(t)\;dt\;.$$
En effet, pour tout r\'eel $t$, on a $e^{-t}=\Lim{n}{\ii}
\lp1-{t\s n}\rp^n$. D\'efinissons, pour tout $n\in\nmat^*$, une
fonction $u_n:\;]0,\ii[\vers\rmat$ par\vv
$$u_n(t)\;=\system{&\lp1-{t\s n}\rp^n\quad&\hbox{si}\quad&0<t\le n\cr
                   &\hfill0\hfill         &\hbox{si}\quad&t>n\;.\cr}$$
Alors $u_n$ est continue sur $\rmat_+^*$ et la suite $(u_n)$ converge
simplement, sur $\rmat_+^*$, vers la fonction $t\mapsto e^{-t}$.\pn
En posant $g_n=u_n\cdot f$, on a une suite $(g_n)$ de fonctions continues
sur $\rmat_+^*$, convergeant simplement vers $g$ sur $\rmat_+^*$.
L'in\'egalit\'e classique $\;\ln\lp1-{t\s n}\rp\le-{t\s n}$, va\-lable pour
$t\in[0,n[\;$, montre que\vv
$$\a n\in\nmat^*\quad\a t\in\rmat_+^*\qquad 0\le u_n(t)\le e^{-t}\quad
\hbox{donc}\quad|g_n(t)|\le|g(t)|\;.$$
L'hypoth\`ese de domination est alors v\'erifi\'ee et le th\'eor\`eme de convergence
domin\'ee s'applique. Il suffit donc d'appliquer ce r\'esultat avec $\;f(t)=t^{x-1}$.

\msk
{\bf 2.} Le changement de variable $t=nu$ donne\vv
$$\int_0^n\lp1-{t\s n}\rp^n\>t^{x-1}\;dt =n^x\;\int_0^1(1-u)^n
                                                         \>u^{x-1}\;du
                                                    =n^x\;B(x,n+1)\;,$$
en notant $\;B(p,q)=\int_0^1u^{p-1}\>(1-u)^{q-1}\>du\;$ pour $p$ et $q$ r\'eels strictement positifs ({\bf int\'egrale eul\'erienne de premi\`ere esp\`ece}). La fonction $u\mapsto u^{p-1}(1-u)^{q-1}$
est bien int\'egrable sur $]0,1[$ et, pour tout $n\in\net$ et $x>0$, une int\'egration
par parties donne\vv
\begin{eqnarray*}
B(x,n+1) & = & \int_0^1u^{x-1}(1-u)^n\;du
                      =\lc(1-u)^n{u^x\s x}\rc_0^1+{n\s x}\;
                                 \int_0^1u^x(1-u)^{n-1}\;du\\
                    & = & {n\s x}\;B(x+1,n)\;.
\end{eqnarray*}
\`A partir de $B(x,1)=\int_0^1u^{x-1}\;du={1\s x}$ pour tout $x>0$, une
r\'ecurrence imm\'ediate donne
$$B(x,n)={(n-1)!\s x(x+1)(x+2)\cdots(x+n-1)}\;.$$
Finalement,\vv
$$\a x\in\rpe\quad\a n\in\net\qquad\int_0^n\lp1-{t\s n}\rp^n\>t^{x-1}\;dt={n^x\;n!\s x(x+1)\cdots(x+n)}\;,$$
d'o\`u le r\'esultat. L'\'equivalence demand\'ee est alors une cons\'equence imm\'ediate.

\msk
{\bf 3.} En vertu de la relation fonctionnelle satisfaite par $f$, l'in\'egalit\'e \`a prouver \'equivaut \`a\vv
$$f\big(\lam(x+n+1)+(1-\lam)(y+n+1)\big)\le\big(f(x+n+1)\big)^{\lam}
  \>\big(f(y+n+1)\big)^{1-\lam}\;,$$
ou encore \`a\vv
$$\ln\Big[f\big(\lam(x+n+1)+(1-\lam)(y+n+1)\big)\Big]\le\lam\>\ln\big(f(x+n+1)\big)+
  (1-\lam)\>\ln\big(f(y+n+1)\big)\;,$$
ce qui r\'esulte de la convexit\'e de la fonction $\ln\circ f$.\msk\sect
L'in\'egalit\'e obtenue peut aussi s'\'ecrire\vv
$${f(t)\s\big(f(x)\big)^{\lam}\>\big(f(y)\big)^{1-\lam}}\le
  {\big(x(x+1)\cdots(x+n)\big)^{\lam}\;\big(y(y+1)\cdots(y+n)\big)^{1-\lam}
  \s t(t+1)\cdots(t+n)}\;.\qquad\quad\hbox{\bf (*)}$$
Faisons tendre $n$ vers $\ii$ en utilisant l'\'equivalence d\'emontr\'ee \`a la fin de la question {\bf 2.} 
Le second membre de {\bf (*)} tend vers $\;{\Gamma(t)\s\big(\Gamma(x)\big)^{\lam}\>\big(\Gamma(y)\big)^{1-\lam}}$.
Il vient alors
$${f(t)\s\Gamma(t)}\le\lp{f(x)\s\Gamma(x)}\rp^{\lam}\>
  \lp{f(y)\s\Gamma(y)}\rp^{1-\lam}\;.$$
\ssk
{\bf 4.} L'in\'egalit\'e obtenue ci-dessus signifie que la fonction $\ln\lp
{f\s\Gamma}\rp\;$ est convexe sur $\rpe$. Or, cette fonction est
1-p\'eriodique. Elle est donc constante~: en effet, si une fonction $g$ est convexe et 1-p\'eriodique sur $\rpe$ avec $g(1)=g(2)=C$, on obtient ais\'ement $g\ie C$ sur $[1,2]$ et $g\se C$ sur $[2,3]$ et la p\'eriodicit\'e entra\^\i ne $g=C$ sur $[1,3]$, donc sur $\rpe$.\msk\sect
Comme $f(1)=\Gamma(1)=1$, on a donc $f=\Gamma$.


\bsk
\hrule
\bsk

{\bf EXERCICE 3 :}\msk
{\bf 1.}. Soit $\ffi:[0,1]\vers\rmat$ une application de classe $\ce^2$.
D\'emontrer l'\'egalit\'e\vvv
$$\int_0^1\ffi(t)\;dt={1\s2}\big(\ffi(0)+\ffi(1)\big)-{1\s2}\>\int_0^1
  t(1-t)\>\ffi''(t)\;dt\;.\eqno\hbox{\bf (*)}$$
On suppose maintenant que $\ffi(0)=\ffi(1)=0$. Montrer l'existence d'une
constante $C$ telle
que $\;\left|\int_0^1\ffi\right|\le C\cdot M$, o\`u $M=\max_{[0,1]}|\ffi''|$.
\msk
{\bf 2.} On note $K$ le pav\'e $[0,1]^2$. Soit $f:K\vers\rmat$, de classe
$\ce^4$. On suppose que $f$ est nulle sur le bord $\ddd K$ du pav\'e $K$ et
que $\;\left|{\ddd^4f\s\ddd x^2\>\ddd y^2}\right|\le M'\;$
sur $K$. Trouver une constante $C'$ telle que
$$\left|\intd_Kf\right|\le C'\cdot M'\;.$$

\msk
\cl{- - - - - - - - - - - - - - - - - - - - - - - - - - - - - - -}
\msk

{\bf 1.} Par deux int\'egrations par parties successives, on obtient\vv
\begin{eqnarray*}
\int_0^1t(1-t)\>\ffi''(t)\;dt & = & \big[t(1-t)\>\ffi'(t)\big]_0^1
                                             +\int_0^1(2t-1)\>\ffi'(t)\;dt\\
                                         & = & \big[(2t-1)\>\ffi(t)\big]_0^1-
                                             2\>\int_0^1\ffi(t)\;dt\\
                                         & = & \ffi(1)+\ffi(0)-2\int_0^1\ffi
                                             \;,
\end{eqnarray*}
d'o\`u la relation {\bf (*)}. Si $\ffi(0)=\ffi(1)=0$, il est alors imm\'ediat
que\vv
$$\left|\int_0^1\ffi\right|={1\s2}\left|\int_0^1t(1-t)\>\ffi''(t)\;dt\right|
  \le{M\s2}\>\int_0^1t(1-t)\;dt={M\s12}\;,$$
d'o\`u la possibilit\'e de choisir $C={1\s12}$.\pn
Ce choix est le ``meilleur'' possible, ainsi qu'on le voit en consid\'erant la
fonction\break $\ffi:t\mapsto t(1-t)\;$ (fonction v\'erifiant $\ffi(0)=\ffi(1)=0$ et
$\ffi''$ constante sur $[0,1]$).

\msk
{\bf 2.} La formule de Fubini permet d'\'ecrire\vv
$$\intd_Kf=\int_0^1\lp\int_0^1f(x,y)\;dy\rp\>dx\;.$$
Or, en appliquant {\bf (*)} \`a $\;y\mapsto f(x,y)\;$ pour un $x\in[0,1]$
fix\'e, puisque $f(x,0)=f(x,1)=0$,\vv
$$\int_0^1f(x,y)\;dy=-{1\s2}\>\int_0^1y(1-y)\>{\ddd^2f\s\ddd y^2}(x,y)\;dy
  \;,$$
puis\vv
\begin{eqnarray*}
\intd_Kf & = & -{1\s2}\>\int_0^1\lp\int_0^1y(1-y)\>{\ddd^2f\s
                            \ddd y^2}(x,y)\;dy\rp\>dx\\
                    & = & -{1\s2}\>\int_0^1y(1-y)\>\lp\int_0^1
                            {\ddd^2f\s\ddd y^2}(x,y)\;dx\rp\>dy\qquad\quad
                            \hbox{(Fubini)}
\end{eqnarray*}
et, de nouveau gr\^ace \`a {\bf (*)}, pour tout $y\in[0,1]$ fix\'e, puisque
$\;{\ddd^2f\s\ddd y^2}(0,y)={\ddd^2f\s\ddd y^2}(1,y)=0$,\vv
$$\int_0^1{\ddd^2f\s\ddd y^2}(x,y)\;dx=-{1\s2}\>\int_0^1x(1-x)\>
  {\ddd^4f\s\ddd x^2\>\ddd y^2}(x,y)\;dx$$
et, finalement, en utilisant une derni\`ere fois Fubini,\vv
$$\intd_Kf={1\s4}\intd_Kxy(1-x)(1-y)\>{\ddd^4f\s\ddd x^2\>\ddd y^2}(x,y)
  \>dx\>dy\;,$$
d'o\`u la majoration\vv
$$\left|\intd_Kf\right|\le{M'\s4}\>\intd_Kxy(1-x)(1-y)\;dx\>dy={M'\s4}\>
  \lp\int_0^1x(1-x)\>dx\rp^2={M'\s144}$$
qui permet de choisir $C'={1\s144}$. Ici encore, la fonction $\;f:(x,y)
\mapsto xy(1-x)(1-y)$, nulle sur le bord du pav\'e $K$ et dont la d\'eriv\'ee
partielle ${\ddd^4f\s\ddd x^2\>\ddd y^2}$ garde une valeur constante, montre
que $C'={1\s144}$ est ``la meilleure'' constante possible.


\eject


{\bf EXERCICE 4 :}\msk
{\bf Produit de convolution dans} $\ce_{2\pi}$\msk
Soit ${\cal E}=\ce_{2\pi}$ le $\cmat$-espace vectoriel des fonctions continues
et $2\pi$-p\'eriodiques de
$\rmat$ vers $\cmat$. Pour tous $f$, $g$ de ${\cal E}$, on d\'efinit une fonction
$f*g$ par la relation\vv
$$\a x\in\rmat\qquad (f*g)(x)=\int_0^{2\pi}f(t)\>g(x-t)\;dt\;.$$\par
{\bf 1.} V\'erifier que $*$ est une loi interne commutative dans ${\cal E}$.\pn
Si l'une des fonctions $f$ ou $g$ est suppos\'ee de classe $\ceun$,
que peut-on dire de $f*g$ ?\msk
{\bf 2.} Montrer que ${\cal E}$, muni des lois $+$ (addition usuelle) et $*$,
poss\`ede une structure de pseudo-alg\`ebre sur $\cmat$ (pas d'\'el\'ement unit\'e).\msk
{\bf 3.} On appelle {\itbf approximation de l'unit\'e $2\pi$-p\'eriodique} toute suite
$(e_n)_{n\in\nmat}$ de fonctions de ${\cal E}$ v\'erifiant\ssk\sect
$\bullet$ $\quad \a n\in\nmat\qquad e_n\ge0\;$ sur $\rmat$~;\ssk\sect
$\bullet$ $\quad \a n\in\nmat\qquad \int_{-\pi}^{\pi}e_n=1\;$~; \ssk\sect
$\bullet$ \quad pour tout $\alpha\in\>]0,\pi[\;$, la suite $(e_n)$
               converge uniform\'ement vers la fonction nulle sur
                $\;[-\pi,-\alpha]\;$ et sur $\;[\alpha,\pi]\;$.\msk\sect
Montrer qu'alors, pour tout $f\in {\cal E}$, la suite de fonctions $(e_n*f)$
converge uniform\'ement vers $f$ sur $\rmat$.\msk
{\bf 4.} Montrer que, pour tous $f$, $g\in {\cal E}$, on a\vv
$$\int_0^{2\pi}f*g=\lp\int_0^{2\pi}f\rp\>\lp\int_0^{2\pi}g\rp\;.$$

\msk
\cl{- - - - - - - - - - - - - - - - - - - - - - - - - - - - - -}
\msk


{\bf 1.} La continuit\'e de $\;(x,t)\mapsto f(t)\>g(x-t)\;$ sur $\rmat\times
[0,2\pi]$ garantit la continuit\'e de $f*g$ sur $\rmat$. La p\'eriodicit\'e est
imm\'ediate.\ssk\sect
La commutativit\'e se d\'emontre en faisant le changement de variable $u=x-t$
et en notant que l'int\'egrale d'une fonction $2\pi$-p\'eriodique sur
$[a,a+2\pi]$ ne d\'epend pas du r\'eel $a$.\ssk\sect
Si $g$ est de classe $\ceun$ sur $\rmat$, la formule de Leibniz montre que
$f*g$ est de classe $\ceun$ sur $\rmat$ avec $(f*g)'=f*g'$.
Gr\^ace \`a la commutativit\'e, si $f$ est $\ceun$, alors $f*g$ est $\ceun$ et
$(f*g)'=f'*g$. Notons que, si $f$ et $g$ sont toutes deux $\ceun$,
alors $f*g'=f'*g$, ce que l'on retrouve par une int\'egration par parties.
\msk
{\bf 2.} La distributivit\'e de la convolution par rapport \`a l'addition\vvv
$$f*(g+h)=f*g+f*h$$\sect
est imm\'ediate.\ssk\sect
Prouvons l'associativit\'e de la loi de convolution~:\vv
\begin{eqnarray*}
\big[(f*g)*h\big](x) & = & \int_0^{2\pi}(f*g)(t)\>h(x-t)\;dt\\
                                & = & \int_0^{2\pi}\lp\int_0^{2\pi}f(u)\>
                                       g(t-u)\;du\rp\>h(x-t)\;dt\\
                                & = & \int_0^{2\pi}f(u)\>\lp\int_0^{2\pi}
                                       g(t-u)\>h(x-t)\;dt\rp\;du\;,
\end{eqnarray*}
d'apr\`es la formule de Fubini. Par ailleurs,\vv
\begin{eqnarray*}
\int_0^{2\pi}g(t-u)\>h(x-t)\;dt & = & \int_{-u}^{2\pi-u}g(s)\>
                                                  h(x-u-s)\;ds\\
                                           & = & \int_0^{2\pi}g(s)\>h(x-u-s)\;
                                                  ds=(g*h)(x-u)\;,
\end{eqnarray*}
donc $\;\big[(f*g)*h\big](x)=\int_0^{2\pi}f(u)\>(g*h)(x-u)\;du=\big[f*(g*h)
\big](x)$.\ssk\sect
$({\cal E},+,*)$ est donc muni d'une structure de pseudo-anneau (pas
d'\'el\'ement unit\'e) et il est imm\'ediat que $\lam (f*g)=(\lam f)*g=f*(\lam g)$
pour $\lam\in\cmat$, $f\in {\cal E}$, $g\in {\cal E}$.\msk\sect
V\'erifions qu'il n'y a effectivement pas d'\'el\'ement unit\'e : si une telle fonction $e$
existait, pour tout $n\in\nmat$, notons $c_n$ la fonction de ${\cal E}$ d\'efinie
par $c_n(x)=\cos nx$. Nous aurions alors, pour tout $n\in\nmat$,
$(c_n*e)(0)=c_n(0)$, soit $\;\int_0^{2\pi}e(-t)\>\cos nt\;dt=1$, ce
qui contredit manifestement le th\'eor\`eme de Riemann-Lebesgue.

\msk
{\bf 3.} Soit $f\in {\cal E}$. Notons $\;M=\|f\|_{\infty}=\max_{[0,2\pi]}|f|$.\ssk\sect
Soit $\alpha\in\>]0,\pi[$. Nous avons, pour tout $x\in\rmat$,\vv
$$(e_n*f)(x)-f(x)=\int_{-\pi}^{\pi}e_n(t)\>\big(f(x-t)-f(x)\big)\;dt
  =I_1+I_2+I_3\;,$$
o\`u $I_1$, $I_2$, $I_3$ sont les int\'egrales de cette m\^eme expression sur les
intervalles $[-\pi,-\alpha]$, $[-\alpha,\alpha]$ et $[\alpha,\pi]$
respectivement.\ssk\sect
Donnons-nous alors un $\eps>0$. Comme $f$ est uniform\'ement continue sur
$\rmat$ (car elle est continue et p\'eriodique), nous pouvons trouver un
$\alpha>0$ tel que\vv
$$\a(x,y)\in\rdeux\qquad|x-y|\le\alpha\impl|f(x)-f(y)|\le{\eps\s3}\;.$$
Pour un tel choix de $\alpha$, nous avons\vv
$$|I_2|\le\int_{-\alpha}^{\alpha}e_n(t)\>|f(x-t)-f(x)|\;dt\le{\eps\s3}\>
  \int_{-\alpha}^{\alpha}e_n(t)\;dt\le{\eps\s3}\int_{-\pi}^{\pi}e_n=
  {\eps\s3}\;.$$
Cet $\alpha$ \'etant maintenant fix\'e, nous avons\vv
$$|I_3|=\left|\int_{\alpha}^{\pi}e_n(t)\>\big(f(x-t)-f(x)\big)\;dt\right|\le
  2M\>\int_{\alpha}^{\pi}e_n(t)\;dt\;,$$
et cette derni\`ere expression tend vers 0 lorsque $n$ tend vers $\ii$ en vertu
de la convergence uniforme de la suite $(e_n)$ vers 0 sur $[\alpha,\pi]$~;
il est donc possible de la rendre inf\'erieure \`a ${\eps\s3}$ pour $n$ assez
grand (et ceci ind\'ependamment de $x$). Proc\'edant de m\^eme pour majorer
$|I_1|$, nous d\'eduisons l'existence d'un entier $N$ tel que\vv
$$\a n\in\nmat\qquad n\ge N\impl\|e_n*f-f\|_{\infty}\le \eps\;,$$
donc la convergence uniforme de $(e_n*f)$ vers $f$ sur $\rmat$.\pn
On dit que la suite $(e_n)$ est une {\bf approximation de l'unit\'e}
$2\pi$-{\bf p\'eriodique} car, pour
tout $f$ de ${\cal E}$, les fonctions $e_n*f$ approchent $f$ uniform\'ement. 
\msk
{\bf 4.} C'est une cons\'equence imm\'ediate de la formule de Fubini~:\vv
$$\int_0^{2\pi}f*g = \int_0^{2\pi}\lp\int_0^{2\pi}f(t)\>g(x-t)\;
                                    dt\rp\;dx=\int_0^{2\pi}f(t)\>\lp
                                    \int_0^{2\pi}g(x-t)\;dx\rp\;dt$$
et l'int\'egrale int\'erieure est \'egale \`a $\;\int_0^{2\pi}g(s)\;ds$,
d'o\`u le r\'esultat.

\bsk
\hrule
\bsk

{\bf EXERCICE 5 :}\msk
{\bf Produit de convolution dans} $\ce(\rplus)$\msk
Pour traiter cet exercice, on pourra admettre la ``formule de Fubini dans un triangle''~:\ssk\new
{\it Soit $a\in\rpe$. Soit $f:T_a\vers\cmat$, continue, o\`u $T_a$ est le ``triangle''~:
\vv
$$T_a=\{(x,y)\in\rdeux\;|\;x\ge0\,,\,y\ge0\,,\,x+y\le a\}\;.$$
On a alors l'\'egalit\'e\vv
$$\int_0^a\lp\int_0^{a-x}f(x,y)\;dy\rp\;dx=\int_0^a\lp\int_0^{a-y}f(x,y)\;
  dx\rp\;dy$$
et la valeur commune de ces deux int\'egrales sera not\'ee $\;\intd_{T_a}f(x,y)
\;dx\>dy\;$ ou $\;\intd_{T_a}f\;$}.\msk
Soit ${\cal E}=\ce(\rplus)$ le $\cmat$-espace vectoriel des fonctions
continues de $\rplus$ vers $\cmat$. Pour tous $f$, $g$ de ${\cal E}$, on
d\'efinit une fonction $f*g$ par la relation\vv
$$\a x\in\rplus\qquad (f*g)(x)=\int_0^x f(t)\>g(x-t)\;dt\;.$$\par
{\bf 1.} V\'erifier que $*$ est une loi interne commutative dans ${\cal E}$.\msk
{\bf 2.} Montrer que ${\cal E}$, muni des lois $+$ (addition usuelle) et $*$,
poss\`ede une
structure de pseudo-alg\`ebre sur $\cmat$ (pas d'\'el\'ement unit\'e).\ssk
{\bf 3.} Montrer que, pour tout $a\in\rplus$, l'int\'egrale $\;\int_0^a(f*g)(x)
\;dx\;$ peut s'exprimer comme une int\'egrale double.\ssk
{\bf 4.} Montrer que, si $f$ et $g$ sont int\'egrables sur $\rplus$, alors
$f*g$ est int\'egrable sur $\rplus$ et\vv
$$\int_{\rplus}f*g=\lp\int_{\rplus}f\rp\>\lp\int_{\rplus}g\rp\;.$$

\msk
\cl{- - - - - - - - - - - - - - - - - - - - - - - - - - - - - -}
\msk

{\bf 1.} Le changement de variable lin\'eaire $t=xu$ donne\vv $$(f*g)(x)=x\>\int_0^1f(xu)\>g\big(x(1-u)\big)\>du$$ 
et on en d\'eduit la continuit\'e de $f*g$ sur $\rplus$ ``par application des th\'eor\`emes usuels'' ({\it comme il est d'usage de dire}),
donc $*$ est une loi interne dans ${\cal E}$.
La commutativit\'e r\'esulte imm\'ediatement du changement de variable $u=x-t$.\msk
{\bf 2.} La distributivit\'e de la convolution par rapport \`a l'addition
$\;f*(g+h)=f*g+f*h\;$ est imm\'ediate.\ssk\sect
L'associativit\'e
utilise ``Fubini dans un triangle''~:\vv
\begin{eqnarray*}
\big[(f*g)*h\big](x) & = & \int_0^x(f*g)(x-t)\>h(t)\;dt\\
                                & = & \int_0^x\lp\int_0^{x-t}f(u)\>
                                       g(x-t-u)\;du\rp\>h(t)\;dt\\
                                & = & \int_0^x\lp\int_0^{x-u}
                                       g(x-t-u)\>h(t)\;dt\rp\>f(u)\>du\\
                                & = & \int_0^x(g*h)(x-u)\>f(u)\>du = \big[f*(g*h)\big](x)\;.
\end{eqnarray*}
Pour prouver qu'il n'y a
pas d'\'el\'ement neutre, on montre que la relation $e*1=1$, avec $e\in{\cal E}$, est impossible~: en effet, cela entra\^\i nerait $\int_0^xe(t)\>dt=1$ pour tout $x\in\rplus$, ce qui est manifestement impossible pour $x=0$.
\msk
{\bf 3.} Gr\^ace \`a ``Fubini dans un triangle'', on obtient\vv
\begin{eqnarray*}
\int_0^a (f*g)(x)\>dx & = & \int_0^a(f*g)(a-t)\;dt\\
                                & = & \int_0^a\lp\int_0^{a-t}f(u)\>
                                       g(a-t-u)\;du\rp\;dt\\
                                & = & \int_0^a\lp\int_0^{a-u}
                                       g(a-u-t)\;dt\rp\>f(u)\>du\\
                                & = & \int_0^a\lp\int_0^{a-u}
                                       g(t)\;dt\rp\>f(u)\>du\\
                                & = & \intd_{T_a}f(x)\>g(y)\;dx\>dy\;,
\end{eqnarray*}
avec $\;T_a=\{(x,y)\in\rdeux\;|\;x\ge0\,,\,y\ge0\,,\,x+y\le a\}$.\msk
{\bf 4.} Supposons $f$ et $g$ int\'egrables sur $\rplus$. Notons d'abord que $|f*g|\ie|f|*|g|$. Ensuite, pour tout $a>0$, notons $R_a$ le pav\'e $[0,a]^2$, on a, d'apr\`es la question pr\'ec\'edente,\vv
$$\int_0^a|f*g|\ie\int_0^a|f|*|g|=\intd_{T_a}|f(x)\>g(y)|\;dx\>dy\ie\intd_{R_a}|f(x)\>g(y)|\;dx\>dy=\lp\int_0^a|f|\rp\lp\int_0^a|g|\rp\;,$$
ce qui prouve l'int\'egrabilit\'e de $f*g$ sur $\rplus$.\msk\sect
Pour tout $a>0$, posons\vv
$$\ffi(a)=\lp\int_0^af\rp\>\lp\int_0^ag\rp-\int_0^af*g=\intd_{R_a}f(x)\>g(y)\;dx\>dy-\intd_{T_a}f(x)\>g(y)\;dx\>dy\;.$$
Alors $\;\ffi(a)=\intd_{R_a\setminus T_a}f(x)\>g(y)\;dx\>dy$, donc $$|\ffi(a)|\ie\intd_{R_a\setminus T_a}|f(x)\>g(y)|\;dx\>dy\ie\intd_{R_a\setminus R_{a\sur 2}}|f(x)\>g(y)|\;dx\>dy\;,$$
c'est-\`a-dire\vv
$$|\ffi(a)|\le\lp\int_0^a|f|\rp\>\lp\int_0^a|g|\rp-\lp\int_0^{a\sur2}|f|\rp\>
  \lp\int_0^{a\sur2}|g|\rp\;,$$
d'o\`u $\Lim{a}{\ii}\ffi(a)=0$, ce qu'il fallait d\'emontrer.

\msk
{\it Pour prouver la ``formule de Fubini dans un triangle'', on peut montrer d'abord que, pour toute fonction d'une variable $\ffi:[0,a]\vers \cmat$, continue, on a\vv
$$\int_0^a\lp\int_0^{a-y}\ffi(x)\;dx\rp\>dy=\int_0^a(a-x)\>\ffi(x)\;dx\;,$$
puis appliquer Fubini (celui qui est au programme) \`a la fonction $g:[0,a]^2\vers\cmat$ d\'efinie par $\;g(x,y)=f(x,y)-f(x,a-x)\;$ si $(x,y)\in T_a$ et $g(x,y)=0$ sinon.}

\bsk
\hrule
\bsk

{\bf EXERCICE 6 :}\msk
{\bf 1.} On admet $\;\int_0^{\ii}{\sin x\s x}\;dx={\pi\s2}$.
\ssk\sect
Soit $f:\rplus\vers\cmat$, continue par
morceaux, int\'egrable sur $\rplus$. On suppose que la fonction $g:t\mapsto
{f(t)-f(0^+)\s t}$ est int\'egrable sur $]0,1]$.
Montrer que\vv
$$\Lim{\lam}{\ii}\int_0^{\ii}f(t)\;{\sin\lam t\s t}\;dt={\pi\s2}\;f(0^+)\;.$$
\eject
{\bf D\'efinition}\ssk\new
{\it Soit $f:\rmat\vers\cmat$ une fonction continue par morceaux (c.p.m.) et int\'egrable sur $\rmat$. Pour tout $\lam\in\rmat$, on peut d\'efinir l'int\'egrale\vv
$$\fhat(\lam)=\int_{-\infty}^{\ii}f(t)\;e^{-i\lam t}\;dt\;.$$
La fonction $\fhat:\rmat\vers\cmat$ est la {\bf transform\'ee de Fourier} de $f$.}
\msk
Dans ce qui suit, la fonction $f$ est suppos\'ee continue par morceaux et de classe ${\cal C}^1$
par morceaux, int\'egrable sur $\rmat$.
On se propose de d\'emontrer la {\bf formule de r\'eciprocit\'e} suivante~:\vv
$$\a x\in\rmat\qquad {f(x^+)+f(x^-)\s2}={1\s2\pi}\;\Lim{A}{\ii}
  \int_{-A}^{A}\fhat(\lam)\;e^{ix\lam}\;d\lam\;.\eqno\hbox{\bf (*)}$$
\msk
{\bf 2.} Pour tous $n\in\nmat^*$ et $\lam\in\rmat$, on pose $F_n(\lam)=
\int_{-n}^nf(t)\;e^{-i\lam t}\;dt$.\pn
Montrer, pour tous $n\in\net$, $x\in\rmat$ et $A\in\rpe$, l'\'egalit\'e\vv
$$\int_{-A}^AF_n(\lam)\;e^{i\lam x}\;d\lam=2\;\int_{x-n}^{x+n}f(x-u)\;
  {\sin Au\s u}\;du\;.$$\ssk
{\bf 3.} En utilisant la question {\bf 1.}, montrer l'\'egalit\'e {\bf (*)} ci-dessus.

\msk
\cl{- - - - - - - - - - - - - - - - - - - - - - - - - - - - - - -}
\msk

{\bf 1.} L'int\'egrale $\;F(\lam)=\int_0^{\ii}f(t)\;{\sin\lam t\s t}\;dt\;$ est bien
d\'efinie pour tout $\lam\in\rmat$~: en effet, on a $\left|f(t)\>{\sin\lam t\s t}\right|\ie|\lam\>f(t)|$, donc la fonction $\;t\mapsto f(t)\>{\sin\lam t\s t}\;$ est int\'egrable sur $\rpe$ pour tout r\'eel $\lam$.
\msk\sect
Pour tout $\lam\in\rmat_+^*$, le changement de variable $x=\lam t$ donne
imm\'ediatement\vv
$$\int_0^{\ii}{\sin\lam t\s t}\;dt=\int_0^{\ii}{\sin x\s x}\;dx={\pi\s2}\;,
  \quad\hbox{donc}$$\ssk\sect
$F(\lam)-{\pi\s2}\>f(0^+)=\int_0^{\ii}\big(f(t)-f(0^+)\big)\;
                                           {\sin\lam t\s t}\;dt$
$$\qquad=\int_0^1{f(t)-f(0^+)\s t}\;\sin\lam t
                                           \;dt+\int_1^{\ii}{f(t)\s t}\;
                                           \sin\lam t\;dt-f(0^+)\;\int_{\lam}
                                           ^{\ii}{\sin x\s x}\;dx\;.$$
La fonction $t\mapsto{f(t)\s t}$ est int\'egrable
sur $[1,\ii[$ et la fonction $g$ est int\'egrable sur $]0,1]$.
Les deux premiers
termes tendent donc vers z\'ero lorsque $\lam$ tend vers $\ii$ (th\'eor\`eme de
Riemann-Lebesgue, {\it cf}. \`a la fin). Enfin, la (semi-)convergence de
l'int\'egrale $\int_0^{\ii}{\sin t\s t}\;dt$ montre que le troisi\`eme terme
aussi tend vers 0 lorsque $\lam$ tend vers $\ii$.
\eject
{\bf 2.} Pour tous $n\in\net$, $x\in\rmat$ et $A\in\rpe$, on a\vv
\begin{eqnarray*}
\int_{-A}^AF_n(\lam)\;e^{ix\lam}\;d\lam
             & = & \int_{-A}^A\lp\int_{-n}^nf(t)\;
                    e^{-i\lam t}\;dt\rp\;e^{ix\lam}\;d\lam\\
             & = & \int_{-n}^nf(t)\;\lp\int_{-A}^A
                    e^{i(x-t)\lam}\;d\lam\rp\;dt
\end{eqnarray*}
(cette interversion des int\'egrations est justifi\'ee par le th\'eor\`eme de
Fubini si $f$ est continue sur $[-n,n]$ et
reste valable si $f$ est
seulement continue par morceaux~: il suffit alors de d\'ecomposer
par la relation de Chasles en faisant intervenir les points de discontinuit\'e
de $f$ dans le segment $[-n,n]$). On a donc\vv
$$\int_{-A}^AF_n(\lam)\;e^{ix\lam}\;d\lam
              = 2\int_{-n}^nf(t)\;{\sin A(x-t)\s x-t}\;dt
              = 2\int_{x-n}^{x+n}f(x-u)\;{\sin Au\s u}\;du\;,$$
la fonction $u\mapsto{\sin Au\s u}$ \'etant \'evidemment prolong\'ee par
continuit\'e en z\'ero.

\msk
{\bf 3.} On en d\'eduit\vv
$$\int_{-A}^AF_n(\lam)\;e^{ix\lam}\;d\lam =
  2\;\lp\int_0^{n-x}f(x+v)\;{\sin Av\s v}\;dv
               + \int_0^{x+n}f(x-u)\;{\sin Au\s u}\; du\rp\;.$$
Pour $x$ et $A$ fix\'es, ces int\'egrales ont des limites finies lorsque $n$ tend vers $\ii$ car
$f$ est int\'egrable sur $\rmat$ et ${\sin Au\s u}$ (\'evidemment prolong\'e par
continuit\'e pour $u=0$) est born\'e.\psn
D'autre part, la majoration\vv
$$|\fhat(\lam)\;e^{ix\lam}-F_n(\lam)\;e^{ix\lam}|=
  |\fhat(\lam)-F_n(\lam)|\le\int_{-\infty}^{-n}|f|+\int_n^{\ii}|f|\;,$$
avec $f$ int\'egrable sur $\rmat$,
montre que la suite de fonctions $\;\lp\lam\mapsto F_n(\lam)\;e^{ix\lam}
\rp_{n\in\net}\;$ converge uniform\'ement sur $\rmat$ vers la fonction
$\lam\mapsto\fhat(\lam)\;e^{ix\lam}$.\pn
Pour tout $A\in\rpe$, posons $\;g(A)=\int_{-A}^A\fhat(\lam)\;e^{ix\lam}
\;d\lam$. On a donc\vv
\begin{eqnarray*}
g(A) & = & \Lim{n}{\ii}\int_{-A}^AF_n(\lam)\;e^{ix\lam}\;d\lam\\
                & = & 2\int_0^{\ii}f(x+u)\;{\sin Au\s u}\;du+2\int_0^{\ii}
                     f(x-u)\;  {\sin Au\s u}\;du\;.
\end{eqnarray*}
Or, si $f$ est c.p.m. et de classe ${\cal C}^1$ par morceaux, les ``taux
d'accroissement'' $\;{f(x+u)-f(x^+)\s u}\;$ et $\;{f(x-u)-f(x^-)\s u}\;$ ont
des limites finies lorsque $u$ tend vers z\'ero par valeurs sup\'erieures.
Les conditions d'application de la question {\bf 1.}
sont alors remplies, ce
qui permet d'\'ecrire que $\;\Lim{A}{\ii}g(A)=\pi\;\big(f(x^+)+f(x^-)\big)$.

\msk\sect
{\it Remarque. Sans hypoth\`ese suppl\'ementaire sur $f$, on a simplement
d\'emontr\'e l'existence d'une limite de l'expression
(``int\'egrale sym\'etrique'')
$\;g(A)=\int_{-A}^A\fhat(\lam)\;e^{ix\lam}\;d\lam\;$
lorsque $A$ tend vers $\ii$. Cela
n'implique pas la convergence (m\^eme la ``semi-convergence'') de
l'int\'egrale g\'en\'eralis\'ee $\;\int_{-\infty}^{\ii}\fhat(\lam)\;e^{ix\lam}
\;d\lam\;$~: en effet, les int\'egrales $\int_{-\infty}^0$ et
$\int_0^{\ii}$, consid\'er\'ees s\'epar\'ement, peuvent \^etre divergentes.\msk\sect
Sous les hypoth\`eses de cet exercice, en supposant
de plus $f$ continue sur $\rmat$, le ``signal''
$f$ peut \^etre enti\`erement retrouv\'e lorsqu'on conna\^\i t sa transform\'ee de
Fourier $\fhat$. Si on suppose de plus $\fhat$ int\'egrable sur $\rmat$
(ce qui peut r\'esulter d'hypoth\`eses de r\'egularit\'e faites sur la fonction $f$),
la formule de r\'eciprocit\'e de Fourier peut alors s'\'ecrire\vv
$$\a x\in\rmat\qquad f(x)={1\s2\pi}\>\int_{-\infty}^{\ii}\fhat(\lam)\>
  e^{ix\lam}\;d\lam\;,$$
soit\vv
$$\a x\in\rmat\qquad f(x)={1\s2\pi}\>\wh{\>\fhat\;}(-x)\;,\;
  \hbox{ou encore}\quad \wh{\>\fhat\;}(x)=2\pi\>f(-x)\;.$$}


Pour finir, voici l'\'enonc\'e et une preuve du {\bf th\'eor\`eme de Riemann-Lebesgue}~:\msk\new
Soit $I$ un intervalle de $\rmat$. Soit $f:I\vers\cmat$ une fonction
continue par morceaux et int\'egrable sur $I$.\ssk\new
Alors l'int\'egrale $\;\ftilde(\lam)=\int_If(t)\;e^{i\lam t}\,dt\;$ tend vers z\'ero lorsque le r\'eel $\lam$ tend vers $\ii$.
\pmn\sect
Preuve~:
L'existence de $\ftilde(\lam)$ r\'esulte trivialement de l'int\'egrabilit\'e de $f$
sur $I$.\psn
$\bullet$ Pla\c cons-nous d'abord dans le cas o\`u $I$ est un segment~:
$I=[a,b]$.\pn
$\triangleright$ si $f$ est la fonction caract\'eristique d'un intervalle $J=[\alpha,\beta]$
(ou $]\alpha,\beta]$ ou $[\alpha,\beta[$ ou $]\alpha,\beta[$) avec
$a\le\alpha\le\beta\le b$,alors\vv
$$|\ftilde(\lam)|=\left|\int_Je^{i\lam t}\;dt\right|=\left|{e^{i\lam\beta}
  -e^{i\lam\alpha}\s i\lam}\right|\le{2\s|\lam|}\;,$$
et le r\'esultat est \'evident.\pn
$\triangleright$ si $f$ est en escalier sur $[a,b]$, le r\'esultat est encore vrai car $f$
est combinaison lin\'eaire de fonctions caract\'eristiques d'intervalles.\pn
$\triangleright$ si $f$ est une fonction c.p.m. quelconque sur $[a,b]$, $f$ est limite
uniforme sur $I$ d'une suite de fonctions en escalier. Cela signifie que,
pour tout $\eps>0$, il existe une fonction $\ffi$, en escalier sur $[a,b]$
telle que $\;\a x\in[a,b]\quad|f(x)-\ffi(x)|\le{\eps\s2(b-a)}$.
On a alors
$\int_I|f-\ffi|\le{\eps\s2}$. Donc, pour tout $\lam\in\rmat$,
$$|\ftilde(\lam)-\tilde\ffi(\lam)|=\left|\int_I\big(f(t)-\ffi(t)\big)\;
  e^{i\lam t}\;dt\right|\le\int_I|f-\ffi|\le{\eps\s2}\;.$$
Puisque $\ffi$ est en escalier, on peut trouver un r\'eel $\Lambda$ tel que,
pour $\lam\ge\Lambda$, on ait $|\tilde\ffi(\lam)|\le{\eps\s2}$ et donc
$|\ftilde(\lam)|\le\eps$.\pmn
$\bullet$ Soit maintenant $I$ un intervalle quelconque de $\rmat$.
Si on se donne $\eps>0$, on peut trouver un
segment $J$ inclus dans $I$ tel que $\int_K|f|\le{\eps\s2}$,
en posant $K=I\setminus J$ ($K$ est, soit un intervalle, soit la
r\'eunion de deux intervalles). Alors\vv
$$\ftilde(\lam)=\int_Jf(t)\;e^{i\lam t}\;dt+
  \int_Kf(t)\;e^{i\lam t}\;dt\;,$$
d'o\`u l'on tire $|\ftilde(\lam)|\le\left|\int_Jf(t)\;e^{i\lam t}\;dt
\right|+{\eps\s2}$. Or, il r\'esulte de l'\'etude faite sur un segment que
$\Lim{\lam}{\ii}\int_Jf(t)\;e^{i\lam t}\;dt=0$~;
on peut alors trouver $\Lambda$ tel que, pour $\lam\ge\Lambda$, on ait
$\left|\int_Jf(t)\;e^{i\lam t}\;dt\right|\le{\eps\s2}$. Pour
$\lam\ge\Lambda$, on aura alors $|\ftilde(\lam)|\le\eps$.
\psn
{\it Remarque. Lorsque $f$ est une fonction de classe $\ceun$ sur un segment
$[a,b]$, une int\'egration par parties, puis une majoration des diff\'erents
termes obtenus, per\-mettent de conclure plus simplement.}


































\end{document}