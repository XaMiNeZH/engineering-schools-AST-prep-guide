\documentclass{article}
\begin{document}

\parindent=-8mm\leftskip=8mm
\def\new{\par\hskip 8.3mm}
\def\sect{\par\quad}
\hsize=147mm  \vsize=230mm
\hoffset=-10mm\voffset=0mm

\everymath{\displaystyle}       % \'evite le textstyle en mode
                                % math\'ematique

\font\itbf=cmbxti10

\let\dis=\displaystyle          %raccourci
\let\eps=\varepsilon            %raccourci
\let\vs=\vskip                  %raccourci


\frenchspacing

\let\ie=\leq
\let\se=\geq



\font\pc=cmcsc10 % petites capitales (aussi cmtcsc10)

\def\tp{\raise .2em\hbox{${}^{\hbox{\seveni t}}\!$}}%



\font\info=cmtt10




%%%%%%%%%%%%%%%%% polices grasses math\'ematiques %%%%%%%%%%%%
\font\tenbi=cmmib10 % bold math italic
\font\sevenbi=cmmi7% scaled 700
\font\fivebi=cmmi5 %scaled 500
\font\tenbsy=cmbsy10 % bold math symbols
\font\sevenbsy=cmsy7% scaled 700
\font\fivebsy=cmsy5% scaled 500
%%%%%%%%%%%%%%% polices de pr\'esentation %%%%%%%%%%%%%%%%%
\font\titlefont=cmbx10 at 20.73pt
\font\chapfont=cmbx12
\font\secfont=cmbx12
\font\headfont=cmr7
\font\itheadfont=cmti7% at 6.66pt



% divers
\def\euler{\cal}
\def\goth{\cal}
\def\phi{\varphi}
\def\epsilon{\varepsilon}

%%%%%%%%%%%%%%%%%%%%  tableaux de variations %%%%%%%%%%%%%%%%%%%%%%%
% petite macro d'\'ecriture de tableaux de variations
% syntaxe:
%         \variations{t    && ... & ... & .......\cr
%                     f(t) && ... & ... & ...... \cr
%
%etc...........}
% \`a l'int\'erieur de cette macro on peut utiliser les macros
% \croit (la fonction est croissante),
% \decroit (la fonction est d\'ecroissante),
% \nondef (la fonction est non d\'efinie)
% si l'on termine la derni\`ere ligne par \cr, un trait est tir\'e en dessous
% sinon elle est laiss\'ee sans trait
%%%%%%%%%%%%%%%%%%%%%%%%%%%%%%%%%%%%%%%%%%%%%%%%%%%%%%%%%%%%%%%%%%%

\def\variations#1{\par\medskip\centerline{\vbox{{\offinterlineskip
            \def\decroit{\searrow}
    \def\croit{\nearrow}
    \def\nondef{\parallel}
    \def\tableskip{\omit& height 4pt & \omit \endline}
    % \everycr={\noalign{\hrule}}
            \def\cr{\endline\tableskip\noalign{\hrule}\tableskip}
    \halign{
             \tabskip=.7em plus 1em
             \hfil\strut $##$\hfil &\vrule ##
              && \hfil $##$ \hfil \endline
              #1\crcr
           }
 }}}\medskip}   

%%%%%%%%%%%%%%%%%%%%%%%% NRZCQ %%%%%%%%%%%%%%%%%%%%%%%%%%%%
\def\nmat{{\rm I\kern-0.5mm N}}  
\def\rmat{{\rm I\kern-0.6mm R}}  
\def\cmat{{\rm C\kern-1.7mm\vrule height 6.2pt depth 0pt\enskip}}  
\def\zmat{\mathop{\raise 0.1mm\hbox{\bf Z}}\nolimits}
\def\qmat{{\rm Q\kern-1.8mm\vrule height 6.5pt depth 0pt\enskip}}  
\def\dmat{{\rm I\kern-0.6mm D}}
\def\lmat{{\rm I\kern-0.6mm L}}
\def\kmat{{\rm I\kern-0.7mm K}}

%___________intervalles d'entiers______________
\def\[ent{[\hskip -1.5pt [}
\def\]ent{]\hskip -1.5pt ]}
\def\rent{{\bf ]}\hskip -2pt {\bf ]}}
\def\lent{{\bf [}\hskip -2pt {\bf [}}

%_____d\'ef de combinaison
\def\comb{\mathop{\hbox{\large C}}\nolimits}

%%%%%%%%%%%%%%%%%%%%%%% Alg\`ebre lin\'eaire %%%%%%%%%%%%%%%%%%%%%
%________image_______
\def\im{\mathop{\rm Im}\nolimits}
%________d\'eterminant_______
\def\det{\mathop{\rm det}\nolimits} 
\def\Det{\mathop{\rm Det}\nolimits}
\def\diag{\mathop{\rm diag}\nolimits}
%________rang_______
\def\rg{\mathop{\rm rg}\nolimits}
%________id_______
\def\id{\mathop{\rm id}\nolimits}
\def\tr{\mathop{\rm tr}\nolimits}
\def\Id{\mathop{\rm Id}\nolimits}
\def\Ker{\mathop{\rm Ker}\nolimits}
\def\bary{\mathop{\rm bar}\nolimits}
\def\card{\mathop{\rm card}\nolimits}
\def\Card{\mathop{\rm Card}\nolimits}
\def\grad{\mathop{\rm grad}\nolimits}
\def\Vect{\mathop{\rm Vect}\nolimits}
\def\Log{\mathop{\rm Log}\nolimits}

%________GL_______
\def\GLR#1{{\rm GL}_{#1}(\rmat)}  
\def\GLC#1{{\rm GL}_{#1}(\cmat)}  
\def\GLK#1#2{{\rm GL}_{#1}(#2)}
\def\SO{\mathop{\rm SO}\nolimits}
\def\SDP#1{{\cal S}_{#1}^{++}}
%________spectre_______
\def\Sp{\mathop{\rm Sp}\nolimits}
%_________ transpos\'ee ________
%\def\t{\raise .2em\hbox{${}^{\hbox{\seveni t}}\!$}}
\def\t{\,{}^t\!\!}

%_______M gothL_______
\def\MR#1{{\cal M}_{#1}(\rmat)}  
\def\MC#1{{\cal M}_{#1}(\cmat)}  
\def\MK#1{{\cal M}_{#1}(\kmat)}  

%________Complexes_________ 
\def\Re{\mathop{\rm Re}\nolimits}
\def\Im{\mathop{\rm Im}\nolimits}

%_______cal L_______
\def\L{{\euler L}}

%%%%%%%%%%%%%%%%%%%%%%%%% fonctions classiques %%%%%%%%%%%%%%%%%%%%%%
%________cotg_______
\def\cotan{\mathop{\rm cotan}\nolimits}
\def\cotg{\mathop{\rm cotg}\nolimits}
\def\tg{\mathop{\rm tg}\nolimits}
%________th_______
\def\tanh{\mathop{\rm th}\nolimits}
\def\th{\mathop{\rm th}\nolimits}
%________sh_______
\def\sinh{\mathop{\rm sh}\nolimits}
\def\sh{\mathop{\rm sh}\nolimits}
%________ch_______
\def\cosh{\mathop{\rm ch}\nolimits}
\def\ch{\mathop{\rm ch}\nolimits}
%________log_______
\def\log{\mathop{\rm log}\nolimits}
\def\sgn{\mathop{\rm sgn}\nolimits}

\def\Arcsin{\mathop{\rm Arcsin}\nolimits}   
\def\Arccos{\mathop{\rm Arccos}\nolimits}  
\def\Arctan{\mathop{\rm Arctan}\nolimits}   
\def\Argsh{\mathop{\rm Argsh}\nolimits}     
\def\Argch{\mathop{\rm Argch}\nolimits}     
\def\Argth{\mathop{\rm Argth}\nolimits}     
\def\Arccotan{\mathop{\rm Arccotan}\nolimits}
\def\coth{\mathop{\rm coth}\nolimits}
\def\Argcoth{\mathop{\rm Argcoth}\nolimits}
\def\E{\mathop{\rm E}\nolimits}
\def\C{\mathop{\rm C}\nolimits}

\def\build#1_#2^#3{\mathrel{\mathop{\kern 0pt#1}\limits_{#2}^{#3}}} 

%________classe C_________
\def\C{{\cal C}}
%____________suites et s\'eries_____________________
\def\suiteN #1#2{(#1 _#2)_{#2\in \nmat }}  
\def\suite #1#2#3{(#1 _#2)_{#2\ge#3 }}  
\def\serieN #1#2{\sum_{#2\in \nmat } #1_#2}  
\def\serie #1#2#3{\sum_{#2\ge #3} #1_#2}  

%___________norme_________________________
\def\norme#1{\|{#1}\|}  
\def\bignorme#1{\left|\hskip-0.9pt\left|{#1}\right|\hskip-0.9pt\right|}

%____________vide (perso)_________________
\def\vide{\hbox{\O }}
%____________partie
\def\P{{\cal P}}

%%%%%%%%%%%%commandes abr\'eg\'ees%%%%%%%%%%%%%%%%%%%%%%%
\let\lam=\lambda
\let\ddd=\partial
\def\bsk{\vspace{12pt}\par}
\def\msk{\vspace{6pt}\par}
\def\ssk{\vspace{3pt}\par}
\let\noi=\noindent
\let\eps=\varepsilon
\let\ffi=\varphi
\let\vers=\rightarrow
\let\srev=\leftarrow
\let\impl=\Longrightarrow
\let\tst=\textstyle
\let\dst=\displaystyle
\let\sst=\scriptstyle
\let\ssst=\scriptscriptstyle
\let\divise=\mid
\let\a=\forall
\let\e=\exists
\let\s=\over
\def\vect#1{\overrightarrow{\vphantom{b}#1}}
\let\ov=\overline
\def\eu{\e !}
\def\pn{\par\noi}
\def\pss{\par\ssk}
\def\pms{\par\msk}
\def\pbs{\par\bsk}
\def\pbn{\bsk\noi}
\def\pmn{\msk\noi}
\def\psn{\ssk\noi}
\def\nmsk{\noalign{\msk}}
\def\nssk{\noalign{\ssk}}
\def\equi_#1{\build\sim_#1^{}}
\def\lp{\left(}
\def\rp{\right)}
\def\lc{\left[}
\def\rc{\right]}
\def\lci{\left]}
\def\rci{\right[}
\def\Lim#1#2{\lim_{#1\vers#2}}
\def\Equi#1#2{\equi_{#1\vers#2}}
\def\Vers#1#2{\quad\build\longrightarrow_{#1\vers#2}^{}\quad}
\def\Limg#1#2{\lim_{#1\vers#2\atop#1<#2}}
\def\Limd#1#2{\lim_{#1\vers#2\atop#1>#2}}
\def\lims#1{\Lim{n}{+\infty}#1_n}
\def\cl#1{\par\centerline{#1}}
\def\cls#1{\pss\centerline{#1}}
\def\clm#1{\pms\centerline{#1}}
\def\clb#1{\pbs\centerline{#1}}
\def\cad{\rm c'est-\`a-dire}
\def\ssi{\it si et seulement si}
\def\lac{\left\{}
\def\rac{\right\}}
\def\ii{+\infty}
\def\eg{\rm par exemple}
\def\vv{\vskip -2mm}
\def\vvv{\vskip -3mm}
\def\vvvv{\vskip -4mm}
\def\union{\;\cup\;}
\def\inter{\;\cap\;}
\def\sur{\above .2pt}
\def\tvi{\vrule height 12pt depth 5pt width 0pt}
\def\tv{\vrule height 8pt depth 5pt width 1pt}
\def\rplus{\rmat_+}
\def\rpe{\rmat_+^*}
\def\rdeux{\rmat^2}
\def\rtrois{\rmat^3}
\def\net{\nmat^*}
\def\ret{\rmat^*}
\def\cet{\cmat^*}
\def\rbar{\ov{\rmat}}
\def\deter#1{\left|\matrix{#1}\right|}
\def\intd{\int\!\!\!\int}
\def\intt{\int\!\!\!\int\!\!\!\int}
\def\ce{{\cal C}}
\def\ceun{{\cal C}^1}
\def\cedeux{{\cal C}^2}
\def\ceinf{{\cal C}^{\infty}}
\def\zz#1{\;{\raise 1mm\hbox{$\zmat$}}\!\!\Bigm/{\raise -2mm\hbox{$\!\!\!\!#1\zmat$}}}
\def\interieur#1{{\buildrel\circ\over #1}}
%%%%%%%%%%%% c'est la fin %%%%%%%%%%%%%%%%%%%%%%%%%%%

\def\boxit#1#2{\setbox1=\hbox{\kern#1{#2}\kern#1}%
\dimen1=\ht1 \advance\dimen1 by #1 \dimen2=\dp1 \advance\dimen2 by #1
\setbox1=\hbox{\vrule height\dimen1 depth\dimen2\box1\vrule}%
\setbox1=\vbox{\hrule\box1\hrule}%
\advance\dimen1 by .4pt \ht1=\dimen1
\advance\dimen2 by .4pt \dp1=\dimen2 \box1\relax}


\catcode`\@=11
\def\system#1{\left\{\null\,\vcenter{\openup1\jot\m@th
\ialign{\strut\hfil$##$&$##$\hfil&&\enspace$##$\enspace&
\hfil$##$&$##$\hfil\crcr#1\crcr}}\right.}
\catcode`\@=12
\pagestyle{empty}
\def\lap#1{{\cal L}[#1]}
\def\DP#1#2{{\partial#1\s\partial#2}}
\def\cala{{\cal A}}
\def\fhat{\widehat{f}}
\let\wh=\widehat
\def\ftilde{\tilde{f}}

% ********************************************************************************************************************** %
%                                                                                                                                                                                   %
%                                                                    FIN   DES   MACROS                                                                              %
%                                                                                                                                                                                   %
% ********************************************************************************************************************** %










\def\lap#1{{\cal L}[#1]}
\def\DP#1#2{{\partial#1\s\partial#2}}



\overfullrule=0mm


\cl{{\bf SEMAINE 20}}\msk
\cl{{\bf INT\'EGRALES CURVILIGNES}}\msk
\cl{{\bf  \'EQUATIONS DIFF\'ERENTIELLES LIN\'EAIRES}}
\bsk

{\bf EXERCICE 1 :}\msk
{\bf 1.} Soit $F\in\cmat(X)$ une fraction rationnelle n'ayant aucun p\^ole de module 1. On note $\Gamma$ le cercle unit\'e parcouru dans le sens direct. Calculer l'int\'egrale curviligne\vv
$$I=\int_{\Gamma}F(z)\>{\rm d}z\;.$$\par
{\bf 2.} En d\'eduire les int\'egrales simples\vv
$$J_n(r)=\int_0^{2\pi}{\cos nt\s1+r^2-2r\>\cos t}\>{\rm d}t\qquad(n\in\nmat\;,\;r\in\rpe\;,\;r\not=1)\;.$$\par
{\bf 3.} Retrouver aussi les int\'egrales de Wallis\vv
$$W_n=\int_0^{2\pi}\cos^n\theta\>{\rm d}\theta\;.$$




\msk
\cl{- - - - - - - - - - - - - - - - - - - - - - - - - - - - - -}
\msk

{\bf 1.} Il suffit de savoir faire le calcul lorsque $F$ est un \'el\'ement simple~: $F(z)={1\s(z-a)^k}$. Posons pour cela $\gamma(t)=e^{it}$ ($0\ie t\ie 2\pi$, param\'etrage de l'arc $\Gamma$). On a alors, pour $k>1$,\vv
$$I=\int_{\Gamma}{{\rm d}z\s(z-a)^k}=\int_0^{2\pi}{\gamma'(t)\>{\rm d}t\s\big(\gamma(t)-a\big)^k}=\lc{1\s1-k}\>{1\s\big(\gamma(t)-a\big)^{k-1}}\rc_0^{2\pi}=0$$
car $\gamma(0)=\gamma(2\pi)$.\msk\sect
Pour $k=1$, on ne dispose pas d'expression d'une primitive de la fonction \`a valeurs complexes $t\mapsto{\gamma'(t)\s\gamma(t)-a}$. Une solution consiste \`a poser $a=\alpha+i\beta$ et \`a consid\'erer l'int\'egrale\break $I(a)=I(\alpha,\beta)=\int_{\Gamma}{{\rm d}z\s z-a}\;$ comme une fonction des deux variables r\'eelles $\alpha$ et $\beta$. On a $\;I(\alpha,\beta)=\int_0^{2\pi}{\gamma'(t)\>{\rm d}t\s\gamma(t)-\alpha-i\beta}$. Sur chacun des deux ouverts connexes par arcs\break $U=\{z\in\cmat\;;\;|z|<1\}$ et $V=\{z\in\cmat\;;\;|z|>1\}$, la fonction $I$ des deux variables r\'eelles $\alpha$ et $\beta$ est de classe ${\cal C}^1$ et\vv
$${\ddd I\s\ddd\alpha}(\alpha,\beta)=\int_0^{2\pi}{\gamma'(t)\>{\rm d}t\s\big(\gamma(t)-\alpha-i\beta\big)^2}=\lc{-1\s\gamma(t)-\alpha-i\beta}\rc_0^{2\pi}=0\;;$$
$${\ddd I\s\ddd\beta}(\alpha,\beta)=\int_0^{2\pi}{i\>\gamma'(t)\>{\rm d}t\s\big(\gamma(t)-\alpha-i\beta\big)^2}=\lc{-i\s\gamma(t)-\alpha-i\beta}\rc_0^{2\pi}=0\;.$$
La fonction $I$ est donc constante sur chacun des deux ouverts $U$ et $V$. De\vv
$$I(0)=\int_0^{2\pi}{\gamma'(t)\s\gamma(t)}\>{\rm d}t=\int_0^{2\pi}{i\>e^{it}\s e^{it}}\>{\rm d}t=2i\pi\;,$$
on d\'eduit $I(a)=2i\pi$ si $|a|<1$.\ssk\new
Si $|a|>1$, \'ecrivons $|z-a|\se\big||z|-|a|\big|=|a|-1$, donc $|I(a)|\ie{2\pi\s|a|-1}$ et $\;\Lim{|a|}{\ii}I(a)=0$, donc $I(a)=0$ pour $|a|>1$.\msk\sect
Enfin, si $P\in\cmat[X]$ est un polyn\^ome, on a facilement $\int_{\Gamma}P(z)\>{\rm d}z=0$.\msk\sect
En conclusion, si $F$ est une fraction rationnelle quelconque (sans p\^ole de module un), on d\'ecompose $F$ en \'el\'ements simples~: la partie enti\`ere n'apporte aucune contribution \`a l'int\'egrale $I$~; dans la partie polaire en un p\^ole $a$ (de la forme ${c_{a,1}\s z-a}+\cdots+{c_{a,r}\s(z-a)^r}$ si le p\^ole est d'ordre $r$), aucun terme n'apporte de contribution \`a l'int\'egrale si $|a|>1$ et seul le terme ${c_{a,1}\s z-a}$ apporte une contribution non nulle si $|a|<1$. Le coefficient $c_{a,1}$ est appel\'e le {\bf r\'esidu} de la fraction rationnelle $F$ au p\^ole $a$ et not\'e ${\rm Res}(F,a)$. On a donc la formule\vv
$$I=\int_{\Gamma}F(z)\>{\rm d}z=2i\pi\cdot\sum_{|a|<1}{\rm Res}(F,a)\;.$$
\ssk
{\bf 2.} On a $\;J_n(r)=\Re\big(I_n(r)\big)$, avec $I_n(r)=\int_0^{2\pi}{e^{int}\>{\rm d}t\s1+r^2-2r\>\cos t}$. Or,\vv
$$I_n(r)=\int_0^{2\pi}{e^{int}\>{\rm d}t\s(1-r\>e^{it})(1-r\>e^{-it})}={1\s i}\>\int_{\Gamma}{z^{n-1}\>{\rm d}z\s(1-rz)\lp1-\dst{r\s z}\rp}=-{1\s ir}\>\int_{\Gamma}{z^n\s(z-r)\lp z-\dst{1\s r}\rp}\>{\rm d}z\;,$$
d'o\`u la discussion~:\ssk\new
$\bullet$ si $0<r<1$, le seul p\^ole de la fraction rationnelle $F={X^n\s(X-r)\lp X-\dst{1\s r}\rp}$ dans le disque unit\'e ouvert est $r$ (p\^ole simple), et on obtient facilement son r\'esidu $\;{\rm Res}(F,r)={r^{n+1}\s r^2-1}$, donc\vv
$$J_n(r)=I_n(r)={2\pi\>r^n\s 1-r^2}\;.$$
$\bullet$ si $r>1$, le seul p\^ole de module inf\'erieur \`a 1 est ${1\s r}$, de r\'esidu ${1\s r^{n-1}(1-r^2)}$, donc
$$J_n(r)=I_n(r)={2\pi\s r^n(r^2-1)}\;.$$

\ssk
{\bf 3.} On a\vv
$$W_n={1\s2^n}\>\int_0^{2\pi}\lp e^{i\theta}+{1\s e^{i\theta}}\rp^n\>{\rm d}\theta={1\s i\>2^n}\>\int_{\Gamma}\lp z+{1\s z}\rp^n\>{{\rm d}z\s z}={1\s i\>2^n}\>\int_{\Gamma}{(z^2+1)^n\s z^{n+1}}\>{\rm d}z\;.$$
Le seul p\^ole de la fraction $F={1\s i\>2^n}\>{(X^2+1)^n\s X^{n+1}}$ est z\'ero (de module $<1$), le r\'esidu de $F$ en ce p\^ole est le coefficient de $X^n$ dans le d\'eveloppement de\vv
$${1\s i\>2^n}\>(X^2+1)^n={1\s i\>2^n}\>\sum_{k=0}^nC_n^kX^{2k}\;.$$
Il est donc nul si $n$ est impair, et vaut ${C_{2p}^p\s i\>2^p}$ si $n=2p$, donc
$$W_{2p+1}=0\qquad{\rm et}\qquad W_{2p}=2\pi\>{C_{2p}^p\s2^{2p}}=2\pi\>{(2p)!\s2^{2p}\>(p!)^2}\;.$$


\hrule
\eject

{\bf EXERCICE 2 :}\msk
{\bf 1.} Soit $g:x\mapsto\int_0^{\ii}{e^{-xt}\s1+t^2}\>{\rm d}t$.\msk\sect
Montrer que la fonction $g$ est continue sur $\rplus$, de classe ${\cal C}^{\infty}$ sur $\rpe$. \'Ecrire une \'equation diff\'erentielle {\bf (E)} v\'erifi\'ee par $g$ sur $\rpe$.\msk
{\bf 2.} En d\'eduire la valeur de $I=\int_0^{\ii}{\sin t\s t}\>{\rm d}t$.


\msk
\cl{- - - - - - - - - - - - - - - - - - - - - - - - - - - - - -}
\msk

{\bf 1.} Posons $f(x,t)={e^{-xt}\s1+t^2}$ pour $x\se0$ et $t\se0$. La fonction $f$ est continue sur $(\rplus)^2$ et on a $0\ie f(x,t)\ie{1\s1+t^2}$, la fonction $t\mapsto{1\s1+t^2}$ \'etant int\'egrable sur $[0,\ii[$~; on en d\'eduit l'existence et la continuit\'e de $g$ sur $\rplus$.\msk\sect
Pour tout $k\in\net$, on a $\;{\ddd^k f\s\ddd x^k}(x,t)=(-1)^k\>{t^k\>e^{-xt}\s1+t^2}\;$ et, si on se donne $a>0$, on a, pour $x\se a$ et $t\se0$,\vv
$$\left|{\ddd^k f\s\ddd x^k}(x,t)\right|\ie{t^k\s1+t^2}\>e^{-at}\ie t^k\>e^{-at}\;,$$
la fonction $t\mapsto t^k\>e^{-at}$ \'etant int\'egrable sur $\rplus$. On en d\'eduit que $g$ est de classe ${\cal C}^{\infty}$ sur $\rpe$, avec $\;g^{(k)}(x)=(-1)^k\>\int_0^{\ii}{t^k\>e^{-xt}\s1+t^2}\>{\rm d}t\;$ pour tout $k\in\net$.\msk\sect
On observe notamment que $\;g''(x)+g(x)=\int_0^{\ii}e^{-xt}\>{\rm d}t={1\s x}$, donc $g$ est solution sur $\rpe$ de l'\'equation diff\'erentielle\vv
$${\bf (E)}\;:\qquad y''+y={1\s x}\;.$$
\ssk
{\bf 2.} L'\'equation homog\`ene {\bf (E}$_0${\bf)}~: $y''+y=0$ admet $(\sin,\cos)$ comme syst\`eme fondamental de solutions. Utilisons la m\'ethode de variation des constantes pour exprimer les solutions de {\bf (E)}. Nous posons $\;y(x)=u(x)\>\sin x+v(x)\>\cos x\;$ en imposant $\;u'(x)\>\sin x+v'(x)\>\cos x=0$. L'\'equation {\bf (E)} devient $\;u'(x)\>\cos x-v'(x)\>\sin x={1\s x}$.\msk\sect
On r\'esout le syst\`eme $\system{&u'(x)&\sin x&+&&v'(x)&\cos x&&=&0\cr &u'(x)&\cos x&-&&v'(x)&\sin x&&=&{1\s x}\cr}$, ce qui donne $\;u'(x)={\cos x\s x}\;$ et $\;v'(x)=-{\sin x\s x}$. On sait que les int\'egrales $\int_1^{\ii}{\sin t\s t}\>{\rm d}t\;$ et $\int_1^{\ii}{\cos t\s t}\>{\rm d}t\;$ sont\break (semi-)convergentes. Une solution particuli\`ere de {\bf (E)} peut donc \^etre exprim\'ee sous la forme
$$y_0:x\mapsto\lp\int_x^{\ii}{\sin t\s t}\>{\rm d}t\rp\>\cos x-\lp\int_x^{\ii}{\cos t\s t}\>{\rm d}t\rp\>\sin x=\int_x^{\ii}{\sin(t-x)\s t}\>{\rm d}t\;.$$\sect
Les solutions de {\bf (E)} sur $\rpe$ sont donc les fonctions de la forme $y=A\>\cos x+B\>\sin x+y_0(x)$~; en particulier, la fonction $g$ de la question {\bf 1.} est de cette forme-l\`a. On a $\lim_{\ii}g=0$ car $\;|g(x)|\ie\int_0^{\ii}e^{-xt}\>{\rm d}t={1\s x}$~; comme $\lim_{\ii}y_0=0$, cela entra\^\i ne $A=B=0$. Finalement, $g=y_0$.\msk\sect
Enfin, $\Lim{x}{0}g(x)=g(0)={\pi\s2}$, alors que\ssk\new
$\bullet$ $\;\Lim{x}{0}\lp\cos x\cdot\int_x^{\ii}{\sin t\s t}\>{\rm d}t\rp=I$~;\ssk\new
$\bullet$ $\;\Lim{x}{0}\lp\sin x\cdot\int_x^{\ii}{\cos t\s t}\>{\rm d}t\rp=0$.\msk\sect
Pour ce deuxi\`eme point en effet, $\Lim{x}{0}\lp\sin x\cdot\int_1^{\ii}{\cos t\s t}\>{\rm d}t\rp=0\;$ et (pour $x<1$),
$$\left|\sin x\cdot\int_x^1{\cos t\s t}\>{\rm d}t\right|\ie x\>\int_x^1{{\rm d}t\s t}=x\>|\ln x|\Vers{x}{0}0\;.$$
\sect
On obtient donc $I=\int_0^{\ii}{\sin t\s t}\>{\rm d}t={\pi\s2}$.

\bsk
\hrule
\bsk

{\bf EXERCICE 3 :}\msk
{\bf 1.} Soit $g(x)=\int_0^{\ii}{e^{-xt^2}\s1+t^2}\>{\rm d}t$. Montrer que $g$ est solution, sur $\rpe$, d'une \'equation diff\'erentielle du premier ordre. En d\'eduire la valeur de l'int\'egrale de Gauss $\;G=\int_0^{\ii}e^{-t^2}\>{\rm d}t$.\msk
{\bf 2.} Soit $\ffi$ la solution, sur $\rpe$, de l'\'equation diff\'erentielle\vv
$$y''-2y'+y={1\s\sqrt{x}}\leqno\hbox{\bf (E)}$$
telle que $\lim_{0}\ffi=\lim_{0}\ffi'=0$. Donner un \'equivalent de $\ffi$ en $\ii$.




\msk
\cl{- - - - - - - - - - - - - - - - - - - - - - - - - - - - - - - }
\msk

{\bf 1.} La fonction $g$ est d\'efinie sur $\rmat_+$ et, si l'on pose $f(x,t)=
{e^{-xt^2}\s1+t^2}$ pour $x\ge0$ et $t\ge0$, la fonction $f$ est continue sur
$(\rplus)^2$ et la majoration
$0\le f(x,t)\le{1\s1+t^2}$ prouve que $g$ est continue sur $\rmat_+$.\pn
Pour tout $(x,t)\in(\rmat_+)^2$, on a ${\ddd f\s\ddd x}(x,t)={-t^2\;e^{-xt^2}\s
1+t^2}$. Si $a$ est un r\'eel strictement positif, on a\vv
$$\a(x,t)\in[a,\ii[\times\rmat_+\qquad\left|{\ddd f\s\ddd x}(x,t)\right|\le
{t^2\s1+t^2}\;e^{-at^2}\;,$$
cette derni\`ere fonction de la variable $t$ \'etant int\'egrable sur $\rmat_+$.
On en d\'eduit que $g$ est de classe ${\cal C}^1$ sur $[a,\ii[$ pour tout
$a>0$, donc sur $\rmat_+^*$, et que\vv
$$\a x\in\rmat_+^*\qquad g'(x)=-\int_0^{\ii}{t^2\;e^{-xt^2}\s1+t^2}\>{\rm d}t
=g(x)-\int_0^{\ii}e^{-xt^2}\>{\rm d}t\;.$$
Le changement de variable $t\,\sqrt{x}=u$ dans cette derni\`ere int\'egrale
montre que $g$ v\'erifie, sur $\rmat_+^*$, l'\'equation diff\'erentielle\vvvv
$$g'(x)-g(x)=-{G\s\sqrt{x}}\;.$$
En posant $g(x)=\lambda(x)\;e^x$ (m\'ethode de ``variation de la constante''),
on obtient\vv
$$\lambda(x)=-G\;\int_0^x{e^{-t}\s\sqrt{t}}\;{\rm d}t+C\;.$$
La fonction $\lambda:x\mapsto e^{-x}\;g(x)$ \'etant continue sur $\rmat_+$,
de $\lambda(0)=g(0)={\pi\s2}$, on tire $C={\pi\s2}$ et\vv
$$g(x)=e^x\;\lambda(x)=e^x\;\lp{\pi\s2}-G\;\int_0^x{e^{-t}\s\sqrt{t}}
  \;{\rm d}t\rp\;.$$
Enfin, la majoration imm\'ediate $0\le\lambda(x)=e^{-x}\;
\int_0^{\ii}{e^{-xt^2}\s1+t^2}\;{\rm d}t\le{\pi\s2}\;e^{-x}$ montre que
$\Lim{x}{\ii}\lambda(x)=0$, donc
$${\pi\s2}=G\;\int_0^{\ii}{e^{-t}\s\sqrt{t}}\;{\rm d}t=G\;\int_0^{\ii}
  {e^{-u^2}\s u}\;2u\;{\rm d}u=2G^2\;.$$
Comme $G$ est positif, on conclut
$$\int_0^{\ii}e^{-u^2}\;{\rm d}u={\sqrt{\pi}\s2}\;.$$

\par
{\bf 2.} Les solutions de l'\'equation sans second membre associ\'ee sont $(\lam x+\mu)\>e^x$.\ssk\sect Utilisons la m\'ethode de variation des constantes pour exprimer les solutions de {\bf (E)}. Posons donc $\;y(x)=\big(\lam(x)+x\>\mu(x)\big)\>e^x$, avec la condition $\lam'(x)+x\mu'(x)=0$. On d\'erive deux fois~: $y'=(\lam+\mu+\mu x)\>e^x$, puis $\;y''=(\lam+2\mu+\mu'+\mu x)\>e^x$. Ainsi,\vv
$${\bf (E)}\iff \mu'(x)\>e^x={1\s\sqrt{x}}\iff\mu'(x)={e^{-x}\s\sqrt{x}}\;,\qquad{\rm puis}\quad \lam'(x)=-\sqrt{x}\>e^{-x}\;.$$
Les conditions initiales sont alors $\lam(0)=0$ et $\lam(0)+\mu(0)=0$, donc $\lam(0)=\mu(0)=0$ et\vv
$$\mu(x)=\int_0^x{e^{-t}\s\sqrt{t}}\>{\rm d}t\qquad{\rm et}\quad \lam(x)=-\int_0^x\sqrt{t}\>e^{-t}\>{\rm d}t=\sqrt{x}\>e^{-x}-\int_0^x{e^{-t}\s2\sqrt{t}}\>{\rm d}t$$
donc\vvv
$$\ffi(x)=\sqrt{x}+\lp x-{1\s2}\rp\>e^x\>\int_0^x{e^{-t}\s\sqrt{t}}\>{\rm d}t=\sqrt{x}+(2x-1)\>e^x\>\int_0^{\sqrt{x}} e^{-u^2}\>{\rm d}u\;.$$
Lorsque $x$ tend vers $\ii$, on a donc\vv
$$\ffi(x)=\sqrt{x}+(2x-1)\>e^x\>\big(G+o(1)\big)=2Gxe^x+o(x\>e^x)\;,\qquad{\rm soit}\quad\ffi(x)\Equi{x}{\ii}\sqrt{\pi}\>x\>e^x\;.$$

\bsk
\hrule
\eject

{\bf EXERCICE 4 :}\msk
{\bf 1. Lemme de Gronwall}\msk
Soit $c$ un r\'eel, soient $u$ et $v$ deux fonctions continues sur $\rplus$, $u$ \`a valeurs r\'eelles, $v$ \`a valeurs positives ou nulles. On suppose\vv
$$\a x\in\rplus\qquad u(x)\ie c+\int_0^xu(t)\>v(t)\>{\rm d}t\;.$$
D\'emontrer que\vv
$$\a x\in\rplus\qquad u(x)\ie c\cdot\exp\lp\int_0^xv(t)\>{\rm d}t\rp\;.$$
\par
{\bf 2.} Soit $p:\rplus\vers\rmat$ une fonction continue, int\'egrable sur $\rplus$. Montrer que les solutions de l'\'equation diff\'erentielle\vvvv
$$y''+\big(1-p(x)\big)\>y=0\leqno\hbox{\bf (E)}$$\vv\new
sont born\'ees sur $\rplus$.




\msk
\cl{- - - - - - - - - - - - - - - - - - - - - - - - - - - - - - - }
\msk

{\bf 1.} Posons $\;w(x)=c+\int_0^xu(t)\>v(t)\>{\rm d}t$. On a $u\ie w$ sur $\rplus$. La fonction $w$ est d\'erivable et $w'=uv\ie wv$. Posons maintenant $\;V(x)=\int_0^xv(t)\>{\rm d}t\;$ et $\;f(x)=e^{-V(x)}\>w(x)$. La fonction $f$ est d\'erivable et $\;f'=e^{-V}(w'-wv)\ie0$, donc $f$ est d\'ecroissante sur $\rplus$ et $\;\a x\in\rplus\quad f(x)\ie f(0)=w(0)=c$, donc\vv
$$\a x\in\rplus\qquad w(x)\ie c\cdot e^{V(x)}\qquad\hbox{et, a fortiori}\;,\quad u(x)\ie c\cdot e^{V(x)}\;,$$
ce qu'il fallait prouver.

\msk
{\bf 2.} Soit $\ffi$ une solution de {\bf (E)} sur $\rplus$. Alors $\ffi$ est solution de\vv
$${\bf (E')}\;:\qquad y''+y=p(x)\>\ffi(x)\;.$$
La m\'ethode de variation des constantes permet alors d'\'ecrire\vv
$$\ffi(x)=A\>\cos x+B\>\sin x+\int_0^x\sin(x-t)\>p(t)\>\ffi(t)\>{\rm d}t\;.$$
On a donc $\;|\ffi(x)|\ie c+\int_0^x|p(t)|\>|\ffi(t)|\>{\rm d}t\;$ avec $c=|A|+|B|$. Le lemme de Gronwall donne alors\vv
$$\a x\in\rplus\qquad |\ffi(x)|\ie c\cdot\exp\lp\int_0^x|p(t)|\>{\rm d}t\rp\ie c\cdot\exp\lp\int_0^{\ii}|p(t)|\>{\rm d}t\rp\;,$$
donc la fonction $\ffi$ est born\'ee sur $\rplus$.

\bsk
\hrule
\eject

{\bf EXERCICE 5 :}\msk
Soit le syst\`eme diff\'erentiel \`a coefficients constants\vv
$${\bf (S)}\;:\quad X'=AX\;,\quad{\rm avec}\quad X:\rmat\vers\cmat^n\;,\quad A\in{\cal M}_n(\cmat)\;.$$
\par
{\bf 1.} Condition n\'ecessaire et suffisante sur la matrice $A$ pour que toutes les solutions du syst\`eme {\bf (S)} soient born\'ees sur $\rmat$~?\msk 
{\bf 2.} \`A quelle condition les solutions de {\bf (S)} sont-elles born\'ees sur $\rplus$~?\msk
{\bf 3.} En d\'eduire que toute matrice antisym\'etrique r\'eelle est diagonalisable sur $\cmat$, avec toutes ses valeurs propres imaginaires pures~?\msk
{\bf 4.} \'Etendre au cas d'une matrice complexe antihermitienne ($A^*=-A$).


\msk
\cl{- - - - - - - - - - - - - - - - - - - - - - - - - - - - - -}
\msk

{\bf 1.} Les solutions de {\bf (S)} s'expriment sous la forme $X(t)=e^{tA}\>X_0$, mais le calcul de $e^{tA}$, partant d'une matrice $A$ quelconque, n'est pas ais\'e.
Effectuons alors la r\'eduction de $A$ suivant ses sous-espaces caract\'eristiques~: on peut \'ecrire $A=PBP^{-1}$, avec $P\in{\rm GL}_n(\cmat)$ et $B\in{\cal M}_n(\cmat)$ diagonale par blocs~: $B=\diag(B_1,\cdots,B_m)$, chaque bloc $B_k\in{\cal M}_{\alpha_k}(\cmat)$ \'etant de la forme $B_k=\lam_kI_{\alpha_k}+N_k$, o\`u $N_k\in{\cal M}_{\alpha_k}(\cmat)$ est nilpotente.\msk\sect
Effectuons le changement de fonctions inconnues $X=PY\iff Y=P^{-1}X$. Alors, en posant $Y_0=Y(0)$ (``condition initiale''), on a\vv
$${\bf (S)}\iff Y'=BY\iff Y(t)=e^{tB}\>Y_0\;.$$\sect
Les solutions de {\bf (S)} sont born\'ees sur $\rmat$ si et seulement si les coefficients de la matrice $e^{tB}$ sont born\'es. Or, la matrice $e^{tB}$ est aussi diagonale par blocs~: $e^{tB}=\diag(e^{tB_1},\cdots,e^{tB_m})$, o\`u\vv
$$e^{tB_k}=e^{t\lam_k}\>e^{tN_k}=e^{t\lam_k}\Big(I_{\alpha_k}+t\>N_k+{t^2\s2}\>N_k^2+\cdots+{t^{\beta_k-1}\s(\beta_k-1)!}\>N_k^{\beta_k-1}\Big)\;,$$
$\beta_k$ \'etant l'indice de nilpotence de la matrice $N_k$ ({\it rappelons que $\beta_k$ est l'ordre de la valeur propre $\lam_k$ en tant que racine du polyn\^ome minimal, alors que $\alpha_k$ est sa multiplicit\'e en tant que racine du polyn\^ome caract\'eristique~; le sous-espace caract\'eristique est $\Ker(A-\lam_k I_n)^{\beta_k}$ et il est de dimension $\alpha_k$}).\msk\sect
Les matrices $I_{\alpha_k}$, $N_k$, $N_k^2$, $\cdots$, $N_k^{\beta_k-1}$ \'etant lin\'eairement ind\'ependantes dans ${\cal M}_{\alpha_k}(\cmat)$ ({\it exercice classique}), on voit que la fonction $t\mapsto e^{tB_k}$ est born\'ee sur $\rmat$ si et seulement si on a, d'une part $t\mapsto e^{t\lam_k}$ born\'ee sur $\rmat$ (c'est-\`a-dire $\lam_k\in i\rmat$) et d'autre part $\beta_k=1$ (c'est-\`a-dire $N_k=0$ ou encore $B_k=\lam_kI_{\alpha_k}$).\msk\sect
Une condition n\'ecessaire et suffisante pour que les solutions de {\bf (S)} soient born\'ees sur $\rmat$ est donc que la matrice $A$ soit diagonalisable sur $\cmat$, ses valeurs propres \'etant toutes imaginaires pures.

\msk
{\bf 2.} En reprenant les calculs pr\'ec\'edents, on voit que les solutions de {\bf (S)} sont born\'ees sur $\rplus$ si et seulement si, pour tout $k\in\[ent1,m\]ent$, toutes les fonctions $t\mapsto t^p\>e^{t\lam_k}$ ($0\ie p\ie\beta_k-1$) sont born\'ees sur $\rplus$. Or,\ssk\new
$\triangleright$ si $\Re(\lam_k)>0$, aucune de ces fonctions n'est born\'ee sur $\rplus$~;\ssk\new
$\triangleright$ si $\Re(\lam_k)=0$, seule celle pour $p=0$ est born\'ee sur $\rplus$~;\ssk\new
$\triangleright$ si $\Re(\lam_k)<0$, elles sont toutes born\'ees sur $\rplus$.
\msk\sect
En conclusion, une condition n\'ecessaire et suffisante pour que les solutions de {\bf (S)} soient born\'ees sur $\rplus$ est que toutes les valeurs propres de $A$ aient une partie r\'eelle n\'egative ou nulle, celles de partie r\'eelle nulle \'etant racines simples du polyn\^ome minimal (pour ces derni\`eres, le sous-espace caract\'eristique doit \^etre confondu avec le sous-espace propre).

\msk
{\bf 3.} Si $A$ est antisym\'etrique r\'eelle, si $X$ est solution de $X'=AX$ (\`a valeurs r\'eelles, il suffit pour cela de prendre $X_0=X(0)\in\rmat^n$) , alors, en notant $\|\cdot\|$ la norme euclidienne canonique sur $\rmat^n$, on a\vv
$${{\rm d}\s{\rm d}t}\big(\|X(t)\|^2\big)=2\big(X'(t)|X(t)\big)=2\big(A\>X(t)|X(t)\big)=0$$
($X\mapsto\|X\|^2$ est une {\bf int\'egrale premi\`ere} du syst\`eme diff\'erentiel {\bf (S)}). Les solutions de {\bf (S)} sont constantes en norme, donc born\'ees (on l'a montr\'e pour les solutions \`a valeurs r\'elles, mais il en est de m\^eme des solutions \`a valeurs complexes, puisque si $X:\rmat\vers\cmat^n$ est une telle solution, alors $\Re(X)$ et $\Im(X)$ sont aussi solutions de {\bf (S)}). De la question {\bf 1.}, on d\'eduit que la matrice $A$ est diagonalisable sur $\cmat$, \`a valeurs propres imaginaires pures.

\msk
{\bf 4.} Si $A$ est antihermitienne, si $X:\rmat\vers\cmat^n$, $X=(x_1,\cdots,x_n)$ est une solution de {\bf (S)}, on a, en notant $\|\cdot\|$ la norme hermitienne canonique sur $\cmat^n$, $\|X(t)\|^2=\t\>\ov{X}X$, donc\vv
$${{\rm d}\s{\rm d}t}\big(\|X\|^2\big)=\t\>\ov{X}\>'X+\t\>\ov{X}X'=\t\>\ov{X}\>\t\>\ov{A}X+\t\>\ov{X}AX=\t\>\ov{X}(A^*+A)X=0$$
car $A^*+A=0$, d'o\`u la m\^eme conclusion qu'\`a la question pr\'ec\'edente (qui est d'ailleurs un cas particulier de celle-ci)~: la matrice $A$ est diagonalisable sur $\cmat$, \`a valeurs propres imaginaires pures.









\end{document}