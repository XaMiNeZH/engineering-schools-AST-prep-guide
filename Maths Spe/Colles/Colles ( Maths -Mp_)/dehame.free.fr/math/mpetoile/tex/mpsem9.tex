\documentclass{article}
\begin{document}

\parindent=-8mm\leftskip=8mm
\def\new{\par\hskip 8.3mm}
\def\sect{\par\quad}
\hsize=147mm  \vsize=230mm
\hoffset=-10mm\voffset=0mm

\everymath{\displaystyle}       % �vite le textstyle en mode
                                % math�matique

\font\itbf=cmbxti10

\let\dis=\displaystyle          %raccourci
\let\eps=\varepsilon            %raccourci
\let\vs=\vskip                  %raccourci


\frenchspacing

\let\ie=\leq
\let\se=\geq



\font\pc=cmcsc10 % petites capitales (aussi cmtcsc10)

\def\tp{\raise .2em\hbox{${}^{\hbox{\seveni t}}\!$}}%



\font\info=cmtt10




%%%%%%%%%%%%%%%%% polices grasses math�matiques %%%%%%%%%%%%
\font\tenbi=cmmib10 % bold math italic
\font\sevenbi=cmmi7% scaled 700
\font\fivebi=cmmi5 %scaled 500
\font\tenbsy=cmbsy10 % bold math symbols
\font\sevenbsy=cmsy7% scaled 700
\font\fivebsy=cmsy5% scaled 500
%%%%%%%%%%%%%%% polices de presentation %%%%%%%%%%%%%%%%%
\font\titlefont=cmbx10 at 20.73pt
\font\chapfont=cmbx12
\font\secfont=cmbx12
\font\headfont=cmr7
\font\itheadfont=cmti7% at 6.66pt



% personnel Monasse
\def\euler{\cal}
\def\goth{\cal}
\def\phi{\varphi}
\def\epsilon{\varepsilon}

%%%%%%%%%%%%%%%%%%%%  tableaux de variations %%%%%%%%%%%%%%%%%%%%%%%
% petite macro d'�criture de tableaux de variations
% syntaxe:
%         \variations{t    && ... & ... & .......\cr
%                     f(t) && ... & ... & ...... \cr
%
%etc...........}
% � l'int�rieur de cette macro on peut utiliser les macros
% \croit (la fonction est croissante),
% \decroit (la fonction est d�croissante),
% \nondef (la fonction est non d�finie)
% si l'on termine la derni�re ligne par \cr, un trait est tir� en dessous
% sinon elle est laiss�e sans trait
%%%%%%%%%%%%%%%%%%%%%%%%%%%%%%%%%%%%%%%%%%%%%%%%%%%%%%%%%%%%%%%%%%%

\def\variations#1{\par\medskip\centerline{\vbox{{\offinterlineskip
            \def\decroit{\searrow}
    \def\croit{\nearrow}
    \def\nondef{\parallel}
    \def\tableskip{\omit& height 4pt & \omit \endline}
    % \everycr={\noalign{\hrule}}
            \def\cr{\endline\tableskip\noalign{\hrule}\tableskip}
    \halign{
             \tabskip=.7em plus 1em
             \hfil\strut $##$\hfil &\vrule ##
              && \hfil $##$ \hfil \endline
              #1\crcr
           }
 }}}\medskip}   % MONASSE

%%%%%%%%%%%%%%%%%%%%%%%% NRZCQ %%%%%%%%%%%%%%%%%%%%%%%%%%%%
\def\nmat{{\rm I\kern-0.5mm N}}  % MONASSE
\def\rmat{{\rm I\kern-0.6mm R}}  % MONASSE
\def\cmat{{\rm C\kern-1.7mm\vrule height 6.2pt depth 0pt\enskip}}  % MONASSE
\def\zmat{\mathop{\raise 0.1mm\hbox{\bf Z}}\nolimits}
\def\qmat{{\rm Q\kern-1.8mm\vrule height 6.5pt depth 0pt\enskip}}  % MONASSE
\def\dmat{{\rm I\kern-0.6mm D}}
\def\lmat{{\rm I\kern-0.6mm L}}
\def\kmat{{\rm I\kern-0.7mm K}}

%___________intervalles d'entiers______________
\def\[ent{[\hskip -1.5pt [}
\def\]ent{]\hskip -1.5pt ]}
\def\rent{{\bf ]}\hskip -2pt {\bf ]}}
\def\lent{{\bf [}\hskip -2pt {\bf [}}

%_____def de combinaison
\def\comb{\mathop{\hbox{\large C}}\nolimits}

%%%%%%%%%%%%%%%%%%%%%%% Alg�bre lin�aire %%%%%%%%%%%%%%%%%%%%%
%________image_______
\def\im{\mathop{\rm Im}\nolimits}
%________determinant_______
\def\det{\mathop{\rm det}\nolimits}  % MONASSE
\def\Det{\mathop{\rm Det}\nolimits}
\def\diag{\mathop{\rm diag}\nolimits}
%________rang_______
\def\rg{\mathop{\rm rg}\nolimits}
%________id_______
\def\id{\mathop{\rm id}\nolimits}
\def\tr{\mathop{\rm tr}\nolimits}
\def\Id{\mathop{\rm Id}\nolimits}
\def\Ker{\mathop{\rm Ker}\nolimits}
\def\bary{\mathop{\rm bar}\nolimits}
\def\card{\mathop{\rm card}\nolimits}
\def\Card{\mathop{\rm Card}\nolimits}
\def\grad{\mathop{\rm grad}\nolimits}
\def\Vect{\mathop{\rm Vect}\nolimits}
\def\Log{\mathop{\rm Log}\nolimits}

%________GL_______
\def\GLR#1{{\rm GL}_{#1}(\rmat)}  % MONASSE
\def\GLC#1{{\rm GL}_{#1}(\cmat)}  % MONASSE
\def\GLK#1#2{{\rm GL}_{#1}(#2)}  % MONASSE
\def\SO{\mathop{\rm SO}\nolimits}
\def\SDP#1{{\cal S}_{#1}^{++}}
%________spectre_______
\def\Sp{\mathop{\rm Sp}\nolimits}
%_________ transpos�e ________
%\def\t{\raise .2em\hbox{${}^{\hbox{\seveni t}}\!$}}
\def\t{\,{}^t\!\!}

%_______M gothL_______
\def\MR#1{{\cal M}_{#1}(\rmat)}  % MONASSE
\def\MC#1{{\cal M}_{#1}(\cmat)}  % MONASSE
\def\MK#1{{\cal M}_{#1}(\kmat)}  % MONASSE

%________Complexes_________ % MONASSE
\def\Re{\mathop{\rm Re}\nolimits}
\def\Im{\mathop{\rm Im}\nolimits}

%_______cal L_______
\def\L{{\euler L}}

%%%%%%%%%%%%%%%%%%%%%%%%% fonctions classiques %%%%%%%%%%%%%%%%%%%%%%
%________cotg_______
\def\cotan{\mathop{\rm cotan}\nolimits}
\def\cotg{\mathop{\rm cotg}\nolimits}
\def\tg{\mathop{\rm tg}\nolimits}
%________th_______
\def\tanh{\mathop{\rm th}\nolimits}
\def\th{\mathop{\rm th}\nolimits}
%________sh_______
\def\sinh{\mathop{\rm sh}\nolimits}
\def\sh{\mathop{\rm sh}\nolimits}
%________ch_______
\def\cosh{\mathop{\rm ch}\nolimits}
\def\ch{\mathop{\rm ch}\nolimits}
%________log_______
\def\log{\mathop{\rm log}\nolimits}
\def\sgn{\mathop{\rm sgn}\nolimits}

\def\Arcsin{\mathop{\rm Arcsin}\nolimits}   % CLENET
\def\Arccos{\mathop{\rm Arccos}\nolimits}   % CLENET
\def\Arctan{\mathop{\rm Arctan}\nolimits}   % CLENET
\def\Argsh{\mathop{\rm Argsh}\nolimits}     % CLENET
\def\Argch{\mathop{\rm Argch}\nolimits}     % CLENET
\def\Argth{\mathop{\rm Argth}\nolimits}     % CLENET
\def\Arccotan{\mathop{\rm Arccotan}\nolimits}
\def\coth{\mathop{\rm coth}\nolimits}
\def\Argcoth{\mathop{\rm Argcoth}\nolimits}
\def\E{\mathop{\rm E}\nolimits}
\def\C{\mathop{\rm C}\nolimits}

\def\build#1_#2^#3{\mathrel{\mathop{\kern 0pt#1}\limits_{#2}^{#3}}} %CLENET

%________classe C_________
\def\C{{\cal C}}
%____________suites et s�ries_____________________
\def\suiteN #1#2{(#1 _#2)_{#2\in \nmat }}  % MONASSE
\def\suite #1#2#3{(#1 _#2)_{#2\ge#3 }}  % MONASSE
\def\serieN #1#2{\sum_{#2\in \nmat } #1_#2}  % MONASSE
\def\serie #1#2#3{\sum_{#2\ge #3} #1_#2}  % MONASSE

%___________norme_________________________
\def\norme#1{\|{#1}\|}  % MONASSE
\def\bignorme#1{\left|\hskip-0.9pt\left|{#1}\right|\hskip-0.9pt\right|}

%____________vide (perso)_________________
\def\vide{\hbox{\O }}
%____________partie
\def\P{{\cal P}}

%%%%%%%%%%%%commandes abr�g�es%%%%%%%%%%%%%%%%%%%%%%%
\let\lam=\lambda
\let\ddd=\partial
\def\bsk{\vspace{12pt}\par}
\def\msk{\vspace{6pt}\par}
\def\ssk{\vspace{3pt}\par}
\let\noi=\noindent
\let\eps=\varepsilon
\let\ffi=\varphi
\let\vers=\rightarrow
\let\srev=\leftarrow
\let\impl=\Longrightarrow
\let\tst=\textstyle
\let\dst=\displaystyle
\let\sst=\scriptstyle
\let\ssst=\scriptscriptstyle
\let\divise=\mid
\let\a=\forall
\let\e=\exists
\let\s=\over
\def\vect#1{\overrightarrow{\vphantom{b}#1}}
\let\ov=\overline
\def\eu{\e !}
\def\pn{\par\noi}
\def\pss{\par\ssk}
\def\pms{\par\msk}
\def\pbs{\par\bsk}
\def\pbn{\bsk\noi}
\def\pmn{\msk\noi}
\def\psn{\ssk\noi}
\def\nmsk{\noalign{\msk}}
\def\nssk{\noalign{\ssk}}
\def\equi_#1{\build\sim_#1^{}}
\def\lp{\left(}
\def\rp{\right)}
\def\lc{\left[}
\def\rc{\right]}
\def\lci{\left]}
\def\rci{\right[}
\def\Lim#1#2{\lim_{#1\vers#2}}
\def\Equi#1#2{\equi_{#1\vers#2}}
\def\Vers#1#2{\quad\build\longrightarrow_{#1\vers#2}^{}\quad}
\def\Limg#1#2{\lim_{#1\vers#2\atop#1<#2}}
\def\Limd#1#2{\lim_{#1\vers#2\atop#1>#2}}
\def\lims#1{\Lim{n}{+\infty}#1_n}
\def\cl#1{\par\centerline{#1}}
\def\cls#1{\pss\centerline{#1}}
\def\clm#1{\pms\centerline{#1}}
\def\clb#1{\pbs\centerline{#1}}
\def\cad{\rm c'est-�-dire}
\def\ssi{\it si et seulement si}
\def\lac{\left\{}
\def\rac{\right\}}
\def\ii{+\infty}
\def\eg{\rm par exemple}
\def\vv{\vskip -2mm}
\def\vvv{\vskip -3mm}
\def\vvvv{\vskip -4mm}
\def\union{\;\cup\;}
\def\inter{\;\cap\;}
\def\sur{\above .2pt}
\def\tvi{\vrule height 12pt depth 5pt width 0pt}
\def\tv{\vrule height 8pt depth 5pt width 1pt}
\def\rplus{\rmat_+}
\def\rpe{\rmat_+^*}
\def\rdeux{\rmat^2}
\def\rtrois{\rmat^3}
\def\net{\nmat^*}
\def\ret{\rmat^*}
\def\cet{\cmat^*}
\def\rbar{\ov{\rmat}}
\def\deter#1{\left|\matrix{#1}\right|}
\def\intd{\int\!\!\!\int}
\def\intt{\int\!\!\!\int\!\!\!\int}
\def\ce{{\cal C}}
\def\ceun{{\cal C}^1}
\def\cedeux{{\cal C}^2}
\def\ceinf{{\cal C}^{\infty}}
\def\zz#1{\;{\raise 1mm\hbox{$\zmat$}}\!\!\Bigm/{\raise -2mm\hbox{$\!\!\!\!#1\zmat$}}}
\def\interieur#1{{\buildrel\circ\over #1}}
%%%%%%%%%%%% c'est la fin %%%%%%%%%%%%%%%%%%%%%%%%%%%
\catcode`@=12 % at signs are no longer letters
\catcode`\�=\active
\def�{\'e}
\catcode`\�=\active
\def�{\`e}
\catcode`\�=\active
\def�{\^e}
\catcode`\�=\active
\def�{\`a}
\catcode`\�=\active
\def�{\`u}
\catcode`\�=\active
\def�{\^u}
\catcode`\�=\active
\def�{\^a}
\catcode`\"=\active
\def"{\^o}
\catcode`\�=\active
\def�{\"e}
\catcode`\�=\active
\def�{\"\i}
\catcode`\�=\active
\def�{\"u}
\catcode`\�=\active
\def�{\c c}
\catcode`\�=\active
\def�{\^\i}


\def\boxit#1#2{\setbox1=\hbox{\kern#1{#2}\kern#1}%
\dimen1=\ht1 \advance\dimen1 by #1 \dimen2=\dp1 \advance\dimen2 by #1
\setbox1=\hbox{\vrule height\dimen1 depth\dimen2\box1\vrule}%
\setbox1=\vbox{\hrule\box1\hrule}%
\advance\dimen1 by .4pt \ht1=\dimen1
\advance\dimen2 by .4pt \dp1=\dimen2 \box1\relax}


\catcode`\@=11
\def\system#1{\left\{\null\,\vcenter{\openup1\jot\m@th
\ialign{\strut\hfil$##$&$##$\hfil&&\enspace$##$\enspace&
\hfil$##$&$##$\hfil\crcr#1\crcr}}\right.}
\catcode`\@=12
\pagestyle{empty}





\overfullrule=0mm


\cl{{\bf SEMAINE 9}}\msk
\cl{{\bf S\'ERIES NUM\'ERIQUES}}
\bsk

{\bf EXERCICE 1 :}\msk
Soit $\gamma$ la {\bf constante d'Euler}~: $\gamma=\Lim{n}{\infty}\lp\sum_{k=1}^n{1\s k}-\ln n\rp$. D\'emontrer l'\'egalit\'e
$$\sum_{n=1}^{\infty}(-1)^n{\ln n\s n}=\gamma\cdot\ln 2-{1\s2}(\ln 2)^2\;.$$

\msk
\cl{- - - - - - - - - - - - - - - - - - - - - - - - - - - - - -}
\msk

La s\'erie de terme g\'en\'eral $(-1)^n{\ln n\s n}$ est convergente car $\Lim{n}{\infty}{\ln n\s n}=0$ et la suite $\lp{\ln n\s n}\rp$ est d\'ecroissante... \`a partir du rang 3, notons $\;s_n=\sum_{k=1}^n(-1)^k{\ln k\s k}\;$ sa somme partielle d'ordre $n$. On peut faire appara\^\i tre les sommes partielles de la s\'erie harmonique en d\'ecomposant $s_{2n}$ de la fa\c con suivante~:\vvvv
\begin{eqnarray*}
s_{2n} & = & \sum_{k=1}^{2n}(-1)^k{\ln k\s k}=-\sum_{k=1}^{2n}{\ln k\s k}+2\>\sum_{k=1}^n{\ln(2k)\s2k}\\
& = & -\sum_{k=1}^{2n}{\ln k\s k}+\lp\sum_{k=1}^n{1\s k}\rp\cdot\ln 2+\sum_{k=1}^n{\ln k\s k}\;.
\end{eqnarray*}
En posant $\;H_n=\sum_{k=1}^n{1\s k}\;$ et $\;S_n=\sum_{k=2}^n{\ln k\s k}$, on a donc
$$s_{2n}=H_n\>\ln 2-\sum_{k=n+1}^{2n}{\ln k\s k}=H_n\>\ln 2+S_n-S_{2n}\;.$$ Le d\'eveloppement asymptotique $\;H_n=\ln n+\gamma+o(1)\;$ est connu ({\it isn't it~?}), l'exercice sera termin\'e si l'on trouve un d\'eveloppement asymptotique de $S_n$ \`a la pr\'ecision $o(1)$.\msk
Par comparaison avec une int\'egrale, on obtient d\'ej\`a $\;S_n\sim{1\s2}(\ln n)^2$, mais cela ne suffit pas. Cherchons donc \`a estimer $\;S_n-{1\s2}(\ln n)^2$. Pour cela, \'ecrivons $\;S_n-{1\s2}(\ln n)^2=\sum_{k=2}^na_k$, avec\vv
\begin{eqnarray*}
a_k & = & {\ln k\s k}-{1\s 2}\big[(\ln k)^2-(\ln(k-1))^2\big]={\ln k\s k}+{1\s2}\ln\lp1-{1\s k}\rp\cdot\big(\ln k+\ln(k-1)\big)\\
& = & {\ln k\s k}+{1\s 2}\lp-{1\s k}-{1\s 2k^2}+o\Big({1\s k^2}\Big)\rp\lp 2\ln k+O\Big({1\s k}\Big)\rp\\
& = & -{\ln k\s 2k^2}+o\lp{\ln k\s k^2}\rp\;.
\end{eqnarray*}
De $a_k\sim-{\ln k\s 2k^2}$, on d\'eduit que la s\'erie de terme g\'en\'eral $a_k$ est convergente donc, en posant $\;l=\sum_{k=2}^{\ii}a_k$, on a le d\'eveloppement asymptotique $\;S_n={1\s2}(\ln n)^2+l+o(1)$, puis\vv
\begin{eqnarray*}
s_{2n} & = & (\ln n+\gamma)(\ln 2)+{1\s2}\big((\ln n)^2-(\ln(2n)^2)\big)+l-l+o(1)\\
          & = & (\ln n+\gamma)(\ln 2)+{1\s2}\big((\ln n)^2-(\ln(n)^2+(\ln 2)^2+2\ln 2\cdot\ln n)\big)+o(1)\\
          & = & \gamma\cdot\ln 2-{1\s2}(\ln 2)^2+o(1)\;.
\end{eqnarray*}
Comme $\;\sum_{n=1}^{\ii}(-1)^n\>{\ln n\s n}=\Lim{n}{\infty}s_{2n}$, on obtient le r\'esultat demand\'e.

\bsk
\hrule
\bsk

{\bf EXERCICE 2 :}\msk
{\bf 1.} Soient $(A_n)$ et $(B_n)$ deux suites complexes de limites $A$ et $B$ respectivement. Montrer que\vv
$$\Lim{n}{\infty}{1\s n+1}\>\sum_{k=0}^nA_kB_{n-k}=AB\;.$$\par
{\bf 2.} Soient $\sum_na_n$ et $\sum_nb_n$ deux s\'eries de nombres complexes, on note $\sum_nc_n$ leur produit de Cauchy~: $c_n=\sum_{k=0}^na_kb_{n-k}$. Montrer que, si les trois s\'eries $\sum_n a_n$, $\sum_n b_n$ et $\sum_n c_n$ sont convergentes, alors on a la relation\vv
$$\sum_{n=0}^{\infty}c_n=\Big(\sum_{n=0}^{\infty}a_n\Big)\>\Big(\sum_{n=0}^{\infty}b_n\Big)\;.$$

\bsk
\cl{- - - - - - - - - - - - - - - - - - - - - - - - - - - - - -}
\bsk

{\bf 1.} Posons $C_n={1\s n+1}\>\sum_{k=0}^nA_kB_{n-k}$, alors
$$C_n-AB={1\s n+1}\sum_{k=0}^n(A_kB_{n-k}-AB)={1\s n+1}\sum_{k=0}^n\big[(A_k-A)B_{n-k}+(B_{n-k}-B)A\Big]\;,$$
donc\vv
$$|C_n-AB|\ie|B_{n-k}|\>\Big({1\s n+1}\sum_{k=0}^n|A_k-A|\Big)+|A|\>\Big({1\s n+1}\sum_{k=0}^n|B_{n-k}-B|\Big)\;.$$
Or, d'apr\`es Cesaro, ${1\s n+1}\sum_{k=0}^n|A_k-A|\;$ et $\;{1\s n+1}\sum_{k=0}^n|B_{n-k}-B|={1\s n+1}\sum_{k=0}^n|B_k-B|$\break tendent vers z\'ero et $|B_{n-k}|$ est major\'e puisque la suite $(B_n)$ est convergente, donc\break $\Lim{n}{\infty}(C_n-AB)=0$.

\msk
{\bf 2.} Pour tout $n$, posons $A_n=\sum_{k=0}^na_k$, $B_n=\sum_{k=0}^n b_k$, $C_n=\sum_{k=0}^nc_k\;$ et enfin, pour tout $N\in\nmat$, posons $\;\Gamma_N=\sum_{n=0}^NC_n$. On a alors
$$\Gamma_N=\sum_{n=0}^N\Big(\sum_{k=0}^nc_k\Big)=(N+1)c_0+Nc_1+\cdots+2c_{N-1}+c_N=\sum_{n=0}^N(N+1-n)c_n\;.$$
On remarque que c'est aussi $\;\sum_{k=0}^NA_kB_{N-k}$, en effet~:\vv
\begin{eqnarray*}
\sum_{k=0}^NA_kB_{N-k} & = & a_0(b_0+\cdots+b_{N-1}+b_N)+(a_0+a_1)(b_0+\cdots+b_ {N-1})+\cdots\cdots+(a_0+a_1+\cdots+a_N)b_0\\
& = & (N+1) a_0b_0+N(a_0b_1+a_1b_0)+(N-1)(a_0b_2+a_1b_1+a_2b_0)+\cdots\cdots+(a_0b_N+\cdots+a_Nb_0)\\
& = & \sum_{n=0}^N(N+1-n)c_n=\Gamma_N\;.
\end{eqnarray*}
Posons enfin $A=\sum_{n=0}^{\infty}a_n$, $B=\sum_{n=0}^{\infty}b_n$, $C=\sum_{n=0}^{\infty}c_n$. Comme $\Lim{n}{\infty}C_n=C$, du th\'eor\`eme de Cesaro, on d\'eduit que $\Lim{n}{\infty}{\Gamma_n\s n+1}=C$, c'est-\`a-dire $\Lim{n}{\infty}{1\s n+1}\sum_{k=0}^nA_kB_{n-k}=C\;$ mais, d'apr\`es la question {\bf 1.}, cette derni\`ere expression tend aussi vers $AB$, donc $C=AB$.


\bsk
\hrule
\bsk

{\bf EXERCICE 3 :}\msk
Convergence et calcul de $\;\sum_{n=2}^{\ii}{(-1)^n\s n}\>E\big(\log_2(n)\big)$.

\bsk
\cl{- - - - - - - - - - - - - - - - - - - - - - - - - - - - - -}
\bsk

Posons $u_n={(-1)^n\s n}\>E\big(\log_2(n)\big)\;$ pour $n\se2$.\msk
$\bullet$ Effectuons une sommation
par paquets en regroupant les entiers $n$ pour lesquels l'expression
$\;E\big(\log_2(n)\big)\;$ garde une valeur constante~:\msk\new
pour tout $k\in\net$ donn\'e, on a $\;E\big(\log_2(n)\big)=k\iff 2^k\ie n<2^{k+1}$.
\msk\sect
Posons alors $A_k=\sum_{n=2^k}^{2^{k+1}-1}u_n=k\cdot\sum_{n=2^k}^{2^{k+1}-1}{(-1)^n\s n}$ pour $k\in\net$ et
montrons la convergence de la s\'erie de terme g\'en\'eral $A_k$.\ssk\sect
Pour cela, introduisons encore quelques notations~:\ssk\sect
- pour $n\in\net$, soit $H_n=\sum_{k=1}^n{1\s k}$ (somme partielle de la s\'erie harmonique)~:\ssk\sect
- pour $n\in\net$, soit $J_n=\sum_{k=1}^n{(-1)^k\s k}$ (somme partielle de la s\'erie harmonique altern\'ee)~:\ssk\sect
- pour $k\in\net$, soit $S_k=H_{2^k-1}=\sum_{p=1}^{2^k-1}{1\s p}$~;\ssk\sect
- pour $k\in\net$, soit $T_k=J_{2^k-1}=\sum_{p=1}^{2^k-1}{(-1)^p\s p}$~;\ssk\sect
On a alors facilement $\;S_k+T_k=S_{k-1}\;$ pour tout $k\se1$ (on convient $S_0=0$),
donc\break $T_k=S_{k-1}-S_k$, puis\vvvv
$$A_k=k(T_{k+1}-T_k)=-k(S_{k+1}-2S_k+S_{k-1})\;.$$\sect
Simplifions les sommes partielles~: pour tout $m\in\net$,
\begin{eqnarray*}
\sum_{k=1}^mA_k & = & -\sum_{k=1}^mk(S_{k+1}-2S_k+S_{k-1})\\
                           & = & -\sum_{k=2}^{m+1}(k-1)S_k+2\sum_{k=1}^mkS_k-\sum_{k=0}^{m-1}(k+1)S_k\\
                           & = & \sum_{k=2}^{m-1}\big[2k-(k-1)-(k+1)\big]S_k
                                   -S_0-2S_1+2S_1+2mS_m-(m-1)S_m-mS_{m+1}\\
                           & = & (m+1)S_m-mS_{m+1}\;.
\end{eqnarray*}
Du d\'eveloppement asymptotique classique~: $H_n=\ln n+\gamma+O\lp
{1\s n}\rp$, o\`u $\gamma$ est la constante d'Euler, on tire\vv
$$S_m=\ln(2^m-1)+\gamma+O\lp{1\s2^m-1}\rp=m\>\ln2+\gamma+O\lp{1\s2^m}\rp\;,$$
puis\vv
\begin{eqnarray*}
\sum_{k=1}^mA_k & = & (m+1)\>\bigg[m\>\ln 2+\gamma+O\lp{1\s2^m}\rp\bigg]-
                                       m\>\bigg[(m+1)\>\ln2+\gamma+O\lp{1\s2^{m+1}}\rp\bigg]\\
                           & = & \gamma+O\lp{m\s2^m}\rp=\gamma+o(1)\;.
\end{eqnarray*}
La s\'erie de terme g�n�ral $A_k$ converge donc et $\;\sum_{k=1}^{\ii}A_k=\gamma$.\msk
$\bullet$ La s\'erie $\sum_nu_n$ n'\'etant pas absolument convergente, on ne peut pas affirmer
directement que\vv
$$\sum_{n=2}^{\ii}u_n=\sum_{k=1}^{\ii}\lp\sum_{n=2^k}^{2^{k+1}-1}u_n\rp=\sum_{k=1}^{\ii}A_k$$
(la convergence de la s\'erie $\sum_nu_n$ n'\'etant d'ailleurs pas encore prouv\'ee).\ssk\sect
Majorons pour cela les sommes partielles dans les paquets~: si $n\se2$ est
tel que $2^m\ie n<2^{m+1}$ avec $m\in\net$, alors\vv
$$\left|\sum_{i=2^m}^nu_i\right|=m\>\left|\sum_{i=2^m}^n{(-1)^i\s i}\right|\ie
  {m\s2^m}$$
(majoration classique d'une somme partielle d'une s\'erie altern\'ee par la valeur absolue
de son premier terme). Donc (toujours avec $2^m\ie n<2^{m+1}$), on a
$\left|\sum_{i=2}^nu_i-\sum_{k=1}^{m-1}A_k\right|\ie{m\s2^m}$.\break Comme
$\Lim{m}{\ii}{m\s2^m}=0$, on en d\'eduit la convergence de la s\'erie $\sum_nu_n$
et le r\'esultat\break $\sum_{n=2}^{\ii}u_n=\gamma\;$~: si on se donne $\eps>0$, il
existe un entier $M$ tel que, pour tout $m\se M$, on ait
${m\s2^m}\ie{\eps\s2}$ et $\;\left|\sum_{k=1}^{m-1}A_k-\gamma\right|\ie{\eps\s2}$~;
pour tout entier $n$ tel que $n\se2^M$, on a alors
$\;\left|\sum_{i=2}^nu_i-\gamma\right|\ie\eps$.\msk
$\bullet$ Conclusion~: $\;\sum_{n=2}^{\ii}{(-1)^n\s n}\>E\big(\log_2(n)\big)=\gamma$.

\bsk
\hrule
\bsk

{\bf EXERCICE 4 :}\msk
Soit $(u_n)$ une suite r\'eelle qui converge vers z\'ero.\msk
Montrer qu'il existe une suite $(\eps_n)$, \`a valeurs dans$\{-1,1\}$, telle que la s\'erie $\sum_n\eps_nu_n$ soit convergente.

\msk
\cl{- - - - - - - - - - - - - - - - - - - - - - - - - - - - - - -}
\msk

Si $\sum_n u_n$ est convergente, alors c'est gagn\'e avec $\eps_n=1$.\msk
Supposons $\sum_{n}u_n$ divergente, donc a fortiori $\sum_nv_n$ est divergente avec $v_n=|u_n|$. Essayons de construire par r\'ecurrence une suite $(\alpha_n)$, \`a valeurs dans $\{-1,1\}$, de fa\c con que les sommes partielles $s_n=\sum_{k=0}^n\alpha_kv_k$ aient une limite nulle, on aura ainsi $\sum_{n=0}^{\infty}\eps_nu_n=0$ avec, pour tout $n$, $\eps_n=\sgn(u_n)\alpha_n$. L'id\'ee pour cela est de toujours ``revenir'' vers z\'ero, c'est-\`a-dire ajouter le terme n\'egatif $\;-v_n\;$ \`a chaque fois que $s_n$ est positif, et le terme positif $\;v_n\;$ \`a chaque fois que $s_n$ est n\'egatif. Allez, on r\'edige~:
\msk
Posons d'abord $\alpha_0=1$, ainsi $s_0=v_0=|u_0|\se0$, ce qui am\`ene \`a poser $\alpha_1=-1$ et ainsi $s_1=v_0-v_1$, on posera ensuite $\alpha_2=+1$ si $s_1<0$ et $\alpha_2=-1$ si $s_1\se0$. Pour $n\in\net$ donn\'e, supposons $\alpha_0$, $\cdots$, $\alpha_n$ construits (\'el\'ements de $\{-1,1\}$), posons $s_n=\sum_{k=0}^n\alpha_kv_k$, puis\break $\alpha_{n+1}=\system{&+1\;&{\rm si}&\;s_n<0\cr &-1\;&{\rm si}&\;s_n\se0\cr}$. 
Remarquons que, les r\'eels $s_n$ et $\alpha_{n+1}v_{n+1}$ \'etant de signes\break contraires, on a $\;|s_{n+1}|=|s_n+\alpha_{n+1}v_{n+1}|\ie\max\{|s_n|,v_{n+1}\}$.
Montrons maintenant que $\Lim{n}{\infty}s_n=0$.\msk
Soit $\eps>0$. Soit $N$ un entier tel que $n\se N\impl0\ie v_n\ie\eps$. Alors,\ssk\sect
{\bf (i)} : si $|s_N|\ie\eps$, par une r\'ecurrence imm\'ediate, on a $|s_n|\ie\eps$ pour tout $n\se N$ et c'est gagn\'e~;\ssk\sect
{\bf (ii)} : si $s_N>\eps$, la s\'erie $\sum_kv_k$ \'etant divergente, il existe un entier $p$ tel que $\sum_{k=N+1}^{N+p}v_k>s_N-\eps$. Pour le plus petit de ces entiers $p$, on aura plus pr\'ecis\'ement $\;s_N-\eps<\sum_{k=N+1}^{N+p}v_k\ie s_N$,\break ce qui am\`ene \`a poser $\alpha_{N+1}=\cdots=\alpha_{N+p}=-1$ et ainsi $\;0\ie s_{N+p}=s_N-\sum_{k=N+1}^{N+p}v_k\ie\eps$, ce qui nous ram\`ene au cas {\bf (i)}~;\ssk\sect
{\bf (iii)} : si $s_N<-\eps$, raisonnement analogue \`a {\bf (ii)}.

\bsk
{\it On a ainsi prouv\'e que, si $(u_n)$ est une suite de limite nulle, mais telle que la s\'erie de terme g\'en\'eral $u_n$ ne soit pas absolument convergente, on peut trouver une suite de coefficients $(\eps_n)$ dans $\{-1,1\}$ telle que $\sum_{n=0}^{\infty}\eps_nu_n=0$. Il est alors imm\'ediat que, pour tout r\'eel $a$ donn\'e, on peut aussi trouver une suite $(\eps_n)\in\{-1,1\}^{\nmat}$ telle que $\sum_{n=0}^{\infty}\eps_nu_n=a$.}



















\end{document}