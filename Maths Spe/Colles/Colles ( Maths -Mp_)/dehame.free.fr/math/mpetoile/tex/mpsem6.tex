\documentclass{article}
\begin{document}

\parindent=-8mm\leftskip=8mm
\def\new{\par\hskip 8.3mm}
\def\sect{\par\quad}
\hsize=147mm  \vsize=230mm
\hoffset=-10mm\voffset=0mm

\everymath{\displaystyle}       % �vite le textstyle en mode
                                % math�matique

\font\itbf=cmbxti10

\let\dis=\displaystyle          %raccourci
\let\eps=\varepsilon            %raccourci
\let\vs=\vskip                  %raccourci


\frenchspacing

\let\ie=\leq
\let\se=\geq



\font\pc=cmcsc10 % petites capitales (aussi cmtcsc10)

\def\tp{\raise .2em\hbox{${}^{\hbox{\seveni t}}\!$}}%



\font\info=cmtt10




%%%%%%%%%%%%%%%%% polices grasses math�matiques %%%%%%%%%%%%
\font\tenbi=cmmib10 % bold math italic
\font\sevenbi=cmmi7% scaled 700
\font\fivebi=cmmi5 %scaled 500
\font\tenbsy=cmbsy10 % bold math symbols
\font\sevenbsy=cmsy7% scaled 700
\font\fivebsy=cmsy5% scaled 500
%%%%%%%%%%%%%%% polices de presentation %%%%%%%%%%%%%%%%%
\font\titlefont=cmbx10 at 20.73pt
\font\chapfont=cmbx12
\font\secfont=cmbx12
\font\headfont=cmr7
\font\itheadfont=cmti7% at 6.66pt



% personnel Monasse
\def\euler{\cal}
\def\goth{\cal}
\def\phi{\varphi}
\def\epsilon{\varepsilon}

%%%%%%%%%%%%%%%%%%%%  tableaux de variations %%%%%%%%%%%%%%%%%%%%%%%
% petite macro d'�criture de tableaux de variations
% syntaxe:
%         \variations{t    && ... & ... & .......\cr
%                     f(t) && ... & ... & ...... \cr
%
%etc...........}
% � l'int�rieur de cette macro on peut utiliser les macros
% \croit (la fonction est croissante),
% \decroit (la fonction est d�croissante),
% \nondef (la fonction est non d�finie)
% si l'on termine la derni�re ligne par \cr, un trait est tir� en dessous
% sinon elle est laiss�e sans trait
%%%%%%%%%%%%%%%%%%%%%%%%%%%%%%%%%%%%%%%%%%%%%%%%%%%%%%%%%%%%%%%%%%%

\def\variations#1{\par\medskip\centerline{\vbox{{\offinterlineskip
            \def\decroit{\searrow}
    \def\croit{\nearrow}
    \def\nondef{\parallel}
    \def\tableskip{\omit& height 4pt & \omit \endline}
    % \everycr={\noalign{\hrule}}
            \def\cr{\endline\tableskip\noalign{\hrule}\tableskip}
    \halign{
             \tabskip=.7em plus 1em
             \hfil\strut $##$\hfil &\vrule ##
              && \hfil $##$ \hfil \endline
              #1\crcr
           }
 }}}\medskip}   % MONASSE

%%%%%%%%%%%%%%%%%%%%%%%% NRZCQ %%%%%%%%%%%%%%%%%%%%%%%%%%%%
\def\nmat{{\rm I\kern-0.5mm N}}  % MONASSE
\def\rmat{{\rm I\kern-0.6mm R}}  % MONASSE
\def\cmat{{\rm C\kern-1.7mm\vrule height 6.2pt depth 0pt\enskip}}  % MONASSE
\def\zmat{\mathop{\raise 0.1mm\hbox{\bf Z}}\nolimits}
\def\qmat{{\rm Q\kern-1.8mm\vrule height 6.5pt depth 0pt\enskip}}  % MONASSE
\def\dmat{{\rm I\kern-0.6mm D}}
\def\lmat{{\rm I\kern-0.6mm L}}
\def\kmat{{\rm I\kern-0.7mm K}}

%___________intervalles d'entiers______________
\def\[ent{[\hskip -1.5pt [}
\def\]ent{]\hskip -1.5pt ]}
\def\rent{{\bf ]}\hskip -2pt {\bf ]}}
\def\lent{{\bf [}\hskip -2pt {\bf [}}

%_____def de combinaison
\def\comb{\mathop{\hbox{\large C}}\nolimits}

%%%%%%%%%%%%%%%%%%%%%%% Alg�bre lin�aire %%%%%%%%%%%%%%%%%%%%%
%________image_______
\def\im{\mathop{\rm Im}\nolimits}
%________determinant_______
\def\det{\mathop{\rm det}\nolimits}  % MONASSE
\def\Det{\mathop{\rm Det}\nolimits}
\def\diag{\mathop{\rm diag}\nolimits}
%________rang_______
\def\rg{\mathop{\rm rg}\nolimits}
%________id_______
\def\id{\mathop{\rm id}\nolimits}
\def\tr{\mathop{\rm tr}\nolimits}
\def\Id{\mathop{\rm Id}\nolimits}
\def\Ker{\mathop{\rm Ker}\nolimits}
\def\bary{\mathop{\rm bar}\nolimits}
\def\card{\mathop{\rm card}\nolimits}
\def\Card{\mathop{\rm Card}\nolimits}
\def\grad{\mathop{\rm grad}\nolimits}
\def\Vect{\mathop{\rm Vect}\nolimits}
\def\Log{\mathop{\rm Log}\nolimits}

%________GL_______
\def\GLR#1{{\rm GL}_{#1}(\rmat)}  % MONASSE
\def\GLC#1{{\rm GL}_{#1}(\cmat)}  % MONASSE
\def\GLK#1#2{{\rm GL}_{#1}(#2)}  % MONASSE
\def\SO{\mathop{\rm SO}\nolimits}
\def\SDP#1{{\cal S}_{#1}^{++}}
%________spectre_______
\def\Sp{\mathop{\rm Sp}\nolimits}
%_________ transpos�e ________
%\def\t{\raise .2em\hbox{${}^{\hbox{\seveni t}}\!$}}
\def\t{\,{}^t\!\!}

%_______M gothL_______
\def\MR#1{{\cal M}_{#1}(\rmat)}  % MONASSE
\def\MC#1{{\cal M}_{#1}(\cmat)}  % MONASSE
\def\MK#1{{\cal M}_{#1}(\kmat)}  % MONASSE

%________Complexes_________ % MONASSE
\def\Re{\mathop{\rm Re}\nolimits}
\def\Im{\mathop{\rm Im}\nolimits}

%_______cal L_______
\def\L{{\euler L}}

%%%%%%%%%%%%%%%%%%%%%%%%% fonctions classiques %%%%%%%%%%%%%%%%%%%%%%
%________cotg_______
\def\cotan{\mathop{\rm cotan}\nolimits}
\def\cotg{\mathop{\rm cotg}\nolimits}
\def\tg{\mathop{\rm tg}\nolimits}
%________th_______
\def\tanh{\mathop{\rm th}\nolimits}
\def\th{\mathop{\rm th}\nolimits}
%________sh_______
\def\sinh{\mathop{\rm sh}\nolimits}
\def\sh{\mathop{\rm sh}\nolimits}
%________ch_______
\def\cosh{\mathop{\rm ch}\nolimits}
\def\ch{\mathop{\rm ch}\nolimits}
%________log_______
\def\log{\mathop{\rm log}\nolimits}
\def\sgn{\mathop{\rm sgn}\nolimits}

\def\Arcsin{\mathop{\rm Arcsin}\nolimits}   % CLENET
\def\Arccos{\mathop{\rm Arccos}\nolimits}   % CLENET
\def\Arctan{\mathop{\rm Arctan}\nolimits}   % CLENET
\def\Argsh{\mathop{\rm Argsh}\nolimits}     % CLENET
\def\Argch{\mathop{\rm Argch}\nolimits}     % CLENET
\def\Argth{\mathop{\rm Argth}\nolimits}     % CLENET
\def\Arccotan{\mathop{\rm Arccotan}\nolimits}
\def\coth{\mathop{\rm coth}\nolimits}
\def\Argcoth{\mathop{\rm Argcoth}\nolimits}
\def\E{\mathop{\rm E}\nolimits}
\def\C{\mathop{\rm C}\nolimits}

\def\build#1_#2^#3{\mathrel{\mathop{\kern 0pt#1}\limits_{#2}^{#3}}} %CLENET

%________classe C_________
\def\C{{\cal C}}
%____________suites et s�ries_____________________
\def\suiteN #1#2{(#1 _#2)_{#2\in \nmat }}  % MONASSE
\def\suite #1#2#3{(#1 _#2)_{#2\ge#3 }}  % MONASSE
\def\serieN #1#2{\sum_{#2\in \nmat } #1_#2}  % MONASSE
\def\serie #1#2#3{\sum_{#2\ge #3} #1_#2}  % MONASSE

%___________norme_________________________
\def\norme#1{\|{#1}\|}  % MONASSE
\def\bignorme#1{\left|\hskip-0.9pt\left|{#1}\right|\hskip-0.9pt\right|}

%____________vide (perso)_________________
\def\vide{\hbox{\O }}
%____________partie
\def\P{{\cal P}}

%%%%%%%%%%%%commandes abr�g�es%%%%%%%%%%%%%%%%%%%%%%%
\let\lam=\lambda
\let\ddd=\partial
\def\bsk{\vspace{12pt}\par}
\def\msk{\vspace{6pt}\par}
\def\ssk{\vspace{3pt}\par}
\let\noi=\noindent
\let\eps=\varepsilon
\let\ffi=\varphi
\let\vers=\rightarrow
\let\srev=\leftarrow
\let\impl=\Longrightarrow
\let\tst=\textstyle
\let\dst=\displaystyle
\let\sst=\scriptstyle
\let\ssst=\scriptscriptstyle
\let\divise=\mid
\let\a=\forall
\let\e=\exists
\let\s=\over
\def\vect#1{\overrightarrow{\vphantom{b}#1}}
\let\ov=\overline
\def\eu{\e !}
\def\pn{\par\noi}
\def\pss{\par\ssk}
\def\pms{\par\msk}
\def\pbs{\par\bsk}
\def\pbn{\bsk\noi}
\def\pmn{\msk\noi}
\def\psn{\ssk\noi}
\def\nmsk{\noalign{\msk}}
\def\nssk{\noalign{\ssk}}
\def\equi_#1{\build\sim_#1^{}}
\def\lp{\left(}
\def\rp{\right)}
\def\lc{\left[}
\def\rc{\right]}
\def\lci{\left]}
\def\rci{\right[}
\def\Lim#1#2{\lim_{#1\vers#2}}
\def\Equi#1#2{\equi_{#1\vers#2}}
\def\Vers#1#2{\quad\build\longrightarrow_{#1\vers#2}^{}\quad}
\def\Limg#1#2{\lim_{#1\vers#2\atop#1<#2}}
\def\Limd#1#2{\lim_{#1\vers#2\atop#1>#2}}
\def\lims#1{\Lim{n}{+\infty}#1_n}
\def\cl#1{\par\centerline{#1}}
\def\cls#1{\pss\centerline{#1}}
\def\clm#1{\pms\centerline{#1}}
\def\clb#1{\pbs\centerline{#1}}
\def\cad{\rm c'est-�-dire}
\def\ssi{\it si et seulement si}
\def\lac{\left\{}
\def\rac{\right\}}
\def\ii{+\infty}
\def\eg{\rm par exemple}
\def\vv{\vskip -2mm}
\def\vvv{\vskip -3mm}
\def\vvvv{\vskip -4mm}
\def\union{\;\cup\;}
\def\inter{\;\cap\;}
\def\sur{\above .2pt}
\def\tvi{\vrule height 12pt depth 5pt width 0pt}
\def\tv{\vrule height 8pt depth 5pt width 1pt}
\def\rplus{\rmat_+}
\def\rpe{\rmat_+^*}
\def\rdeux{\rmat^2}
\def\rtrois{\rmat^3}
\def\net{\nmat^*}
\def\ret{\rmat^*}
\def\cet{\cmat^*}
\def\rbar{\ov{\rmat}}
\def\deter#1{\left|\matrix{#1}\right|}
\def\intd{\int\!\!\!\int}
\def\intt{\int\!\!\!\int\!\!\!\int}
\def\ce{{\cal C}}
\def\ceun{{\cal C}^1}
\def\cedeux{{\cal C}^2}
\def\ceinf{{\cal C}^{\infty}}
\def\zz#1{\;{\raise 1mm\hbox{$\zmat$}}\!\!\Bigm/{\raise -2mm\hbox{$\!\!\!\!#1\zmat$}}}
\def\interieur#1{{\buildrel\circ\over #1}}
%%%%%%%%%%%% c'est la fin %%%%%%%%%%%%%%%%%%%%%%%%%%%
\catcode`@=12 % at signs are no longer letters
\catcode`\�=\active
\def�{\'e}
\catcode`\�=\active
\def�{\`e}
\catcode`\�=\active
\def�{\^e}
\catcode`\�=\active
\def�{\`a}
\catcode`\�=\active
\def�{\`u}
\catcode`\�=\active
\def�{\^u}
\catcode`\�=\active
\def�{\^a}
\catcode`\"=\active
\def"{\^o}
\catcode`\�=\active
\def�{\"e}
\catcode`\�=\active
\def�{\"\i}
\catcode`\�=\active
\def�{\"u}
\catcode`\�=\active
\def�{\c c}
\catcode`\�=\active
\def�{\^\i}


\def\boxit#1#2{\setbox1=\hbox{\kern#1{#2}\kern#1}%
\dimen1=\ht1 \advance\dimen1 by #1 \dimen2=\dp1 \advance\dimen2 by #1
\setbox1=\hbox{\vrule height\dimen1 depth\dimen2\box1\vrule}%
\setbox1=\vbox{\hrule\box1\hrule}%
\advance\dimen1 by .4pt \ht1=\dimen1
\advance\dimen2 by .4pt \dp1=\dimen2 \box1\relax}


\catcode`\@=11
\def\system#1{\left\{\null\,\vcenter{\openup1\jot\m@th
\ialign{\strut\hfil$##$&$##$\hfil&&\enspace$##$\enspace&
\hfil$##$&$##$\hfil\crcr#1\crcr}}\right.}
\catcode`\@=12
\pagestyle{empty}





\overfullrule=0mm
\cl{{\bf SEMAINE 6}}\msk
\cl{{\bf TOPOLOGIE DES ESPACES M\'ETRIQUES}}
\bsk

{\bf EXERCICE 1 :}\msk
Soit $(K,d)$ un espace m\'etrique compact. Soit $f:K\vers K$ une application
telle que\vv
$$\a(x,y)\in K^2\qquad d\big(f(x),f(y)\big)\se d(x,y)\;.$$\par
Montrer que $f$ est une isom\'etrie de $K$ (bijection de $K$ sur $K$
conservant la distance).

\msk

{\it Source : CHAMBERT-LOIR, FERMIGIER, MAILLOT : Exercices de math\'ematiques
pour\break l'agr\'egation, Analyse 1, \'Editions Masson}, ISBN : 2-225-84692-8.

\msk
\cl{- - - - - - - - - - - - - - - - - - - - - - - - - - - - - -}
\msk

$\bullet$ L'injectivit\'e de $f$ est imm\'ediate.\msk
$\bullet$ L'application $f$ conserve la distance~: soient $a$ et $b$ deux
points de $K$, notons $a_n=f^n(a)$, $b_n=f^n(b)$ leurs it\'er\'es par $f$
($n\in\nmat$). Comme $K$ est compact, il existe une application $\ffi:\nmat
\vers\nmat$, strictement croissante, telle que les suites extraites
$(a_{\ffi(n)})$ et $(b_{\ffi(n)})$ soient convergentes, de limites
$\alpha$ et $\beta$ respectivement.\ssk\sect
Donnons-nous $\eps>0$. Alors il existe un entier $k$ strictement positif tel
que $d(a,a_k)<{\eps\s2}$ et $d(b,b_k)<{\eps\s2}$~: en effet, il existe des
entiers $N_1$ et $N_2$ (avec $N_1<N_2$) tels que\vv
$$d(a_{\ffi(N_1)},\alpha)<{\eps\s4}\;;\quad
  d(a_{\ffi(N_2)},\alpha)<{\eps\s4}\;;\quad
  d(b_{\ffi(N_1)},\beta)<{\eps\s4}\;;\quad
  d(b_{\ffi(N_2)},\beta)<{\eps\s4}\;.$$
En posant $k=\ffi(N_2)-\ffi(N_1)\in\net$, on a\vv
$$d(a,a_k)\ie d\big(f(a),f(a_k)\big)\ie\ldots\ie
  d\Big(f^{\ffi(N_1)}(a),f^{\ffi(N_1)}(a_k)\Big)=d(a_{\ffi(N_1)},a_{\ffi(N_2)})
  <{\eps\s2}$$
et on majore de m\^eme $d(b,b_k)$.\new
On en d\'eduit alors\vv
$$d(a,b)\ie d\big(f(a),f(b)\big)=d(a_1,b_1)\ie d(a_2,b_2)\ie\ldots
  \ie d(a_k,b_k)\ie d(a,b)+\eps$$
et ceci pour tout $\eps>0$, donc $d\big(f(a),f(b)\big)=d(a,b)$.\msk
$\bullet$ L'ensemble image $f(K)$ est dense dans $K$~: on vient de voir
que toute orbite de $f$ est {\bf r\'ecurrente}, c'est-\`a-dire\vv
$$\a a\in K\quad\a \eps>0\quad \e k\in\net\qquad d\big(a,f^k(a)\big)<\eps$$
({\it l'orbite du point $a$ repasse aussi pr\`es de $a$ que l'on veut}), donc\vv
$$\a a\in K\quad \a\eps>0\quad \e y\in f(K)\qquad d(a,y)\ie\eps\;.$$
\par
$\bullet$ Enfin, $f$ est surjective puisque, $f$ \'etant continue car
1-lipschitzienne, l'ensemble $f(K)$ est compact, donc ferm\'e dans $K$.
Comme il est dense dans $K$, on a $f(K)=K$.


\bsk\hrule\hrule\bsk

{\bf EXERCICE 2 :}\msk
Soit $(K,d)$ un espace m\'etrique compact.\ssk
Si $u$ et $v$ sont deux \'el\'ements de $K^\nmat$, on pose\vv
$$\delta(u,v)=\sum_{n=0}^{\infty}{d(u_n,v_n)\s 2^n}\;.$$
\par
{\bf 1.} V\'erifier que $\delta$ est une distance sur $K^\nmat={\cal F}(\nmat,K)$
et qu'elle d\'efinit sur cet espace la ``topologie de la convergence simple'', c'est-\`a-dire~:
une suite $(v^p)_{p\in\nmat}$ d'\'el\'ements de $K^\nmat$ converge vers $\lam\in
K^\nmat$ si et seulement si, pour tout $n\in\nmat$, $\Lim{p}{\infty}v^p_n=\lam_n$.
\ssk
{\bf 2.} Montrer que $(K^\nmat,\delta)$ est un espace m\'etrique compact.
\ssk
{\bf 3.} Montrer que l'ensemble ${\cal P}$ des suites p\'eriodiques est
dense dans $(K^\nmat,\delta)$.\ssk
{\bf 4.} Montrer que, en posant $\gamma(u,v)=\sup_{n\in\nmat}d(u_n,v_n)$, on
d\'efinit une distance sur $K^\nmat$, mais que l'espace m\'etrique
$(K^\nmat,\gamma)$ n'est pas compact.

\bsk
\cl{- - - - - - - - - - - - - - - - - - - - - - - - - - - - - -}
\bsk

{\bf 1.} L'espace m\'etrique $(K,d)$ est born\'e car compact~; soit $\Delta$
son diam\`etre. Si $u$ et $v$ sont deux \'el\'ements de $K$, la s\'erie
$\sum_n {d(u_n,v_n)\s2^n}$ est convergente, d'o\`u l'existence de $\delta(u,v)$.
L'axiome de s\'eparation, la sym\'etrie et l'in\'egalit\'e triangulaire se
laissent v\'erifier sans opposer de r\'esistance.\msk\sect
Montrons donc l'\'equivalence\vv
$$\Lim{p}{\infty}v^p=\lam\quad{\rm dans}\;(K^\nmat,\delta)\iff
  \a n\in\nmat\quad\Lim{p}{\ii}v_n^p=\lam_n\;.$$\sect
$\bullet$ Si $\Lim{p}{\infty}v^p=\lam\;$ dans $(K^\nmat,\delta)$, alors
$\Lim{p}{\infty}\delta(v^p,\lam)=0$ d'o\`u, clairement, pour tout $n\in\nmat$,
$\Lim{p}{\infty}{d(v_n^p,\lam_n)\s2^n}=0$, soit
$\Lim{p}{\infty}d(v_n^p,\lam_n)=0$, ce qu'il fallait d\'emontrer.
\ssk\sect
$\bullet$ Supposons $\;\a n\in\nmat\quad\Lim{p}{\ii}v_n^p=\lam_n$ et donnons-nous
$\eps>0$.\sect\quad
Il existe un entier $N$ tel que $\sum_{n=N+1}^{\infty}{1\s2^n}
\ie{\eps\s2\Delta}$, o� $\Delta$ est le diam\`etre de $(K,d)$.\sect\quad
Pour tout $n\ie N$ fix\'e, on a $\Lim{p}{\infty}d(v_n^p,\lam_n)=0$,
donc on peut trouver un entier $P_n$ tel que $p\se P_n\impl d(v_n^p,\lam_n)\ie
{\eps\s4}$.\sect\quad
Soit $P=\max_{0\ie n\ie N}P_n$. Pour tout $p\se P$, l'in\'egalit\'e $d(v_n^p,\lam_n)
\ie{\eps\s4}$ est r\'ealis\'ee pour tout $n\ie N$, donc\vv
$$\sum_{n=0}^N{d(v_n^p,\lam_n)\s 2^n}\ie{\eps\s4}\;\sum_{n=0}^N{1\s2^n}\ie{\eps\s2}
  \qquad\hbox{pour tout}\quad p\se P\;.$$
Ainsi, pour tout $p\se P$, on a\vv
$$\delta(v^p,\lam)=\sum_{n=0}^N{d(v_n^p,\lam_n)\s 2^n}+\sum_{n=N+1}^\infty{d(v_n^p,\lam_n)\s 2^n}
  \ie{\eps\s2}+{\eps\s2\Delta}\Delta=\eps\;.$$
On a ainsi prouv\'e que $\Lim{p}{\infty}\delta(v^p,\lam)=0$.\msk\sect
La distance $\delta$ sur $K^\nmat$ d\'efinit bien la topologie de la
convergence simple.

\bsk
{\bf 2.} Soit $(u^p)_{p\in\nmat}$ une suite d'\'el\'ements de $K^\nmat$. On va
en extraire une sous-suite convergente par le {\bf proc\'ed\'e diagonal}, ce
qui prouvera la compacit\'e de l'espace m\'etrique $(K^\nmat,\delta)$.\ssk\sect
$\triangleright$ De la suite $(u_0^p)_{p\in\nmat}$, \`a valeurs dans $K$, on peut extraire une
sous-suite convergente $(u_0^{\ffi_0(p)})_{p\in\nmat}$, de limite $\lam_0\in K$.
\sect
$\triangleright$ De la suite $(u_1^{\ffi_0(p)})_{p\in\nmat}$, \`a valeurs dans $K$, on peut extraire une
sous-suite convergente $(u_1^{\ffi_0\circ\ffi_1(p)})_{p\in\nmat}$, de limite $\lam_1\in K$.
\ssk\sect
$\triangleright$ Soit $k\in\net$. Supposons ainsi construites $k$ extractions $\ffi_0$, $\ffi_1$,
$\ldots$, $\ffi_{k-1}$ (applications strictement croissantes de $\nmat$
vers $\nmat$). De la suite $\lp u_k^{\ffi_0\circ\ffi_1\circ\ldots\circ\ffi_{k-1}(p)}\rp_{p\in\nmat}$,
\`a valeurs dans $K$, on peut extraire une sous-suite convergente
$\lp u_k^{\ffi_0\circ\ffi_1\circ\ldots\circ\ffi_{k-1}\circ\ffi_k(p)}\rp_{p\in\nmat}$,
de limite $\lam_k\in K$.\msk\sect
Pour tout $p\in\nmat$, posons maintenant $\psi(p)=\ffi_0\circ\ffi_1\circ\ldots
\circ\ffi_p(p)$. On a ainsi d\'efini une application $\psi$ de $\nmat$ vers $\nmat$. Elle
est strictement croissante car\vv
\begin{eqnarray*}
\psi(p+1) & = & \ffi_0\circ\ldots\circ\ffi_p\big(\ffi_{p+1}(p+1)\big)\se
                               \ffi_0\circ\ldots\circ\ffi_p(p+1)\qquad\hbox{\bf (1)}\\
                     & > & \ffi_0\circ\ldots\circ\ffi_p(p)=\psi(p)\qquad\qquad\qquad\qquad\qquad\qquad\;\;\hbox{\bf (2)}\;:
\end{eqnarray*}
\new\quad {\bf (1)} car $\ffi_{p+1}(p+1)\se p+1$ et $\ffi_0\circ\ldots\circ\ffi_p$ est croissante~;
\new\quad {\bf (2)} car $\ffi_0\circ\ldots\circ\ffi_p$ est strictement croissante.
\msk\sect
Posons maintenant $v^p=u^{\psi(p)}$~: $(v^p)_{p\in\nmat}$ est une suite extraite de
$(u^p)_{p\in\nmat}$.\ssk\sect
Pour tout $n\in\nmat$, on a $\Lim{p}{\infty}v_n^p=
\Lim{p}{\infty}u_n^{\psi(p)}=\lam_n$ car $(u_n^{\psi(p)})_{p>n}$ est une
suite extraite de la suite $\lp u_n^{\ffi_0\circ\ldots\circ\ffi_n(p)}\rp_{p\in\nmat}$,
qui converge vers $\lam_n$ dans $K$~: en effet, l'application\break
$p\mapsto\ffi_{n+1}\circ\ldots\circ\ffi_p(p)$ est strictement croissante
sur $\[ent n+1,\ii\[ent$ par un raisonnement analogue \`a celui fait ci-dessus
pour $\psi$.\msk\sect
De la question {\bf 1.}, on d\'eduit enfin que la suite $(v^p)_{p\in\nmat}$,
extraite de la suite $(u^p)_{p\in\nmat}$, converge vers $\lam$ dans l'espace
m\'etrique $(K^\nmat,\delta)$.

\bsk
{\bf 3.} Soit $u\in K^\nmat$, soit $\eps>0$. Comme en {\bf 1.}, introduisons
un entier $N$ tel que $\sum_{n=N+1}^{\infty}{1\s 2^n}\ie{\eps\s\Delta}$, o\`u
$\Delta$ est le diam\`etre de $K$. Alors toute suite $v$ de $K^\nmat$ dont les
$N+1$ premiers termes co\"\i ncident avec ceux de $u$ ($v_n=u_n$ pour $n\in\[ent0,
N\]ent$) v\'erifie $\delta(u,v)\ie\eps$. Parmi ces suites, il en existe une
qui est $(N+1)$-p\'eriodique, donc l'ensemble ${\cal P}$ est dense dans $(K^\nmat,\delta)$.

\bsk
{\bf 4.} La distance $\gamma$ sur $K^\nmat={\cal F}(\nmat,K)$ d\'efinit la
``topologie de la convergence uniforme''. Montrons que $(K^\nmat,\gamma)$
n'est pas compact.\ssk\sect
Soient $a$ et $b$ deux \'el\'ements de $K$ distincts ({\it pour les pinailleurs, on
suppose $K$ non r\'eduit \`a un point}). Soit $d_0=d(a,b)$. Pour tout $p\in\nmat$,
soit $u^p$ la suite d'\'el\'ements de $K$ d\'efinie par $u_n^p=\system{
&a&\quad&{\rm si}\quad n=p\cr &b&\quad&{\rm sinon}\hfill\cr}$.\ssk\sect
Si $p$ et $q$ sont deux entiers distincts, on a $\gamma(u^p,u^q)=d_0$~; on
ne peut donc extraire de la suite $(u^p)_{p\in\nmat}$ aucune sous-suite
convergente pour la m\'etrique d\'efinie par $\gamma$.

\bsk
{\it Remarque}. On peut r\'epondre \`a la question {\bf 2.} en court-circuitant
la question {\bf 1.} Soit, en effet, $(u^p)_{p\in\nmat}$ une suite d'\'el\'ements
de $K^\nmat$. On d\'efinit des extractions $\ffi_0$, $\ldots$, $\ffi_n$, $\ldots$
comme dans la solution de la question {\bf 2.}, telles que, pour tout $n\in\nmat$,
la suite $\lp u_n^{\ffi_0\circ\ldots\circ\ffi_n(p)}\rp_{p\in\nmat}$ admette
une limite $\lam_n$ dans $K$.\ssk\sect
On veut montrer que l'\'el\'ement $\lam=(\lam_n)_{n\in\nmat}$
de $K^\nmat$ est valeur d'adh\'erence de la suite $(u^p)_{p\in\nmat}$. Pour cela,
plut\^ot que d'expliciter, par le proc\'ed\'e diagonal, une sous-suite de $(u^p)_p$
qui converge vers $\lam$ dans $(K^\nmat,\delta)$, on montre l'assertion\vv
$$\a\eps>0\quad\a P\in\nmat\quad\e p\se P\qquad\delta(u^p,\lam)\ie\eps\;.$$\sect
Donnons-nous donc $\eps>0$ et $P\in\nmat$, soit d'autre part un entier $N$
tel que $\;\sum_{n=N+1}^{\infty}{1\s2^n}\ie{\eps\s2\Delta}$, o� $\Delta$
est le diam\`etre de $K$. On a alors, pour tout $p\in\nmat$,\vv
$$\delta(u^p,\lam)\ie\sum_{n=0}^N{d(u_n^p,\lam_n)\s2^n}+{\eps\s2}\;.$$\sect
Il suffit donc de montrer que, pour un certain $p\se P$, on peut rendre
la premi\`ere somme inf\'erieure � ${\eps\s2}$ et, pour cela, il suffit que l'on
ait\vv
$$\a n\in\[ent0,N\]ent\qquad d(u_n^p,\lam_n)\ie{\eps\s4}\eqno\hbox{\bf (*)}\;.$$\sect
Or, pour tout $n\in\[ent0,N\]ent$, la suite
$\lp u_n^{\ffi_0\circ\ldots\circ\ffi_n(k)}\rp_{k\in\nmat}$ converge vers $\lam_n$,
donc aussi la suite $\lp u_n^{\ffi_0\circ\ldots\circ\ffi_N(k)}\rp_{k\in\nmat}$,
qui en est extraite. Il existe alors un entier naturel $k_0$ tel que
$$\a k\se k_0\quad\a n\in\[ent0,N\]ent\qquad d\lp u_n^{\ffi_0\circ\ldots\circ\ffi_N(k)},
  \lam_n\rp\ie{\eps\s4}\;.$$\sect
Comme $\Lim{k}{\infty}\ffi_0\circ\ldots\circ\ffi_N(k)=\ii$, il existe un entier
$k_1$ tel que $k\se k_1\impl\ffi_0\circ\ldots\circ\ffi_N(k)\se P$. En choisissant
$k\se\max\{k_0,k_1\}$ et en posant $p=\ffi_0\circ\ldots\circ\ffi_N(k)$, l'entier
$p$ est sup\'erieur \`a $P$ et v\'erifie bien {\bf (*)}.



\bsk
\hrule\hrule
\bsk

{\bf EXERCICE 3 :}\msk
Dans $E=\rmat^m$, soit $(F_n)_{n\in\nmat}$ une suite de sous-espaces affines
de dimension au plus \'egale � $m-2$.\ssk
Montrer que le compl\'ementaire de leur
r\'eunion $\;A=E\setminus\lp\bigcup_{n\in\nmat}F_n\rp\;$ est un ensemble
connexe par arcs.

\bsk
\cl{- - - - - - - - - - - - - - - - - - - - - - - - - - - - - - }
\bsk

Cet exercice utilise la propri\'et\'e de Baire.
Commen\c cons par quelques ``rappels'' sur les espaces m\'etriques complets.\eject
{\dotfill}\par
{\bf 1.} Si $(E,d)$ est un espace m\'etrique complet, alors toute suite $(B_n)_{n\in\nmat}$
de ferm\'es born\'es non vides, d\'ecroissante pour l'inclusion, et dont les diam\`etres
tendent vers z\'ero, a pour intersection (not\'ee $B$) un singleton ({\bf th\'eor\`eme des
ferm\'es embo\^\i t\'es}).\par
{\dotfill}
\msk\sect
En effet, pour tout $n\in\nmat$, soit $u_n\in B_n$, soit l'ensemble
$U_n=\{u_p\;;\;p\se n\}$. On a $U_n\subset B_n$, donc le diam\`etre de
l'ensemble $U_n$ tend vers z\'ero lorsque $n$ tend vers $\ii$, ce qui signifie
que la suite $(u_n)$ est de Cauchy. Elle converge donc vers un \'el\'ement $l$ de $E$.
\ssk\sect
Mais, pour tout $n$, on a $l\in\ov{U_n}\subset\ov{B_n}=B_n$, donc $l\in B$.
\ssk\sect
Par ailleurs, si $x\in E$ est distinct de $l$, alors on peut trouver un entier
$n$ tel que\break ${\rm diam}(B_n)<d(x,l)$ donc tel que $x\not\in B_n$, donc $x\not\in B$.\ssk\sect
En conclusion, $B=\bigcap_{n\in\nmat}B_n=\{l\}$.

\bsk
{\dotfill}\par
{\bf 2.} Tout espace m\'etrique complet v\'erifie la {\bf propri\'et\'e de Baire}, c'est-\`a-dire~:
toute intersection d\'enombrable $\bigcap_{n\in\nmat}\Omega_n$ d'ouverts denses
est un ensemble dense dans $E$.\par
{\dotfill}
\msk\sect
Posons $\Omega=\bigcap_{n\in\nmat}\Omega_n$.\ssk\sect
Soit $U$ un ouvert non vide de $E$, il suffit de montrer que $\Omega\inter U\not=\emptyset$.\ssk\sect
L'ouvert $\Omega_0$ \'etant dense dans $E$, il existe une boule ferm\'ee $B_0$
(de diam\`etre $\delta_0>0$) incluse dans $\Omega_0\inter U$.\ssk\sect
L'ouvert dense $\Omega_1$ rencontre l'ouvert non vide $\interieur{B_0}$,
l'intersection $\interieur{B_0}\inter\Omega_1$ contient donc une boule
ferm\'ee $B_1$ de diam\`etre $\delta_1>0$ et on peut toujours supposer que
$\delta_1<{\delta_0\s2}$.\ssk\sect
Par r\'ecurrence, on construit une suite $(B_n)$ de boules ferm\'ees, de
diam\`etres $\delta_n$ avec\break $0<\delta_n<{\delta_{n-1}\s2}$ pour tout $n\in\net$,
v\'erifiant $B_n\subset\interieur{B_{n-1}}\inter\Omega_n\subset B_{n-1}$. On a alors
$\Lim{n}{\infty}\delta_n=0$, donc $\bigcap_{n\in\nmat}B_n$
est un singleton $\{x\}$ d'apr\`es {\bf 1.}
L'\'el\'ement $x$ de $E$ est alors dans $\Omega\inter U$, ce qui ach\`eve
la d\'emonstration.

\bsk
{\dotfill}\par
{\bf 3.} Dans un espace m\'etrique complet, toute r\'eunion d\'enombrable
de ferm\'es d'int\'erieurs vides a un int\'erieur vide.\par
{\dotfill}
\msk\sect
C'est la propri\'et\'e ``duale'' de la pr\'ec\'edente (elle s'en d\'eduit par passage
au compl\'ementaire puisqu'une partie de $E$ a un int\'erieur vide si et seulement
si son compl\'ementaire est dense dans $E$).


\eject
{\dotfill}\par
{\bf 4.} Attaquons maintenant l'exercice proprement dit!\par
{\dotfill}
\msk\sect
Nous allons montrer que deux \'el\'ements quelconques de $A$ peuvent \^etre joints
(dans $A$) par une ligne bris\'ee form\'ee de deux segments, autrement dit\vv
$$\a(x,y)\in A^2\quad\e z\in A\qquad [x,z]\union[z,y]\subset A\;.$$\sect
Soient $x$ et $y$ deux \'el\'ements de $A$. Pour tout $n\in\nmat$, il existe\ssk\new
- un hyperplan affine $H_n$ contenant $x$ et $F_n$~;\ssk\new
- un hyperplan affine $H'_n$ contenant $y$ et $F_n$.\ssk\sect
Tout hyperplan affine de $E$ est un ferm\'e d'int\'erieur vide (``{\bf ensemble rare}'')
donc, d'apr\`es {\bf 3.}, l'ensemble $\;M=\lp\bigcup_{n\in\nmat}H_n\rp\union
\lp\bigcup_{n\in\nmat}H'_n\rp\;$ est un ensemble d'int\'erieur vide (``{\bf ensemble
maigre}'', c.\`a.d. union d\'enombrable de ferm\'es d'int\'erieur vide). Son
compl\'ementaire $E\setminus M$ est donc dense dans $E$ et il est donc non vide.
\ssk\sect
Soit donc $z\in E\setminus M$, soit le segment $S=[x,z]$~; alors $S$ ne rencontre
aucun des sous-espaces affines $F_n$
(si on avait un point $a$ dans $S\inter F_n$, on aurait alors $x\in H_n$, $a\in H_n$
avec $x\not=a$, donc la droite affine $(ax)$ serait incluse dans $H_n$ donc
le point $z$, qui appartient \`a cette droite, serait dans $H_n$, absurde).\ssk\sect
De m\^eme, le segment $S'=[z,y]$ ne rencontre aucun des $F_n$ et $S\union S'\subset
A$, ce qu'il fallait d\'emontrer.


\bsk
\hrule
\hrule
\bsk

{\bf EXERCICE 4 :}\msk
\def\K{{\cal K}}

Soit $\K$ l'ensemble des parties compactes non vides de $\cmat$.\ssk
Pour $F\in\K$ et $\eps>0$, on note $V_{\eps}(F)=\{z\in\cmat\;|\;d(z,F)\ie\eps\}$\quad
($\eps$-voisinage de $F$).\ssk
Pour $F\in\K$ et $G\in\K$, on pose\vv
$$\delta(F,G)=\min\{\eps\se0\;|\;F\subset V_{\eps}(G)\quad{\rm et}\quad
  G\subset V_{\eps}(F)\}\;.$$\par
{\bf a.} V\'erifier l'existence de $\delta(F,G)$.\ssk
{\bf b.} Montrer que $\delta$ est une distance sur $\K$.\ssk
{\bf c.} Soit $(G_n)$ une suite d'\'el\'ements de $\K$, d\'ecroissante pour l'inclusion.
Montrer que, dans l'espace m\'etrique $(\K,\delta)$, on a $\;\Lim{n}{\infty}G_n=
\bigcap_{n\in\nmat}G_n$.\ssk
{\bf d.} Montrer que $(\K,\delta)$ est un espace m\'etrique complet.

\msk
\cl{- - - - - - - - - - - - - - - - - - - - - - - - - - - - - -}
\msk


{\bf a.} Les parties $F$ et $G$ \'etant born\'ees, l'ensemble de r\'eels\vv
$$I_{F,G}=\{\eps\se0\;|\;F\subset V_{\eps}(G)\quad{\rm et}\quad
  G\subset V_{\eps}(F)\}$$
est non vide~; il est minor\'e par $0$, donc admet une
borne inf\'erieure $\delta$.\new {\it Il est alors clair que $I_{F,G}$ est, soit
l'intervalle $]\delta,\ii[$, soit l'intervalle $[\delta,\ii[$}.\ssk
\sect
Pour tout $\alpha>0$, on a $\delta+\alpha\in I_{F,G}$, donc
$\;F\subset V_{\delta+\alpha}(G)$, donc $\;F\subset\bigcap_{\alpha>0}
V_{\delta+\alpha}(G)=V_{\delta}(G)\;$ et, de m�me, $G\subset V_{\delta}(F)$.
Finalement, $\delta\in I_{F,G}$ et $\delta=\delta(F,G)=\min I_{F,G}$.
\ssk\sect
{\it Les lecteurs aimant jongler avec les inf et les sup v\'erifieront que l'on
peut aussi \'ecrire\vv
$$\delta(F,G)=\max\{\max_{x\in F}d(x,G),\max_{y\in G}d(y,F)\}\;,$$
les autres feront un dessin, ce qui est largement aussi instructif}.

\bsk
{\bf b.} Si $F$ est un ferm\'e de $\cmat$, on a $V_0(F)=\ov{F}=F$, donc\vv
$$\delta(F,G)=0\iff F\subset G\quad{\rm et}\quad G\subset F\iff F=G\;.$$\sect
La sym\'etrie $\delta(G,F)=\delta(F,G)$ est imm\'ediate.\ssk\sect
Pour l'in\'egalit\'e triangulaire, notons tout d'abord que, pour tout $K\in\K$ et
tous $\alpha>0$, $\beta>0$, on a $\;V_{\alpha}\big(V_{\beta}(K)\big)\subset
V_{\alpha+\beta}(K)$. Alors, soient $F$, $G$, $H$ trois \'el\'ements de $\K$,
posons $\alpha=\delta(F,G)$ et $\beta=\delta(G,H)$~; on a $\;F\subset V_{\alpha}(G)\;$
et $\;G\subset V_{\beta}(H)\;$ d'o\`u\vv
$$F\subset V_{\alpha}(G)\subset V_{\alpha}\big(V_{\beta}(H)\big)\subset
  V_{\alpha+\beta}(H)\;.$$
De m\^eme, $H\subset V_{\alpha+\beta}(F)$, donc $\delta(F,H)\ie\alpha+\beta$,
ce qu'il fallait d\'emontrer.
\ssk\sect
{\it Remarque. L'espace m\'etrique $\cmat$ \'etant en fait une partie convexe
d'un e.v.n., le lecteur se convaicra ais\'ement de l'\'egalit\'e $\;V_{\alpha}
\big(V_{\beta}(K)\big)=V_{\alpha+\beta}(K)\;$ pour tout compact $K$ non vide.}\msk\sect
La distance $\delta$ est la {\bf distance de Hausdorff}.

\bsk
{\bf c.} Posons $G=\bigcap_{n\in\nmat}G_n$.\ssk\sect
On a $G\in\K$~: en effet, $G$ est une intersection de ferm\'es born\'es, c'en est
donc encore un.
Par ailleurs, il est classique que {\it toute suite d\'ecroissante de compacts
non vides a une intersection non vide}~: en effet, si, pour tout $n$, on se
donne $x_n\in G_n$, la suite $(x_n)$, dont tous les \'el\'ements appartiennent
au compact $G_0$, admet une valeur d'adh\'erence $x=\Lim{n}{\infty}x_{\phi(n)}$.
Pour tout $n$, $x$ est alors la limite de la suite $\big(x_{\ffi(k)}\big)_{k\se n}$
d'\'el\'ements du ferm\'e $G_{\ffi(n)}$, donc $x\in\bigcap_{n\in\nmat}G_{\ffi(n)}=G$.\ssk
\ssk\sect
Montrons que $\Lim{n}{\ii}G_n=G$ dans l'espace m\'etrique $(\K,\delta)$, c'est-\`a-dire
$\Lim{n}{\ii}\delta(G_n,G)=0$.\break On a d\'ej\`a $G\subset G_n$ pour tout $n$.
Ensuite, si on se donne $\eps>0$, il existe $N$ tel que\break $G_N\subset V_{\eps}(G)$~:
sinon, pour tout $n$, on pourrait trouver $x_n\in G_n$ tel que $x_n\not\in
V_{\eps}(G)$, c'est-\`a-dire $d(x_n,G)>\eps$~; la suite $(x_n)$, \`a valeurs dans
le compact $G_0$, admet une valeur d'adh\'erence $x$ qui v\'erifie alors, par
passage \`a la limite dans l'in\'egalit\'e, $d(x,G)\se\eps$ donc $x\not\in G$, ce qui
est absurde puisque, pour tout $n$, $x$ est valeur d'adh\'erence de la suite
$(x_p)_{p\se n}$ \`a valeurs dans le compact $G_n$ donc $x\in G_n$. La d\'ecroissance
de la suite $(G_n)$ fait alors que $G_n\subset V_{\eps}(G)$ pour tout $n\se N$,
donc $\delta(G_n,G)\ie\eps$ pour $n\se N$, ce qu'il fallait d\'emontrer.
\msk

{\bf d.} Soit $(F_n)_{n\in\nmat}$ une suite de Cauchy dans l'espace
m\'etrique $(\K,\delta)$. Pour tout $n$, posons
$$G_n=\ov{\bigcup_{k\se n}F_k}\;,\quad{\rm puis}\quad
  G=\bigcap_{n\in\nmat}G_n\;.$$
\sect
Chaque $G_n$ est ferm\'e, non vide car il contient $F_n$, il est born\'e
car, la suite $(F_n)$ \'etant de Cauchy,\vv
$$\e N\in\nmat\quad k\se N\impl F_k\subset V_1(F_N)\;.$$
On a donc $G_n\in\K$ pour tout $n$.
Enfin, la suite $(G_n)$ est d\'ecroissante pour l'inclusion.\ssk
\sect
Montrons que $\Lim{n}{\infty}\delta(F_n,G_n)=0$, ce qui ach\'evera la d\'emonstration~:
on sait en effet que $\Lim{n}{\infty}G_n=G$, il en r\'esultera que la suite $(F_n)$
converge vers $G$ dans $(\K,\delta)$.
\ssk\sect
Si on se donne $\eps>0$, on peut trouver $N$ tel que $\delta(F_p,F_q)\ie\eps$
pour tous $p\se N$, $q\se N$. Pour $n\se N$, on a alors $F_k\subset
V_{\eps}(F_n)$ pour tout $k\se n$, ce qui entra\^\i ne $G_n\subset V_{\eps}(F_n)$~;
comme, par ailleurs, $F_n\subset G_n$, on a $\delta(F_n,G_n)\ie\eps$
pour tout $n\se N$, ce qu'il fallait prouver.


\bsk
\hrule
\hrule
\bsk

{\bf EXERCICE 5 :}\msk
{\bf Rel\`evements et hom\'eomorphismes du cercle}\bsk
On note ${\cal U}$ le cercle unit\'e dans le plan complexe. On note $\eps:\rmat
\vers{\cal U}$ le morphisme de groupe $\theta\mapsto\eps(\theta)=e^{2i\pi\theta}$.
\msk
{\bf 1. Rel\`evement d'une application continue de $[a,b]$ vers ${\cal U}$}\ssk\sect
Soient $a$ et $b$ deux r\'eels avec $a<b$. Soit $\ffi:[a,b]\vers{\cal U}$
une application continue. Montrer qu'il existe une application $\Phi:[a,b]
\vers\rmat$, continue, telle que $\ffi=\eps\circ\Phi$.\ssk\sect
Y a-t-il unicit\'e de $\Phi$~?\msk
{\bf 2. Rel\`evement d'une application continue de ${\cal U}$ vers ${\cal U}$}\ssk\sect
Soit $f:{\cal U}\vers{\cal U}$ une application continue. Montrer l'existence
d'une application $F:\rmat\vers\rmat$, continue, telle que\vvv
$$f\circ\eps=\eps\circ F\;.$$\sect
Une telle application $F$ est appel\'ee un rel\`evement de $f$.\msk
{\bf 3.} Soit $f:{\cal U}\vers{\cal U}$, continue. Comparer deux rel\`evements de $f$.
V\'erifier que le nombre $F(1)-F(0)$ est un entier relatif, qui ne d\'epend pas du choix
du rel\`evement $F$ de $f$~; on note ce nombre $N(f)$, c'est le {\bf nombre de
rotation} de $f$.\msk
{\bf 4.} Calculer $N(f)$ lorsque $f:z\mapsto\omega z^k$ avec $\omega\in{\cal U}$
et $k\in\zmat$.\msk
{\bf 5.} Soient $f,g:{\cal U}\vers{\cal U}$ continues. Montrer que\vv
$$N(g\circ f)=N(g)\>N(f)\;.$$\par
{\bf 6.} Soit $f:{\cal U}\vers{\cal U}$ un hom\'eomorphisme. Quelles sont les
valeurs possibles de $N(f)$~?\ssk\sect
Montrer que tout rel\`evement $F$ de $f$ est alors un hom\'eomorphisme de $\rmat$
sur lui-m\^eme.

\msk
\cl{- - - - - - - - - - - - - - - - - - - - - - - - - - - - - - -}
\msk

{\bf 1.} Notons tout d'abord que, si une application $h:I\vers{\cal U}$,
continue sur un intervalle $I$ de $\rmat$, ne recouvre pas le cercle ${\cal U}$
tout entier, alors elle admet un rel\`evement $H$~: en effet, soit $u_0=e^{2i\pi\theta_0}$
un \'el\'ement de ${\cal U}$ n'appartenant pas \`a l'image $h(I)$, alors l'application\break
$\alpha:{\cal U}\setminus\{u_0\}\vers]\theta_0,\theta_0+1[$ qui, \`a tout point
$u\in{\cal U}\setminus\{u_0\}$, associe l'unique r\'eel $\theta\in]\theta_0,\theta_0+1[$
tel que $e^{2i\pi\theta}=u$ est une ``d\'etermination continue de l'argument'' sur
${\cal U}\setminus\{u_0\}$ (c'est, plus pr\'ecis�ment, un hom\'eomorphisme de
${\cal U}\setminus\{u_0\}$ vers $]\theta_0,\theta_0+1[$ dont la r\'eciproque
est une restriction de $\eps$)
et l'application $H=\alpha\circ h$ r\'epond \`a la question.
On peut donc relever toute application continue $h$ d'un intervalle $I$
de $\rmat$ vers ${\cal U}$ lorsque le diam\`etre de l'ensemble-image $h(I)$
est strictement inf\'erieur \`a 2.\msk\sect
L'application $\ffi$, continue sur $[a,b]$, est uniform\'ement continue. Il existe
donc un $\alpha>0$ tel que\vv
$$\a(x,y)\in[a,b]^2\qquad |x-y|\ie\alpha\impl|\phi(x)-\phi(y)|\ie1\;.$$
Cela permet de construire une subdivision $a=c_0<c_1<\ldots<c_{n-1}<c_n=b$ de
l'intervalle $[a,b]$ telle que, pour tout $k\in\[ent0,n-1\]ent$, l'ensemble
$\phi([c_k,c_{k+1}])$ ait un diam\`etre au plus \'egal \`a 1 (il suffit que le pas de
la subdivision soit inf\'erieur \`a $\alpha$). Pour tout $k\in\[ent0,n-1\]ent$,
la restriction $\ffi_k$ de $\ffi$ au segment $[c_k,c_{k+1}]$ admet donc un rel\`evement
$\Phi_k:[c_k,c_{k+1}]\vers\rmat$ (application continue telle que $\ffi_k=\eps
\circ\Phi_k$).\ssk\sect
Il reste \`a raccorder ces rel\`evements~: en chaque ``point de jonction'' $c_k$
($1\ie k\ie n-1$), le nombre $\Phi_k(c_k)-\Phi_{k-1}(c_k)$ est un entier relatif
$N_k$ puisque $\exp\big(2i\pi\Phi_k(c_k)\big)=\exp\big(2i\pi\Phi_{k-1}(c_k)\big)=\ffi(c_k)$.
Consid\'erons alors l'application $\Phi:[a,b]\vers\rmat$ d\'efinie par\ssk\new
$\bullet$ $\Phi(x)=\Phi_0(x)$ pour tout $x\in[c_0,c_1]$~;\ssk\new
$\bullet$ $\Phi(x)=\Phi_1(x)-N_1$ pour tout $x\in[c_1,c_2]$~;\ssk\new
{\dotfill\hfill}\ssk\new
$\bullet$ pour tout $k\in\[ent1,n-1\]ent$, $\Phi(x)=\Phi_k(x)-(N_1+N_2+\ldots+N_k)$
 sur $[c_k,c_{k+1}]$.\ssk\sect
L'application $\Phi$ ainsi construite est bien d\'efinie et continue sur $[a,b]$
et v\'erifie $\ffi=\eps\circ\Phi$.\msk\sect
Si $\Phi$ et $\Psi$ sont deux rel\`evements de $\ffi$, alors, pour tout $x\in[a,b]$,
on a $e^{2i\pi\big(\Psi(x)-\Phi(x)\big)}=1$~; la fonction $\Psi-\Phi$, continue,
est \`a valeurs dans $\zmat$, elle est donc constante.\ssk\sect
En conclusion, si $\Phi_0$ est un rel\`evement de $\ffi$ sur $[a,b]$ (il en existe), les
rel\`evements de $\ffi$ sur $[a,b]$ sont les fonctions $\Phi_0+m$, o�
$m$ est un entier relatif fix\'e.

\bsk
{\bf 2.} Soit $f:{\cal U}\vers{\cal U}$ continue. Posons $\ffi=f\circ\eps$.
Alors $\ffi$ est une application continue et\break 1-p\'eriodique de $\rmat$ vers ${\cal U}$.
La restriction $\psi$ de $\ffi$ au segment $[0,1]$ est continue et admet donc
un rel\`evement $\Psi$ ($\Psi:[0,1]\vers\rmat$ continue telle que $\psi=\eps\circ\Psi$).
Comme $\psi(0)=\psi(1)$, le nombre $\Psi(1)-\Psi(0)$ est un entier relatif $N$.
\msk\sect
D\'efinissons alors $F:\rmat\vers\rmat$ par\vv
$$\a k\in\zmat\quad\a x\in[k,k+1[\qquad F(x)=\Psi(x-k)+kN\;.$$
L'application $F$ est continue sur $\rmat$, il suffit de v\'erifier les raccordements
aux points entiers~:\vv
$$F(k^-)=\Psi(1)+(k-1)N=\Psi(0)+kN=F(k)=F(k^+)$$
et on a bien $\eps\circ F=\ffi=f\circ\eps$~: si $x\in[k,k+1[$, on a en effet\vv
$$(\eps\circ F)(x)=\eps\big(\Psi(x-k)\big)=\psi(x-k)=\ffi(x-k)=\ffi(x)$$
car $\ffi$ est 1-p\'eriodique.

\msk
{\bf 3.} Si $F$ et $G$ sont deux rel\`evements de $f$, alors $\eps\circ F=\eps\circ G$,
donc la fonction $F-G$, continue sur $\rmat$, est \`a valeurs enti\`eres, donc constante.
Ici encore, \'etant donn� un rel\`evement $F_0$ de $f$, tous les rel\`evements de $f$
sont les applications $F_0+m$, o\`u $m$ est un entier relatif.\ssk\sect
Le nombre $F(1)-F(0)$ ne d\'epend donc pas du choix de ce rel\`evement, et ne
d\'epend donc que de $f$.\msk\sect
{\it Remarque}~: si $F$ est un rel\`evement de $f$, alors on a
$F(x+1)-F(x)=N(f)$ pour tout r\'eel $x$~: en effet, de $\eps\circ F=f\circ\eps$,
on tire $e^{2i\pi F(x+1)}=e^{2i\pi F(x)}=f(e^{2i\pi x})$~; la fonction
$x\mapsto F(x+1)-F(x)$ est donc continue et \`a valeurs dans $\zmat$, donc constante.
On a donc aussi $F(x+k)-F(x)=k\>N(f)$ pour tout $x\in\rmat$ et tout
$k\in\zmat$.

\msk
{\bf 4.} Posons $\omega=e^{2i\pi\alpha}$. On peut choisir comme rel\`evement
de $f:z\mapsto \omega z^k$ l'application affine $F:\rmat\vers\rmat$,
$x\mapsto kx+\alpha$. Le nombre de rotation de $f$
est donc $\;N(f)=k$.

\msk
{\bf 5.} On v\'erifie imm\'ediatement que, si $F$ est un rel\`evement de $f$ et
$G$ un rel\`evement de $g$, alors $G\circ F$ est un rel\`evement de $g\circ f$, donc\vv
$$N(g\circ f)=G\big(F(1)\big)-G\big(F(0)\big)=
  G\big(F(0)+N(f)\big)-G\big(F(0)\big)=N(g)\cdot N(f)$$
d'apr\`es la remarque formul\'ee \`a la fin de la question {\bf 3.}

\msk
{\bf 6.} Soit $g=f^{-1}$ l'hom\'eomorphisme r\'eciproque de $f$. De la question
{\bf 5.}, on d\'eduit\vv
$$N(g)\>N(f)=N(g\circ f)=N(\id_{{}_{{\cal U}}})=1\;,$$
donc $N(f)$ est un \'el\'ement inversible de l'anneau $\zmat$, d'o\`u $N(f)\in\{-1,
1\}$. Ces deux valeurs sont effectivement possibles~: $N(f)=1$ avec $f=
\id_{{}_{{\cal U}}}$, et $N(f)=-1$ avec $f:e^{i\theta}\mapsto e^{-i\theta}$
(c'est-\`a-dire $f:z\mapsto{1\s z}$).\ssk\sect
Soit $F$ un rel\`evement de $f$, soit $G$ un rel\`evement de $g=f^{-1}$. Alors $G\circ F$
et $F\circ G$ sont des rel\`evements de $\id_{{}_{{\cal U}}}$ ({\it cf}. d\'ebut
de la question {\bf 5.}). Comme $\id_{\rmat}$ est un rel\`evement de
$\id_{{}_{{\cal U}}}$, d'apr\`es la question {\bf 3.}, il existe des entiers
relatifs $m$ et $n$ tels que\vv
$$F\circ G(x)=x+n\qquad{\rm et}\qquad G\circ F(x)=x+m\;.$$
L'application $F\circ G$ est surjective, donc $F$ est surjective. L'application
$G\circ F$ est injective, donc $F$ est injective. Le rel\`evement $F$ est donc
une bijection continue de $\rmat$ sur lui-m\^eme, donc un hom\'eomorphisme
({\it une bijection continue $H$ d'un intervalle $I$ de $\rmat$ sur un intervalle
$J$ de $\rmat$ est toujours strictement monotone et sa bijection r\'eciproque
est alors continue, $H$ est alors un hom\'eomorphisme de $I$ sur $J$}).



\bsk
\hrule\hrule
\bsk

{\bf EXERCICE 6 :}\msk
Soit $f:\rpe\vers\rmat$, continue, telle que\vv
$$\a x\in\rpe\qquad \Lim{n}{\ii}f(nx)=0\;.$$
Montrer que $\Lim{x}{\ii}f(x)=0$. {\it On pourra pour cela utiliser la propri\'et\'e de Baire}.

\msk
\cl{- - - - - - - - - - - - - - - - - - - - - - - - - - - - - -}
\msk

Pour un rappel de la propri\'et\'e de Baire, {\it cf.} exercice 3. Un \'enonc\'e possible est~:\ssk\new
{\it Dans un espace m\'etrique complet, toute r\'eunion d\'enombrable de ferm\'es d'int\'erieur vide est d'int\'erieur vide}.\msk
Bon, mais alors, vous allez me dire~: l'ensemble $\rpe$, muni de la m\'etrique induite par la distance usuelle de $\rmat$, n'est pas complet. Certes, mais je r\'epondrai que, si la compl\'etude est une notion ``m\'etrique'' (i.e. attach\'ee \`a une distance), la propri\'et\'e de Baire, elle, est ``topologique'' (i.e. conserv\'ee par hom\'eomorphisme), or l'application exponentielle est un hom\'eomorphisme de $\rmat$ (complet, donc de Baire) sur $\rpe$. La propri\'et\'e de Baire reste donc vraie dans $\rpe$.\bsk

Donnons-nous $\eps>0$. Pour tout entier naturel $p$, soit\vv
$$F_p=\{x\in\rpe\;;\;\a n>p\quad |f(nx)|\ie\eps\}\;.$$\par
Comme $f$ est continue, chaque $\;F_p=\bigcap_{n>p}\{x\in\rpe\;;\;|f(nx)|\ie\eps\}\;$ est un ferm\'e relatif de $\rpe$.\msk
Or, les $F_p$ recouvrent $\rpe$ puisque, pour tout $x\in\rpe$, on a $\Lim{n}{\ii}f(nx)=0$. La r\'eunion $\bigcup_{p\in\nmat}F_p$ est d'int\'erieur non vide, puisque c'est $\rpe$ tout entier. Par contraposition de la propri\'et\'e de Baire ci-dessus, l'un au moins des $F_p$ est d'int\'erieur non vide.\msk
Soit $p\in\nmat$ tel que $\interieur{F_p}\not=\emptyset$, il existe alors deux r\'eels $u$ et $v$ avec $0<u<v$ tels que $[u,v]\subset F_p$, ce qui signifie que\vv
$$\a n>p\quad \a x\in[u,v]\qquad |f(nx)|\ie\eps\;.$$
On a donc $|f(x)|\ie\eps$ pour tout $x$ appartenant \`a l'ensemble $V=\bigcup_{n>p}[nu,nv]$. Or, cet ensemble $V$ est un voisinage de $\ii$ puisque, pour $n$ assez grand, les intervalles $[nu,nv]$ se chevauchent~: plus pr\'ecis\'ement, si $N$ est un entier sup\'erieur \`a ${u\s v-u}$, alors, pour $n\se N$, on a $(n+1)u\ie nv$~; l'intervalle $[Nu,\ii[$ est donc inclus dans $V$.\msk
On a ainsi prouv\'e\vv
$$\a \eps>0\quad\e A\in\rpe\quad x\se A\impl |f(x)|\ie\eps\;,$$
c'est-\`a-dire $\Lim{x}{\ii}f(x)=0$.












\end{document}