\documentclass{article}
\begin{document}

\parindent=-8mm\leftskip=8mm
\def\new{\par\hskip 8.3mm}
\def\sect{\par\quad}
\hsize=147mm  \vsize=230mm
\hoffset=-10mm\voffset=0mm

\everymath{\displaystyle}       % \'evite le textstyle en mode
                                % math\'ematique

\font\itbf=cmbxti10

\let\dis=\displaystyle          %raccourci
\let\eps=\varepsilon            %raccourci
\let\vs=\vskip                  %raccourci


\frenchspacing

\let\ie=\leq
\let\se=\geq



\font\pc=cmcsc10 % petites capitales (aussi cmtcsc10)

\def\tp{\raise .2em\hbox{${}^{\hbox{\seveni t}}\!$}}%



\font\info=cmtt10




%%%%%%%%%%%%%%%%% polices grasses math\'ematiques %%%%%%%%%%%%
\font\tenbi=cmmib10 % bold math italic
\font\sevenbi=cmmi7% scaled 700
\font\fivebi=cmmi5 %scaled 500
\font\tenbsy=cmbsy10 % bold math symbols
\font\sevenbsy=cmsy7% scaled 700
\font\fivebsy=cmsy5% scaled 500
%%%%%%%%%%%%%%% polices de pr\'esentation %%%%%%%%%%%%%%%%%
\font\titlefont=cmbx10 at 20.73pt
\font\chapfont=cmbx12
\font\secfont=cmbx12
\font\headfont=cmr7
\font\itheadfont=cmti7% at 6.66pt



% divers
\def\euler{\cal}
\def\goth{\cal}
\def\phi{\varphi}
\def\epsilon{\varepsilon}

%%%%%%%%%%%%%%%%%%%%  tableaux de variations %%%%%%%%%%%%%%%%%%%%%%%
% petite macro d'\'ecriture de tableaux de variations
% syntaxe:
%         \variations{t    && ... & ... & .......\cr
%                     f(t) && ... & ... & ...... \cr
%
%etc...........}
% \`a l'int\'erieur de cette macro on peut utiliser les macros
% \croit (la fonction est croissante),
% \decroit (la fonction est d\'ecroissante),
% \nondef (la fonction est non d\'efinie)
% si l'on termine la derni\`ere ligne par \cr, un trait est tir\'e en dessous
% sinon elle est laiss\'ee sans trait
%%%%%%%%%%%%%%%%%%%%%%%%%%%%%%%%%%%%%%%%%%%%%%%%%%%%%%%%%%%%%%%%%%%

\def\variations#1{\par\medskip\centerline{\vbox{{\offinterlineskip
            \def\decroit{\searrow}
    \def\croit{\nearrow}
    \def\nondef{\parallel}
    \def\tableskip{\omit& height 4pt & \omit \endline}
    % \everycr={\noalign{\hrule}}
            \def\cr{\endline\tableskip\noalign{\hrule}\tableskip}
    \halign{
             \tabskip=.7em plus 1em
             \hfil\strut $##$\hfil &\vrule ##
              && \hfil $##$ \hfil \endline
              #1\crcr
           }
 }}}\medskip}   

%%%%%%%%%%%%%%%%%%%%%%%% NRZCQ %%%%%%%%%%%%%%%%%%%%%%%%%%%%
\def\nmat{{\rm I\kern-0.5mm N}}  
\def\rmat{{\rm I\kern-0.6mm R}}  
\def\cmat{{\rm C\kern-1.7mm\vrule height 6.2pt depth 0pt\enskip}}  
\def\zmat{\mathop{\raise 0.1mm\hbox{\bf Z}}\nolimits}
\def\qmat{{\rm Q\kern-1.8mm\vrule height 6.5pt depth 0pt\enskip}}  
\def\dmat{{\rm I\kern-0.6mm D}}
\def\lmat{{\rm I\kern-0.6mm L}}
\def\kmat{{\rm I\kern-0.7mm K}}

%___________intervalles d'entiers______________
\def\[ent{[\hskip -1.5pt [}
\def\]ent{]\hskip -1.5pt ]}
\def\rent{{\bf ]}\hskip -2pt {\bf ]}}
\def\lent{{\bf [}\hskip -2pt {\bf [}}

%_____d\'ef de combinaison
\def\comb{\mathop{\hbox{\large C}}\nolimits}

%%%%%%%%%%%%%%%%%%%%%%% Alg\`ebre lin\'eaire %%%%%%%%%%%%%%%%%%%%%
%________image_______
\def\im{\mathop{\rm Im}\nolimits}
%________d\'eterminant_______
\def\det{\mathop{\rm det}\nolimits} 
\def\Det{\mathop{\rm Det}\nolimits}
\def\diag{\mathop{\rm diag}\nolimits}
%________rang_______
\def\rg{\mathop{\rm rg}\nolimits}
%________id_______
\def\id{\mathop{\rm id}\nolimits}
\def\tr{\mathop{\rm tr}\nolimits}
\def\Id{\mathop{\rm Id}\nolimits}
\def\Ker{\mathop{\rm Ker}\nolimits}
\def\bary{\mathop{\rm bar}\nolimits}
\def\card{\mathop{\rm card}\nolimits}
\def\Card{\mathop{\rm Card}\nolimits}
\def\grad{\mathop{\rm grad}\nolimits}
\def\Vect{\mathop{\rm Vect}\nolimits}
\def\Log{\mathop{\rm Log}\nolimits}

%________GL_______
\def\GLR#1{{\rm GL}_{#1}(\rmat)}  
\def\GLC#1{{\rm GL}_{#1}(\cmat)}  
\def\GLK#1#2{{\rm GL}_{#1}(#2)}
\def\SO{\mathop{\rm SO}\nolimits}
\def\SDP#1{{\cal S}_{#1}^{++}}
%________spectre_______
\def\Sp{\mathop{\rm Sp}\nolimits}
%_________ transpos\'ee ________
%\def\t{\raise .2em\hbox{${}^{\hbox{\seveni t}}\!$}}
\def\t{\,{}^t\!\!}

%_______M gothL_______
\def\MR#1{{\cal M}_{#1}(\rmat)}  
\def\MC#1{{\cal M}_{#1}(\cmat)}  
\def\MK#1{{\cal M}_{#1}(\kmat)}  

%________Complexes_________ 
\def\Re{\mathop{\rm Re}\nolimits}
\def\Im{\mathop{\rm Im}\nolimits}

%_______cal L_______
\def\L{{\euler L}}

%%%%%%%%%%%%%%%%%%%%%%%%% fonctions classiques %%%%%%%%%%%%%%%%%%%%%%
%________cotg_______
\def\cotan{\mathop{\rm cotan}\nolimits}
\def\cotg{\mathop{\rm cotg}\nolimits}
\def\tg{\mathop{\rm tg}\nolimits}
%________th_______
\def\tanh{\mathop{\rm th}\nolimits}
\def\th{\mathop{\rm th}\nolimits}
%________sh_______
\def\sinh{\mathop{\rm sh}\nolimits}
\def\sh{\mathop{\rm sh}\nolimits}
%________ch_______
\def\cosh{\mathop{\rm ch}\nolimits}
\def\ch{\mathop{\rm ch}\nolimits}
%________log_______
\def\log{\mathop{\rm log}\nolimits}
\def\sgn{\mathop{\rm sgn}\nolimits}

\def\Arcsin{\mathop{\rm Arcsin}\nolimits}   
\def\Arccos{\mathop{\rm Arccos}\nolimits}  
\def\Arctan{\mathop{\rm Arctan}\nolimits}   
\def\Argsh{\mathop{\rm Argsh}\nolimits}     
\def\Argch{\mathop{\rm Argch}\nolimits}     
\def\Argth{\mathop{\rm Argth}\nolimits}     
\def\Arccotan{\mathop{\rm Arccotan}\nolimits}
\def\coth{\mathop{\rm coth}\nolimits}
\def\Argcoth{\mathop{\rm Argcoth}\nolimits}
\def\E{\mathop{\rm E}\nolimits}
\def\C{\mathop{\rm C}\nolimits}

\def\build#1_#2^#3{\mathrel{\mathop{\kern 0pt#1}\limits_{#2}^{#3}}} 

%________classe C_________
\def\C{{\cal C}}
%____________suites et s\'eries_____________________
\def\suiteN #1#2{(#1 _#2)_{#2\in \nmat }}  
\def\suite #1#2#3{(#1 _#2)_{#2\ge#3 }}  
\def\serieN #1#2{\sum_{#2\in \nmat } #1_#2}  
\def\serie #1#2#3{\sum_{#2\ge #3} #1_#2}  

%___________norme_________________________
\def\norme#1{\|{#1}\|}  
\def\bignorme#1{\left|\hskip-0.9pt\left|{#1}\right|\hskip-0.9pt\right|}

%____________vide (perso)_________________
\def\vide{\hbox{\O }}
%____________partie
\def\P{{\cal P}}

%%%%%%%%%%%%commandes abr\'eg\'ees%%%%%%%%%%%%%%%%%%%%%%%
\let\lam=\lambda
\let\ddd=\partial
\def\bsk{\vspace{12pt}\par}
\def\msk{\vspace{6pt}\par}
\def\ssk{\vspace{3pt}\par}
\let\noi=\noindent
\let\eps=\varepsilon
\let\ffi=\varphi
\let\vers=\rightarrow
\let\srev=\leftarrow
\let\impl=\Longrightarrow
\let\tst=\textstyle
\let\dst=\displaystyle
\let\sst=\scriptstyle
\let\ssst=\scriptscriptstyle
\let\divise=\mid
\let\a=\forall
\let\e=\exists
\let\s=\over
\def\vect#1{\overrightarrow{\vphantom{b}#1}}
\let\ov=\overline
\def\eu{\e !}
\def\pn{\par\noi}
\def\pss{\par\ssk}
\def\pms{\par\msk}
\def\pbs{\par\bsk}
\def\pbn{\bsk\noi}
\def\pmn{\msk\noi}
\def\psn{\ssk\noi}
\def\nmsk{\noalign{\msk}}
\def\nssk{\noalign{\ssk}}
\def\equi_#1{\build\sim_#1^{}}
\def\lp{\left(}
\def\rp{\right)}
\def\lc{\left[}
\def\rc{\right]}
\def\lci{\left]}
\def\rci{\right[}
\def\Lim#1#2{\lim_{#1\vers#2}}
\def\Equi#1#2{\equi_{#1\vers#2}}
\def\Vers#1#2{\quad\build\longrightarrow_{#1\vers#2}^{}\quad}
\def\Limg#1#2{\lim_{#1\vers#2\atop#1<#2}}
\def\Limd#1#2{\lim_{#1\vers#2\atop#1>#2}}
\def\lims#1{\Lim{n}{+\infty}#1_n}
\def\cl#1{\par\centerline{#1}}
\def\cls#1{\pss\centerline{#1}}
\def\clm#1{\pms\centerline{#1}}
\def\clb#1{\pbs\centerline{#1}}
\def\cad{\rm c'est-\`a-dire}
\def\ssi{\it si et seulement si}
\def\lac{\left\{}
\def\rac{\right\}}
\def\ii{+\infty}
\def\eg{\rm par exemple}
\def\vv{\vskip -2mm}
\def\vvv{\vskip -3mm}
\def\vvvv{\vskip -4mm}
\def\union{\;\cup\;}
\def\inter{\;\cap\;}
\def\sur{\above .2pt}
\def\tvi{\vrule height 12pt depth 5pt width 0pt}
\def\tv{\vrule height 8pt depth 5pt width 1pt}
\def\rplus{\rmat_+}
\def\rpe{\rmat_+^*}
\def\rdeux{\rmat^2}
\def\rtrois{\rmat^3}
\def\net{\nmat^*}
\def\ret{\rmat^*}
\def\cet{\cmat^*}
\def\rbar{\ov{\rmat}}
\def\deter#1{\left|\matrix{#1}\right|}
\def\intd{\int\!\!\!\int}
\def\intt{\int\!\!\!\int\!\!\!\int}
\def\ce{{\cal C}}
\def\ceun{{\cal C}^1}
\def\cedeux{{\cal C}^2}
\def\ceinf{{\cal C}^{\infty}}
\def\zz#1{\;{\raise 1mm\hbox{$\zmat$}}\!\!\Bigm/{\raise -2mm\hbox{$\!\!\!\!#1\zmat$}}}
\def\interieur#1{{\buildrel\circ\over #1}}
%%%%%%%%%%%% c'est la fin %%%%%%%%%%%%%%%%%%%%%%%%%%%

\def\boxit#1#2{\setbox1=\hbox{\kern#1{#2}\kern#1}%
\dimen1=\ht1 \advance\dimen1 by #1 \dimen2=\dp1 \advance\dimen2 by #1
\setbox1=\hbox{\vrule height\dimen1 depth\dimen2\box1\vrule}%
\setbox1=\vbox{\hrule\box1\hrule}%
\advance\dimen1 by .4pt \ht1=\dimen1
\advance\dimen2 by .4pt \dp1=\dimen2 \box1\relax}


\catcode`\@=11
\def\system#1{\left\{\null\,\vcenter{\openup1\jot\m@th
\ialign{\strut\hfil$##$&$##$\hfil&&\enspace$##$\enspace&
\hfil$##$&$##$\hfil\crcr#1\crcr}}\right.}
\catcode`\@=12
\pagestyle{empty}
\def\lap#1{{\cal L}[#1]}
\def\DP#1#2{{\partial#1\s\partial#2}}
\def\cala{{\cal A}}
\def\fhat{\widehat{f}}
\let\wh=\widehat
\def\ftilde{\tilde{f}}

% ********************************************************************************************************************** %
%                                                                                                                                                                                   %
%                                                                    FIN   DES   MACROS                                                                              %
%                                                                                                                                                                                   %
% ********************************************************************************************************************** %










\def\lap#1{{\cal L}[#1]}
\def\DP#1#2{{\partial#1\s\partial#2}}



\overfullrule=0mm


\cl{{\bf SEMAINE 14}}\msk
\cl{{\bf S\'ERIES ENTI\`ERES}}
\bsk

{\bf EXERCICE 1 :}\msk
Soit $D_n$ le nombre de partitions de l'ensemble $E_n=\[ent1,n\]ent=
\{1,2,\ldots,n\}$ pour tout $n\in\net$, avec $D_0=1$ par convention.
\ssk
{\bf 1.} Montrer que, pour tout $n\in\nmat$, on a $\;\;D_{n+1}=\sum_{k=0}^nC_n^kD_k$.\ssk
{\bf 2.} Montrer, pour tout $n\in\nmat$, la relation\vv
$$D_n={1\s e}\>\sum_{k=0}^{\infty}{k^n\s k!}\;.$$\par
{\bf 3.} Montrer que la s\'erie enti\`ere $\;\sum_{n\se0}{D_n\s n!}x^n\;$
a un rayon de convergence infini, et calculer sa somme.

\msk
\cl{- - - - - - - - - - - - - - - - - - - - - - - - - - - - - -}
\msk

{\bf 1.} On a $D_1=1=\sum_{k=0}^0C_0^kD_k$.\ssk\sect
Soit $n\se1$. Soit une partition de $E_{n+1}$.
Soit $k$ le cardinal du compl\'ementaire $F$ de la classe
de l'\'el\'ement $n+1$. Cet entier $k\in\[ent0,n\]ent$ \'etant fix\'e,
il y a $C_n^k$ fa\c cons de choisir cet ensemble $F$ (de $k$ \'el\'ements choisis dans
$E_n$) et, pour chaque choix de cet ensemble $F$, il y a $D_k$ partitions de
cet ensemble. Le nombre de partitions de $E_{n+1}$ est donc\vv
$$D_{n+1}=\sum_{k=0}^nC_n^kD_k\;.$$
\ssk
{\bf 2.} Pour tout $n\in\nmat$, posons $\;S_n={1\s e}\>\sum_{k=0}^{\infty}{k^n\s k!}$~:
la s\'erie est bien convergente par la r\`egle de d'Alembert~:\vv
$${(k+1)^n\s(k+1)!}\times{k!\s k^n}={1\s k+1}\lp1+{1\s k}\rp^n\Vers{k}{\infty}0\;.$$
Alors $S_0=1=D_0$ et, pour tout $n$ entier naturel,\vv
\begin{eqnarray*}
\sum_{k=0}^nC_n^kS_k & = & {1\s e}\>\sum_{k=0}^nC_n^k\lp\sum_{p=0}^{\infty}{p^k\s p!}\rp\\
                                & = & {1\s e}\>\sum_{p=0}^{\infty}\lp{1\s p!}\>\sum_{k=0}^nC_n^kp^k\rp\\
                                & = & {1\s e}\>\sum_{p=0}^{\infty}{(p+1)^n\s p!}
                                  = {1\s e}\>\sum_{p=0}^{\infty}{(p+1)^{n+1}\s(p+1)!}\\
                                & = & {1\s e}\>\sum_{p=1}^{\infty}{p^{n+1}\s p!}
                                  = {1\s e}\>\sum_{p=0}^{\infty}{p^{n+1}\s p!}
                                  = S_{n+1}
\end{eqnarray*}
(il n'y a pas de probl\`eme d'interversion de sommation puisque l'une des
deux sommes est finie).
La suite $(S_n)$ et la suite $(D_n)$ ont le m\^eme premier terme et v\'erifient
la m\^eme relation de r\'ecurrence, donc $D_n=S_n$ pour tout $n\in\nmat$.

\msk
{\bf 3.} On cherche \`a calculer (apr\`es avoir prouv\'e son existence) la somme
$\;f(x)={1\s e}\sum_{n=0}^{\ii}{x^n\s n!}\lp\sum_{k=0}^{\ii}{k^n\s k!}\rp$
pour $x\in\cmat$ donn\'e. Or, la suite double $\lp{k^nx^n\s k!\> n!}\rp_{(n,k)\in
\nmat^2}$ est sommable~: en effet,\ssk\new
- pour tout $k\in\nmat$ fix\'e, la s\'erie $\sum_{n}{k^n|x|^n\s k!\>n!}$ est
convergente, et $\sum_{n=0}^{\infty}{k^n|x|^n\s k!\>n!}={e^{k|x|}\s k!}$~;\ssk\new
- la s\'erie $\sum_{k\in\nmat}{e^{k|x|}\s k!}$ est convergente, de somme $e^{e^{|x|}}$.\ssk\sect
Cela prouve l'existence de $f(x)$ pour tout $x$ r\'eel (la s\'erie enti\`ere $\;\sum_n
{D_n\s n!}x^n\;$ a un rayon de convergence infini) et on peut intervertir les
sommations~: pour tout $x\in\cmat$,
\begin{eqnarray*}
f(x) & = & {1\s e}\>\sum_{n=0}^{\ii}{x^n\s n!}\lp\sum_{k=0}^{\ii}{k^n\s k!}\rp
                  = {1\s e}\>\sum_{k=0}^{\ii}{1\s k!}\lp\sum_{n=0}^{\ii}{(kx)^n\s n!}\rp\\ \nssk
                & = & {1\s e}\>\sum_{k=0}^{\ii}{(e^x)^k\s k!}
                  = {1\s e}\>e^{e^x}=e^{e^x-1}\;.
\end{eqnarray*}

Voici une fonction r\'ecursive MAPLE pour calculer les nombres $D_n$ (appel\'es
{\bf nombres de Bell})~:\msk

{\info
 >  bell:=proc(n::nonnegint) local k; option remember;
         if n=0   then 1
 else add(binome(n-1,k)*bell(k),k=0..n-1)
  fi
}

\bsk
\hrule
\bsk

{\bf EXERCICE 2 :}\msk
Soit $\;\sum_{n\se0}a_nz^n\;$ une s\'erie enti\`ere, de rayon de convergence $R>0$.\msk
Pour $z\in\cmat$ avec $|z|<R$, on pose $\;f(z)=\sum_{n=0}^{\infty}a_nz^n$.\msk
Soit $z\in\cmat$ avec $|z|<R$, soit $r$ un r\'eel tel que $|z|<r<R$. Calculer l'int\'egrale\vv
$$I=\int_0^{2\pi}{\Im f(r\>e^{it})\s r-ze^{-it}}\>dt\;.$$\par
Que peut-on dire d'une fonction $f$, d\'eveloppable en s\'erie enti\`ere de rayon de convergence $R>0$ s'il existe un r\'eel $r$ ($0<r<R$) tel que $f$ soit r\'eelle sur le cercle de centre $O$ et de rayon $r$~?

\msk
\cl{- - - - - - - - - - - - - - - - - - - - - - - - - - - - - - -}
\msk

Utilisons $\;\Im f(r\>e^{it})={1\s 2i}\>\big(f(r\>e^{it})-\ov{f(r\>e^{it})\>}\big)\;$ et calculons s\'epar\'ement les deux int\'egrales\ssk\new $J=\int_0^{2\pi}{f(r\>e^{it})\s r-ze^{-it}}\>dt\;$
et $\;J'=\int_0^{2\pi}{\ov{f(r\>e^{it})}\s r-ze^{-it}}\>dt$.

\msk
Comme $\left|{z\s r}\right|<1$, on peut d\'evelopper~: ${1\s r-ze^{-it}}={1\s r}\>\lp1-{z\s r}\>e^{-it}\rp^{-1}={1\s r}\>\sum_{k=0}^{\infty}\lp{z\s r}\rp^k\>e^{-ikt}$. On a donc, pour tout $t\in[0,2\pi]$,\vv
$${f(r\>e^{it})\s r-z\>e^{-it}}={1\s r}\>f(r\>e^{it})\cdot\sum_{k=0}^{\ii}\lp{z\s r}\rp^k\>e^{-ikt}\;,$$
la s\'erie (de fonctions de la variable r\'eelle $t$) \'etant normalement convergente sur $[0,2\pi]$ car $f$ est born\'ee ($|f|\ie M_r$) sur le cercle ${\cal C}(O,r)$ et $\;\left|f(r\>e^{it})\>\lp{z\s r}\rp^k\>e^{-ikt}\right|\ie M_r\>\left|{z\s r}\right|^k$.
\ssk
Int\'egrons donc terme \`a terme~: $J=\int_0^{2\pi}{f(r\>e^{it})\s r-ze^{-it}}\>dt={1\s r}\>\sum_{k=0}^{\ii}\lp{z\s r}\rp^kJ_k$, avec
$$J_k = \int_0^{2\pi}f(r\>e^{it})\>e^{-ikt}\>dt = \int_0^{2\pi}\Big(\sum_{n=0}^{\ii}a_nr^n\>e^{i(n-k)t}\Big)\>dt
                                  = \sum_{n=0}^{\ii}a_nr^n\>\int_0^{2\pi}e^{i(n-k)t}\>dt$$
car la s\'erie (de fonctions de $t$) $\sum_{n=0}^{\ii}a_nr^n\>e^{i(n-k)t}$ converge normalement sur $[0,2\pi]$. Mais $\;\int_0^{2\pi}e^{i(n-k)t}\>dt=\system{&0\quad&{\rm si }&\;n\not=k\cr
&2\pi\quad&{\rm si }&\;n=k\cr}$, donc $J_k=2\pi a_kr^k$ pour tout $k\in\nmat$ et
$$J=\int_0^{2\pi}{f(r\>e^{it})\s r-ze^{-it}}\>dt={2\pi\s r}\>\sum_{k=0}^{\ii}a_k z^k={2\pi\s r}\>f(z)\;.$$

\ssk
On calcule l'int\'egrale $J'$ de fa\c con analogue~: $J'=\int_0^{2\pi}{\ov{f(r\>e^{it})}\s r-ze^{-it}}\>dt={1\s r}\>\sum_{k=0}^{\ii}\lp{z\s r}\rp^kJ'_k$, avec
$$J'_k=\int_0^{2\pi}\ov{f(r\>e^{it})}\;e^{-ikt}\>dt = \sum_{n=0}^{\ii}\ov{a_n}\>r^n\>\int_0^{2\pi}e^{-i(n+k)t}\>dt$$
({\it justifications analogues pour les interversions de $\sum$ et $\int$}).\new Mais $\;\int_0^{2\pi}e^{-i(n+k)t}\>dt=\system{&0\quad&{\rm si }&\;(n,k)\not=(0,0)\cr
&2\pi\quad&{\rm si }&\;(n,k)=(0,0)\cr}$, donc il reste $\;J'={2\pi\s r}\>\ov{a_0}\;$ et finalement
$$I=\int_0^{2\pi}{\Im f(r\>e^{it})\s r-ze^{-it}}\>dt={1\s 2i}(J-J')={i\>\pi\s r}\>\big(\ov{a_0}-f(z)\big)\;.$$

Si $f$ est \`a  valeurs r\'eelles sur  le cercle ${\cal C}(O,r)$, on a, pour tout $z\in\cmat$ tel que $|z|<r$,\vv
$$\int_0^{2\pi}{\Im f(r\>e^{it})\s r-ze^{-it}}\>dt={i\>\pi\s r}\>\big(\ov{a_0}-f(z)\big)=0\;,$$
donc $f$ est constante~: $f(z)=\ov{a_0}$ sur le disque ouvert $D(O,r)$ de centre $O$ et de rayon $r$. Les coefficients $a_n$ ($n\in\net$) sont donc tous nuls et $f$ est une fonction constante r\'eelle. 


\bsk
\hrule
\bsk

{\bf EXERCICE 3 :}\msk
{\bf Probl\`eme des parenth\'esages de Catalan}\msk
On pose $P_0=0$, $P_1=1$ et, pour tout entier $n\se2$, $P_n$ est le nombre de fa\c cons de parenth\'eser l'expression $a_1a_2\cdots a_n$ (en conservant l'ordre des $a_i$), o\`u  les $a_i$ sont des symboles que l'on peut interpr\'eter comme des \'el\'ements d'un ensemble muni d'une loi de composition interne a priori non associative. Le nombre $P_n$ est le nombre de fa\c cons a priori diff\'erentes de calculer le ``produit'' $a_1a_2\cdots a_n$.
Ainsi,\ssk\sect
$\bullet$ $P_2=1$ : un seul parenth\'esage $a_1a_2$~;\ssk\sect
$\bullet$ $P_3=2$ : deux parenth\'esages $a_1(a_2a_3)$ et $(a_1a_2)a_3$~;\ssk\sect
$\bullet$ $P_4=5$ : cinq parenth\'esages $a_1(a_2(a_3a_4))$, $a_1((a_2a_3)a_4)$, $(a_1a_2)(a_3a_4)$, $(a_1(a_2a_3))a_4\;$ et $\;((a_1a_2)a_3)a_4$.\msk
{\bf 1.} Prouver la relation $\;P_n=\sum_{k=1}^{n-1}P_kP_{n-k}\;$ pour tout $n\se 2$.\msk
{\bf 2.} Montrer que la s\'erie enti\`ere $\;f(x)=\sum_{n=0}^{\ii}P_nx^n\;$ a un rayon de convergence $R$ non nul ({\it on ne demande pas de calculer $R$ pour le moment}).\msk
{\bf 3.} Calculer $f(x)$ pour $x\in\>]-R,R[\>$. En d\'eduire une expression simple de $P_n$.


\msk
\cl{- - - - - - - - - - - - - - - - - - - - - - - - - - - - - - -}
\msk

{\bf 1.}  L'observation des cinq parenth\'esages de $a_1a_2a_3a_4$, \'ecrits dans l'ordre o\`u ils sont list\'es dans l'\'enonc\'e, nous guide un peu. Si la loi n'est pas suppos\'ee associative, seul un produit de deux facteurs peut \^etre d\'efini sans ambigu\"\i t\'e en l'absence de parenth\`eses~: une \'ecriture correcte d'un produit de $n$ facteurs $a_1$, $\cdots$, $a_n$ pris dans cet ordre est donc n\'ecessairement, pour un certain $k$ variant de 1 \`a $n-1$, le produit de $a_1\cdots a_k$ (\'ecrit avec un certain parenth\'esage~: $P_k$ choix possibles) par $a_{k+1}\cdots a_n$ (\'ecrit avec un certain parenth\'esage~: $P_{n-k}$ choix possibles), d'o\`u la relation \`a d\'emontrer.

\msk
{\bf 2.} On obtient facilement la majoration $P_n\ie 4^n$~: en effet, un parenth\'esage de l'expression $a_1\cdots a_n$ est d\'etermin\'e par la position des parenth\`eses ouvrantes et fermantes~; or, il y a $n$ positions possibles pour les ouvrantes (avant chaque symbole $a_i$), donc au plus $2^n$ possibilit\'es pour l'ensemble des positions des parenth\`eses ouvrantes, et m\^eme chose pour les fermantes, d'o\`u une majoration grossi\`ere de $P_n$ par $4^n$. 
La s\'erie enti\`ere $\;\sum_nP_nx^n\;$ a donc un rayon de convergence $R$ au moins \'egal \`a ${1\s4}$.

\msk
{\bf 3.} La relation obtenue \`a la question {\bf 1.} fait penser \`a un produit de Cauchy~; plus pr\'ecis\'ement, pour $\>x\in]-R,R[\>$,\vv
$$f(x)^2  =  \Big(\sum_{p=0}^{\infty}P_px^p\Big)\Big(\sum_{q=0}^{\infty}P_qx^q\Big) = \sum_{n=0}^{\infty}\Big(\sum_{k=0}^nP_kP_{n-k}\Big)x^n\;.$$
Or, $\sum_{k=0}^nP_kP_{n-k}=P_n$ pour tout $n\in\nmat$... sauf pour $n=1$ puisque $\sum_{k=0}^1P_kP_{1-k}=0$ alors que $P_1=1$. On a donc\vv
$$f(x)^2=\sum_{n=2}^{\infty}P_nx^n=\sum_{n=0}^{\infty}P_nx^n-x=f(x)-x\;.$$
Pour tout $x\in\>]-R,R[\>$, le r\'eel $f(x)$ v\'erifie donc l'\'equation alg\'ebrique du second degr\'e $\;f(x)^2-f(x)+x=0$, de discriminant $\Delta=1-4x$. Ce discriminant doit \^etre positif, donc $x\ie{1\s 4}$, le rayon de convergence de la s\'erie est donc au plus \'egal \`a ${1\s4}$, donc $R={1\s 4}$. Par ailleurs, $f(x)={1\s2}\big(1\pm\sqrt{1-4x}\big)$. Comme la fonction $f$ est continue sur $]-R,R[$ avec $f(0)=0$, on conclut\vv
$$\a x\in\lci-{1\s4},{1\s4}\rci\qquad f(x)={1\s2}\big(1-\sqrt{1-4x}\big)\;.$$
Je laisse \`a l'\'eventuel lecteur le soin de v\'erifier que le d\'eveloppement en s\'erie enti\`ere de $\sqrt{1-u}$ est\vv
$$\sqrt{1-u}=1-\sum_{n=1}^{\infty}{(2n-2)!\s 2^{2n-1}\>n!\>(n-1)!}\;u^n\qquad (u\in\>]-1,1[\>)\;.$$
On a donc, pour tout $x\in\>\lci-{1\s4},{1\s4}\rci\>$, $\;f(x)=\sum_{n=1}^{\infty}{(2n-2)!\s n!\>(n-1)!}\;x^n$. Par identification, on a enfin, pour $n\se1$,\vv
$$P_n={(2n-2)!\s n!\>(n-1)!}={1\s n}\>C_{2n-2}^{n-1}\;.$$


\bsk
\hrule
\bsk

{\bf EXERCICE 4 :}\msk
Soit $\sum_{n\se0}a_n$ une s\'erie convergente (\`a termes complexes).\msk
{\bf 1.} Montrer que, pour tout r\'eel $C$ strictement positif, la s\'erie enti\`ere $\sum_{n\se0}a_nz^n$ converge uniform\'ement sur la partie du plan\vv
$${\cal D}_C=\{z\in\cmat\;;\;|1-z|\ie C\big(1-|z|\big)\}\;.$$\ssk
{\bf 2.} En d\'eduire que cette m\^eme s\'erie enti\`ere converge uniform\'ement sur l'enveloppe convexe ${\cal E}_r$ de l'ensemble $\{1\}\cup\ov{D}(O,r)$ pour tout $r\in\>]0,1[\>$.

\msk
\cl{- - - - - - - - - - - - - - - - - - - - - - - - - - - - - -}
\msk

{\bf 1.} Effectuons une transformation d'Abel~: en posant $A_p^q=\sum_{k=p}^qa_k$ pour $q\se p$, on a, pour $n>m$,\vv
$$\sum_{k=m}^na_kz^k=A_m^nz^n+\sum_{k=m}^{n-1}A_m^k(z^k-z^{k+1})\;.$$
On en tire\vv
$$\left|\sum_{k=m}^na_kz^k\right|\ie|A_m^n|\>|z|^n+\sum_{k=m}^{n-1}|A_m^k|\>|z^k-z^{k+1}|\;.$$
Or, pour tout $z\in{\cal D}_C$, on a $|z|\ie 1$ (${\cal D}_C$ est inclus dans le disque unit\'e ferm\'e et le seul \'el\'ement de ${\cal D}_C$ de module 1 est le nombre 1).\ssk\sect
Pour tout $m\in\nmat$ fix\'e, posons $K_m=\sup_{k\se m}|A_m^k|$ ($K_m$ existe et $\Lim{m}{\ii}K_m=0$ en vertu du crit\`ere de Cauchy puisque la s\'erie $\sum_na_n$ est convergente). Alors\vv
$$\left|\sum_{k=m}^na_kz^k\right|\ie K_m\;\Big(1+\sum_{k=m}^{n-1}|z^k-z^{k+1}|\Big)\;.$$
Or, on a, si $|z|\not=1$,\vv
$$\sum_{k=m}^{n-1}|z^k-z^{k+1}|=|1-z|\>\sum_{k=m}^{n-1}|z|^k=|1-z|\>|z|^m\>{1-|z|^{n-m}\s 1-|z|}\;,$$
quantit\'e que l'on peut majorer par ${|1-z|\s1-|z|}$, puis finalement par $C$, lorsque $z\in{\cal D}_C$\break (y compris pour $z=1$ \'evidemment). On a ainsi prouv\'e\vv
$$\a (m,n)\in\nmat^2\quad(m<n)\quad\a z\in{\cal D}_C\qquad \left|\sum_{k=m}^na_kz^k\right|\ie(1+C)K_m$$
avec $\Lim{m}{\ii}K_m=0$, autrement dit le crit\`ere de Cauchy uniforme est v\'erifi\'e par la s\'erie enti\`ere $\sum_na_nz^n$ sur ${\cal D}_C$, d'o\`u la convergence uniforme sur cette partie du plan.

\msk
{\bf 2.} Pour tout $C>0$, l'ensemble ${\cal D}_C$ est une partie convexe du plan~: en effet, si $z\in{\cal D}_C$, $z'\in{\cal D}_C$, $\lam\in\rplus$, $\mu\in\rplus$ avec $\lam+\mu=1$, alors
$\;|1-z|\ie C(1-|z|)$, $|1-z'|\ie C(1-|z'|)$, d'o\`u\vv
\begin{eqnarray*}
|1-(\lam z+\mu z')| & = & |\lam(1-z)+\mu(1-z')| \ie \lam\>|1-z|+\mu\>|1-z'|\\
                           & \ie & C\>\Big[\lam(1-|z|)+\mu(1-|z'|)\Big] = C\>\big(1-(\lam\>|z|+\mu\>|z'|)\big)\\
                           & \ie & C\>(1-|\lam z+\mu z'|)\;,
\end{eqnarray*}
donc $\lam z+\mu z'\in{\cal D}_C$.
\ssk\sect
Par ailleurs, si $|z|\ie r<1$, alors\vv
$${|1-z|\s 1-|z|}\ie{|1-z|\s 1-r}\ie {1+|z|\s 1-r}\ie{1+r\s 1-r}\;,$$
donc $z\in{\cal D}_C$ avec $C={1+r\s 1-r}$. Pour $C={1+r\s 1-r}$, l'ensemble ${\cal D}_C$, convexe, contient le disque ferm\'e $\ov{D}(O,r)$ et le nombre 1, donc contient l'ensemble ${\cal E}_r$. Comme la s\'erie enti\`ere $\sum_na_nz^n$ converge uniform\'ement sur ${\cal D}_C$, elle converge donc uniform\'ement sur ${\cal E}_r$~: c'est le {\bf th\'eor\`eme de Picard}.


\bsk
\hrule
\bsk

{\bf EXERCICE 5 :}\msk
Soit $\sum_na_nz^n$ une s\'erie enti\`ere, de rayon de convergence $R>0$, de somme $f(z)$. Pour tout $r\in\>]0,R[\>$, on pose $\;M(r)=\sup_{|z|=r}|f(z)|$.\msk
{\bf 1.} Montrer que $\;\a n\in\nmat\quad\a r\in[0,R[\quad|a_nr^n|\ie M(r)$.\msk
{\bf 2.} On suppose que $R=\ii$ et qu'il existe $k>0$ et $\rho>0$ tels que\vv
$$\a r>0\qquad r\se\rho\impl M(r)\ie e^{kr}\;.$$\sect
Montrer qu'il existe un entier naturel $N$ tel que\vv
$$\a n\in\net\qquad n\se N\impl|a_n|\ie\lp{ke\s n}\rp^n\;.$$
\par
{\bf 3.} 0n suppose toujours $R=\ii$. Montrer , dans $\ov{\rplus^{}}$, l'\'egalit\'e\vv
\def\limsup#1#2{\;\build{\rm lim\;sup}_{{#1}\vers {#2}}^{}}
$$\limsup{n}{\ii}\big(n\>\root n\of{|a_n|}\big)=e\cdot\limsup{r}{\ii}{\ln M(r)\s r}\;.$$
{\it Indication : on pourra d\'emontrer, puis utiliser l'in\'egalit\'e} $\;n^n\se e^{n-1}\>(n-1)!$

\msk
\cl{- - - - - - - - - - - - - - - - - - - - - - - - - - - - - - -}
\msk

{\bf 1.} Pour $r<R$ et $n\in\nmat$, on a\vv
$$\int_0^{2\pi}f(r\>e^{it})\>e^{-int}\>dt=\int_0^{2\pi}\big(\sum_{k=0}^{\infty}a_kr^k\>e^{i(k-n)t}\big)\>dt=
\sum_{k=0}^{\infty}a_kr^k\>\int_0^{2\pi}e^{i(k-n)t}\>dt=2\pi a_nr^n$$
car la s\'erie de fonctions $\;t\mapsto a_k r^k\>e^{i(k-n)t}\;$ converge normalement sur $[0,2\pi]$. On en d\'eduit la majoration\vv
$$|a_n|r^n={1\s2\pi}\>\left|\int_0^{2\pi}f(r\>e^{it})\>e^{-int}\>dt\right|\ie{1\s 2\pi}\cdot2\pi\;M(r)=M(r)\;.$$

{\bf 2.} Soit $n\in\net$. Pour tout $r\se\rho$, on a $|a_n|\ie{e^{kr}\s r^n}$, donc $|a_n|\ie\inf_{r\se\rho}{e^{kr}\s r^n}$. La fonction $\ffi_n:r\mapsto{e^{kr}\s r^n}$ est d\'ecroissante sur $\lci0,{n\s k}\rc$, puis croissante sur $\lc{n\s k},\ii\rci$ et atteint au point ${n\s k}$ un minimum de valeur $m_n=\ffi_n\lp{n\s k}\rp=\lp{ke\s n}\rp^n$. Si $\rho\ie{n\s k}$, c'est-\`a-dire si $n\se k\rho$, alors
$$\inf_{r\se\rho}{e^{kr}\s r^n}=\inf_{r\in[\rho,\ii[}\ffi_n(r)=\ffi_n\lp{n\s k}\rp=\lp{ke\s n}\rp^n\;,$$
donc en choisissant $N=E(k\rho)+1$, on a bien\vv
$$\a n\in\net\qquad n\se N\impl|a_n|\ie\lp{ke\s n}\rp^n\;.$$

{\bf 3.} Posons $\;\alpha=\limsup{n}{\ii}\big(n\>\root n\of{|a_n|}\big)\;$ et $\;\beta=\limsup{r}{\ii}{\ln M(r)\s r}$.
\msk\sect
$\bullet$ Supposons $\beta<\ii$ et montrons qu'alors $\alpha\ie\beta$ (si $\beta=\ii$, cette in\'egalit\'e est triviale). De la d\'efinition d'une limite sup\'erieure, il r\'esulte que\vv
$$\a\eps>0\quad\e\rho>0\qquad r>\rho\impl{\ln M(r)\s r}\ie\beta+\eps\;.$$
Pour $r>\rho$, on a donc $\;M(r)\ie e^{(\beta+\eps)r}$, donc (question {\bf 2.}) il existe un entier $N$ tel que\vv
$$n\se N\impl|a_n|\ie\lp{(\beta+\eps)\>e\s n}\rp^n\;,\quad\hbox{c'est-\`a-dire}\quad n\se N\impl n\root n\of{|a_n|}\ie e(\beta+\eps)\;.$$
On en d\'eduit que $\alpha\ie e(\beta+\eps)$. Cette in\'egalit\'e \'etant vraie pour tout $\eps>0$, on a donc $\;\alpha\ie e\>\beta$.
\msk\sect
$\bullet$ Supposons $\alpha<\ii$ et montrons $\beta\ie{\alpha\s e}$. On a\vv
$$\a \eps>0\quad\e N\in\nmat\qquad n\se N\impl n\root n\of{|a_n|}\ie\alpha+\eps\;,$$
donc $\;|a_n|\ie\lp{\alpha+\eps\s n}\rp^n\;$ pour $n\se N$.\ssk\new
Si $|z|=r$, alors $|f(z)|\ie\sum_{n=0}^{\infty}|a_n|\>r^n$, donc\vv
$$\a r\in\rpe\qquad M(r)\ie\sum_{n=0}^N|a_n|\>r^n+\sum_{n=N+1}^{\ii}\lp{\alpha+\eps\s n}\rp^n\>r^n\;.$$
De l'in\'egalit\'e $\;n^n\se e^{n-1}\>(n-1)!$ ({\it cf. ci-dessous}), on tire\vv
\begin{eqnarray*}
M(r) & \ie & \sum_{n=0}^N|a_n|\>r^n+e\>\sum_{n=N+1}^{\ii}\lp{\alpha+\eps\s e}\rp^n\>{r^n\s(n-1)!}
= \sum_{n=0}^N|a_n|\>r^n+(\alpha+\eps)r\>\sum_{n=N}^{\ii}{1\s n!}\>\lp{\alpha+\eps\s e}\>r\rp^n\\
& \ie & \sum_{n=0}^N|a_n|\>r^n+(\alpha+\eps)r\>e^{{}^{{\sst\alpha+\eps\sur\sst e}\sst r}}
= e^{{}^{{\sst\alpha+\eps\sur\sst e}\sst r}}\;\ffi_N(r)\;,
\end{eqnarray*}
en ayant pos\'e $\;\ffi_N(r)=(\alpha+\eps)r+e^{{}^{\sst -{\sst\alpha+\eps\sur\sst e}\sst r}}\cdot\sum_{n=0}^N|a_n|r^n$.\msk\new
Il en r\'esulte $\;{\ln M(r)\s r}\ie{\alpha+\eps\s e}+{1\s r}\>\ln\ffi_N(r)\;$ pour tout $r>0$. Mais $\ffi_N(r)\Equi{r}{\ii}(\alpha+\eps)r$, donc $\Lim{r}{\ii}{1\s r}\ln\ffi_N(r)=0$, donc $\limsup{r}{\ii}{\ln M(r)\s r}\ie{\alpha+\eps\s e}\;$ et ceci pour tout $\eps>0$, donc $\beta\ie{\alpha\s e}$.

\bsk
{\it D\'emonstration de $\;n^n\se e^{n-1}\>(n-1)!\;$~: cette in\'egalit\'e \'equivaut \`a $\;\ln\big[(n-1)!\big]\ie n\>\ln n-n+1$.\par Or, par comparaison \`a une int\'egrale, la fonction $\ln$ \'etant croissante, on a imm\'ediatement\vv
$$\ln\big[(n-1)!\big]=\sum_{k=1}^{n-1}\ln k\ie\sum_{k=1}^{n-1}\int_k^{k+1}\ln x\>dx=\int_1^n\ln x\>dx=n\>\ln n-n+1\;.$$}


\bsk
\hrule
\bsk


{\bf EXERCICE 6 :}\msk
Pour tout $x>0$, on pose $\;\Gamma(x)=\int_0^{\ii}e^{-t}\>t^{x-1}\>dt$.\msk
Pour tout $x>1$, on pose $\;\zeta(x)=\sum_{n=1}^{\ii}{1\s n^x}$.\msk
Enfin, la {\bf constante d'Euler} est d\'efinie par $\;\gamma=\Lim{n}{\infty}\lp\sum_{k=1}^n{1\s k}-\ln n\rp$.\msk
{\bf 1.} D\'emontrer la relation~: \qquad $\a x\in\rpe\qquad\Gamma(x)=\Lim{n}{\infty}\int_0^n\lp1-{t\s n}\rp^n\>t^{x-1}\>dt$.\msk
{\bf 2.} En d\'eduire~:\qquad$\a x\in\rpe\qquad \Gamma(x)=\Lim{n}{\ii}{n^x\;n!\s x(x+1)\cdots(x+n)}$.\msk
{\bf 3.} Soit $x\in]0,1[$. D\'emontrer la relation\vv
$$\ln\Gamma(x)=-\ln x-\gamma x+\sum_{k=2}^{\ii}{(-1)^k\s k}\>\zeta(k)\>x^k\;.$$\par
{\bf 4.} Prouver que $\;\gamma=\sum_{k=2}^{\ii}{(-1)^k\s k}\>\zeta(k)$.

\msk
\cl{- - - - - - - - - - - - - - - - - - - - - - - - - - - - - -}
\msk

{\bf 1.} Plus g\'en\'eralement, soit $f:\;]0,\ii[\vers\cmat$ une fonction continue telle que la
fonction\break $g:t\mapsto e^{-t}\>f(t)$ soit int\'egrable sur $\rmat_+^*$. Alors\vv
$$\int_0^{\ii}e^{-t}\;f(t)\;dt=\Lim{n}{\ii}\int_0^n\lp1-{t\s n}\rp
^n\;f(t)\;dt\;.$$
En effet, pour tout r\'eel $t$, on a $e^{-t}=\Lim{n}{\ii}
\lp1-{t\s n}\rp^n$. D\'efinissons, pour tout $n\in\nmat^*$, une
fonction $u_n:\;]0,\ii[\vers\rmat$ par\vv
$$u_n(t)\;=\system{&\lp1-{t\s n}\rp^n\quad&\hbox{si}\quad&0<t\le n\cr
                   &\hfill0\hfill         &\hbox{si}\quad&t>n\;.\cr}$$
Alors $u_n$ est continue sur $\rmat_+^*$ et la suite $(u_n)$ converge
simplement, sur $\rmat_+^*$, vers la fonction $t\mapsto e^{-t}$.\pn
En posant $g_n=u_n\cdot f$, on a une suite $(g_n)$ de fonctions continues
sur $\rmat_+^*$, convergeant simplement vers $g$ sur $\rmat_+^*$.
L'in\'egalit\'e classique $\;\ln\lp1-{t\s n}\rp\le-{t\s n}$, va\-lable pour
$t\in[0,n[\;$, montre que\vv
$$\a n\in\nmat^*\quad\a t\in\rmat_+^*\qquad 0\le u_n(t)\le e^{-t}\quad
\hbox{donc}\quad|g_n(t)|\le|g(t)|\;.$$
L'hypoth\`ese de domination est alors v\'erifi\'ee et le th\'eor\`eme de convergence
domin\'ee s'applique. Il suffit donc d'appliquer ce r\'esultat avec $\;f(t)=t^{x-1}$.

\msk
{\bf 2.} Le changement de variable $t=nu$ donne\vv
$$\int_0^n\lp1-{t\s n}\rp^n\>t^{x-1}\;dt =n^x\;\int_0^1(1-u)^n
                                                         \>u^{x-1}\;du
                                                    =n^x\;B(x,n+1)\;,$$
en notant $\;B(p,q)=\int_0^1u^{p-1}\>(1-u)^{q-1}\>du\;$ pour $p$ et $q$ r\'eels strictement positifs ({\bf int\'egrale eul\'erienne de premi\`ere esp\`ece}). La fonction $u\mapsto u^{p-1}(1-u)^{q-1}$
est bien int\'egrable sur $]0,1[$ et, pour tout $n\in\net$ et $x>0$, une int\'egration
par parties donne\vv
\begin{eqnarray*}
B(x,n+1) & = & \int_0^1u^{x-1}(1-u)^n\;du
                      =\lc(1-u)^n{u^x\s x}\rc_0^1+{n\s x}\;
                                 \int_0^1u^x(1-u)^{n-1}\;du\\
                    & = & {n\s x}\;B(x+1,n)\;.
\end{eqnarray*}
\`A partir de $B(x,1)=\int_0^1u^{x-1}\;du={1\s x}$ pour tout $x>0$, une
r\'ecurrence imm\'ediate donne
$$B(x,n)={(n-1)!\s x(x+1)(x+2)\cdots(x+n-1)}\;.$$
Finalement,\vv
$$\a x\in\rpe\quad\a n\in\net\qquad\int_0^n\lp1-{t\s n}\rp^n\>t^{x-1}\;dt={n^x\;n!\s x(x+1)\cdots(x+n)}\;,$$
d'o\`u le r\'esultat. 

\msk
{\bf 3.} On a $\;\ln n=\sum_{k=1}^n{1\s k}-\gamma+o(1)$.
De la relation obtenue \`a la question {\bf 2.}, on d\'eduit donc, pour tout $x>0$,\vv\vvvv
\begin{eqnarray*}
\ln\Gamma(x) & = & \Lim{n}{\ii}\lc x\ln n+\sum_{k=1}^n\ln k-
                              \sum_{k=0}^n\ln(x+k)\rc\\
                        & = & \Lim{n}{\ii}\lc x\ln n-\ln x-\sum_{k=1}^n
                              \ln\lp1+{x\s k}\rp\rc\\
                        & = & -\ln x+\Lim{n}{\ii}\lc x\>\lp\sum_{k=1}^n{1\s k}
                              -\gamma\rp-\sum_{k=1}^n\ln\lp1+{x\s k}\rp\rc\\
                        & = & -\ln x-\gamma x+\sum_{n=1}^{\ii}\lp{x\s n}-\ln\lp
                              1+{x\s n}\rp\rp\;.
\end{eqnarray*}
\sect
Si $x\in\>]0,1[\>$, alors ${x\s n}\in\>]0,1[$ pour tout $n\in\net$ et on peut d\'evelopper~:\vv
$${x\s n}-\ln\lp1+{x\s n}\rp=\sum_{k=2}^{\infty}(-1)^k\>{x^k\s k\>n^k}\;.$$
Cherchons \`a intervertir les sommations~; pour cela, v\'erifions que la famille $\lp(-1)^k\>{x^k\s k\>n^k}\rp_{n\se1,k\se2}$ est sommable~: en effet,\ssk\new 
- pour $n\se1$ fix\'e, on a $\;u_n=\sum_{k=2}^{\infty}{x^k\s k\>n^k}=-{x\s n}-\ln\lp1-{x\s n}\rp$~;\ssk\new
- la s\'erie de terme g\'en\'eral $u_n$ est convergente puisque $u_n\sim{x^2\s2n^2}$.\msk\sect
On peut donc \'ecrire\vv
$$\sum_{n=1}^{\ii}\lp{x\s n}-\ln\lp1+{x\s n}\rp\rp=\sum_{k=2}^{\infty}\lp\sum_{n=1}^{\infty}(-1)^k\>{x^k\s k\>n^k}\rp=\sum_{k=2}^{\ii}{(-1)^k\s k}\>\zeta(k)\>x^k\;,$$
ce qu'il fallait d\'emontrer.

\msk
{\bf 4.} L'identit\'e\vv
$$\ln\Gamma(x)+\ln x+\gamma x=\sum_{k=2}^{\ii}{(-1)^k\s k}\>\zeta(k)\>x^k$$
est valable pour tout $x\in\>]0,1[\>$. Faisons tendre $x$ vers 1. Le premier membre tend vers $\gamma$. Pour ce qui est du second membre, la convergence de la s\'erie $\sum_{k\se2}{(-1)^k\s k}\>\zeta(k)$, en vertu du crit\`ere pour les s\'eries altern\'ees (rappelons que $\>\Lim{x}{\ii}\zeta(x)=1$), assure la continuit\'e au point 1 de la fonction $\;f:x\mapsto\sum_{k=2}^{\infty}{(-1)^k\s k}\>\zeta(k)\>x^k$ ({\bf lemme d'Abel radial}), d'o\`u le r\'esultat.\msk\sect
{\it Plus conform\'ement au programme, la s\'erie d\'efinissant $f(x)$ v\'erifie, pour tout $x\in\>]0,1]$, le crit\`ere des s\'eries altern\'ees, ce qui permet de majorer le reste d'ordre $n$ en valeur absolue par ${\zeta(n+1)\s n+1}\>x^{n+1}$, donc a fortiori par ${\zeta(n+1)\s n+1}$ qui tend vers z\'ero ind\'ependamment de $x$, d'o\`u la convergence uniforme de cette s\'erie sur $]0,1]$ et donc la continuit\'e de $f$ sur cet intervalle.}


























\end{document}